\section{题目}

\startexercise
\begin{exercise}[series=exer]
  \item 设 $ \Gamma $ 是由球面 $ x^{2} + y^{2} + z^{2} = a^{2} $ 和平面 $ x + y + z = 0 $ 交成的圆周, 从第一卦限内看 $ \Gamma $, 它的方向是逆时针. 计算第二型曲线积分
  \begin{align*}
    \int_{\Gamma} z \dx* + x \dd{y} + y \dd{z}.
  \end{align*}
  \begin{hint}
    这里给出两种方法:
    \begin{method}
      \item 可利用 Stokes 公式转换为面积.
      \item 利用该圆的参数方程进行计算:
        \begin{align*}
          \begin{cases}
            x = a\qty(\frac{1}{\sqrt{6}}\cos \theta + \frac{1}{\sqrt{2}}\sin \theta)\\
            y = a\qty(-\frac{2}{\sqrt{6}}\cos \theta)\\
            z = a\qty(\frac{1}{\sqrt{6}}\cos \theta - \frac{1}{\sqrt{2}}\sin \theta)
          \end{cases}, \quad \theta\in [0, 2\pi],
        \end{align*}
    \end{method}
  \end{hint}
  \begin{answer}
    这里给出两种方法:
    \begin{method}
      \item 利用 Stokes 公式: 设被 $ \Gamma $ 围成的圆面为 $ \Sigma $, 由右手定则可知 $ \Sigma $ 的单位外法向量为 $ (1/\sqrt{3}, 1/\sqrt{3}, 1/\sqrt{3}) $ 那么就有
        \begin{align*}
          \int_{\Gamma} z \dx* + x \dd{y} + y \dd{z}
          & = \iint_{\Sigma}
            \begin{vmatrix}
              \dy\dz  & \dz\dx    & \dx\dy \\
              \pdv{x} & \pdv{y}   & \pdv{z} \\
              z       & x         & y
            \end{vmatrix} \\
          & = \iint_{\Sigma} \dy\dz + \dz\dx + \dx\dy \\
          & = \iint_{\Sigma} (1, 1, 1) \cdot \qty(\frac{1}{\sqrt{3}}, \frac{1}{\sqrt{3}}, \frac{1}{\sqrt{3}}) \dd{\sigma} \\
          &  = \sqrt{3} \iint_{\Sigma} \dd{\sigma} = \sqrt{3} S_{\Sigma} = \sqrt{3} \pi a^{2} 
        \end{align*}
        其中 $ S_{\Sigma} $ 为 $ \Sigma $ 的面积.
      \item 可以写出 $ \Gamma $ 的参数方程:
        \begin{align*}
          \begin{cases}
            x = a\qty(\frac{1}{\sqrt{6}}\cos \theta + \frac{1}{\sqrt{2}}\sin \theta)\\
            y = a\qty(-\frac{2}{\sqrt{6}}\cos \theta)\\
            z = a\qty(\frac{1}{\sqrt{6}}\cos \theta - \frac{1}{\sqrt{2}}\sin \theta)
          \end{cases}, \quad \theta\in [0, 2\pi],
        \end{align*}
        就可以直接计算第二型曲线积分:
        \begin{align*}
          \int_{\Gamma} z \dx* + x \dy* + y \dz* 
          & = \int_{0}^{2\pi} 
            \begin{aligned}[t]
              & \mathrel{\phantom{+}}a\qty(\frac{1}{\sqrt{6}}\cos \theta - \frac{1}{\sqrt{2}}\sin \theta) \cdot a \qty(-\frac{1}{\sqrt{6}}\sin\theta + \frac{1}{\sqrt{2}}\cos\theta) \\
              & + a\qty(\frac{1}{\sqrt{6}}\cos \theta + \frac{1}{\sqrt{2}}\sin \theta) \cdot a \qty(\frac{2}{\sqrt{6}}\sin\theta) \\
              & + a\qty(-\frac{2}{\sqrt{6}}\cos \theta) \cdot a \qty(-\frac{1}{\sqrt{6}}\sin\theta - \frac{1}{\sqrt{2}}\cos\theta) \dd{\theta}
            \end{aligned} \\
          & = \sqrt{3} \pi a^{2}.
        \end{align*}
    \end{method}
  \end{answer}
  \item 设 $ f(x) $ 在 $ [1, +\infty) $ 上可导, 且 $ \limit{x}{+\infty}f'(x) = +\infty $, 证明 $ f(x) $ 在 $ [1, +\infty) $ 上不一致连续.
  \begin{hint}
      利用 $ \limit{x}{+\infty}f'(x) = +\infty $ 与 Lagrange 中值定理得到不一致连续的定义.
  \end{hint}
  \begin{answer}
    由 $ \limit{x}{+\infty} f'(x) = +\infty $ 可知: 对任意 $ A > 0 $, 都存在 $ X > 0 $, 使得当 $ x > X $ 时, $ f'(x) > A $. 
    
    于是对于 $ \varepsilon_{0} = 1 $, 对任意的 $ \delta > 0 $, 取 $ A_{\delta} > 2/\delta $, 那么存在 $ X_{\delta} $, 使得当 $ x > X_{\delta} $ 时 $ f'(x) > A_{\delta} $. 于是对于 $ x_{1} > x_{2} > X_{\delta} $, 当 $ x_{1} - x_{2} = \delta/2 $ 时, 由 Lagrange 中值定理可知存在一个 $ \xi_{\delta} \in (x_{2}, x_{1}) $, 使得
    \begin{align*}
      \abs{f(x_{1}) - f(x_{2})} = f'(\xi_{\delta})\abs{x_{1} - x_{2}} > A_{\delta} \cdot \frac{\delta}{2} > 1,
    \end{align*}
    这就是 $ f(x) $ 不一直连续的定义.
  \end{answer}
  \item 设 $ \set{a_{n}} $ 为非负递减的数列, 如果级数 $ \sum_{n=0}^{\infty}a_{n} $ 收敛, 那么 $ \limit{n}{\infty} na_{n} = 0 $.
  \begin{hint}
      用 Cauchy 收敛准则与 $ \set{a_{n}} $ 的单调性考虑 $ \sum_{k = n}^{2n} a_{n} $.
  \end{hint}
  \begin{answer}
    由级数 $ \sum_{n = 0}^{\infty} a_{n} $ 收敛, 对任意 $ \varepsilon > 0 $, 存在 $ N \in \N $, 使得当 $ n > N $ 时, 有
    \begin{align*}
      \sum_{k = n}^{2n} a_{n} < \varepsilon
    \end{align*}
    这里不加绝对值是因为 $ a_{n} > 0 $. 又由于 $ a_{n} $ 单调递减, 那么就有
    \begin{align*}
      na_{2n} < \varepsilon
    \end{align*}
    即 $ (2n)a_{2n} < 2\varepsilon $, 这说明 $ \set{a_{n}} $ 的偶子列极限
    \begin{align*}
      \limit{n}{\infty} (2n)a_{2n} = 0,
    \end{align*}
    又因为级数 $ \sum_{n = 1}^{\infty} a_{n} $ 收敛, 那么 $ \limit{n}{\infty} a_{n} = 0 $, 即对上述 $ \varepsilon $, 存在 $ N' > N $, 使得 $ n > N' $ 时, $ a_{n} < \varepsilon $, 这时就有
    \begin{align*}
      (2n + 1)a_{2n + 1} < (2n + 1)a_{2n} = (2n)a_{2n} + a_{2n} < 3\varepsilon,
    \end{align*}
    这说明 $ \set{a_{n}} $ 的奇子列极限
    \begin{align*}
      \limit{n}{\infty} (2n + 1)a_{2n + 1} = 0,
    \end{align*}
    综上所数, $ \limit{n}{\infty} na_{n} = 0 $.
  \end{answer}
  \item (\emph{Frullani 积分}) 设 $ a, b > 0 $, $ f \in C[0, +\infty) $, 证明:
  \begin{exercise}
      \item  如果 $ \limit{x}{+\infty}f(x) = f(+\infty) $ 存在, 那么
      \begin{align*}
          \int_{0}^{+\infty} \frac{f(ax) - f(bx)}{x} \dx* = \qty(f(0) - f(+\infty))\ln\frac{b}{a}.
      \end{align*}
      \item 如果无穷积分 $ \int_{1}^{+\infty} f(x)/x \dx* $ 收敛, 那么
      \begin{align*}
          \int_{0}^{+\infty} \frac{f(ax) - f(bx)}{x} \dx* = f(0)\ln\frac{b}{a}.
      \end{align*}
      \item 如果 $ f(+\infty) $ 存在, 且积分 $ \int_{0}^{1} f(x)/x \dx* $ 收敛, 那么
      \begin{align*}
          \int_{0}^{+\infty} \frac{f(ax) - f(bx)}{x} \dx* = - f(+\infty)\ln\frac{b}{a}.
      \end{align*}
  \end{exercise}
  \begin{hint}
      (1) 可以考虑
      \begin{align*}
          \int_{A}^{B} \frac{f(ax)}{x} \dx* - \int_{A}^{B} \frac{f(bx)}{x} \dx* = \int_{aA}^{aB} \frac{f(x)}{x} \dx* - \int_{bA}^{bB} \frac{f(x)}{x} \dx* = \int_{aA}^{bA} \frac{f(x)}{x} \dx* - \int_{aB}^{bB} \frac{f(x)}{x} \dx*
      \end{align*}
      再利用积分中值定理与 $ A \to 0,\ B \to +\infty $ 得到结论. (2), (3) 同理.
  \end{hint}
  \begin{answer}
    \begin{answersheet}
      \item 考虑
      \begin{align*}
        \int_{A}^{B} \frac{f(ax)}{x} \dx* - \int_{A}^{B} \frac{f(bx)}{x} \dx* = \int_{aA}^{aB} \frac{f(x)}{x} \dx* - \int_{bA}^{bB} \frac{f(x)}{x} \dx* = \int_{aA}^{bA} \frac{f(x)}{x} \dx* - \int_{aB}^{bB} \frac{f(x)}{x} \dx*
      \end{align*}
      利用积分第一中值定理可知存在位于 $ aA, bA $ 之间的 $ \xi_{A} $, 与 $ aB, bB $ 之间的 $ \xi_{B} $, 使得
      \begin{align*}
        \int_{A}^{B} \frac{f(ax)}{x} \dx* - \int_{A}^{B} \frac{f(bx)}{x} \dx* 
        & = f(\xi_{A}) \int_{aA}^{bA} \frac{\dx}{x} - f(\xi_{B}) \int_{aB}^{bB} \frac{\dx}{x} \\
        & = (f(\xi_{A}) - f(\xi_{B})) \ln \frac{b}{a},
      \end{align*}
      再令 $ A \to 0,\ B \to +\infty $, 那么利用 $ f(x) $ 的连续性以及 $ \limit{x}{+\infty}f(x) = f(+\infty) $, 就有 $ f(\xi_{A}) \to f(0),\ f(\xi_{B}) \to f(+\infty) $, 于是
      \begin{align*}
        \int_{0}^{+\infty} \frac{f(ax)}{x} \dx* - \int_{A}^{B} \frac{f(bx)}{x} \dx* = (f(0) - f(+\infty)) \ln \frac{b}{a}
      \end{align*}
      \item 与 (1) 类似, 由于 $ \int_{1}^{+\infty} f(x)/x \dx* $ 收敛, 于是由 Cauchy 收敛准则可知: 对于任意 $ \varepsilon > 0 $, 都存在 $ X $, 使得当 $ \min(aB, bB) > X $ 时, 有
      \begin{align*}
        \abs{\int_{aB}^{bB} \frac{f(x)}{x} \dx*} < \varepsilon
      \end{align*}
      那么令 $ B \to +\infty $ 时就有
      \begin{align*}
        \limit{B}{+\infty} \int_{aB}^{bB} \frac{f(x)}{x} \dx* = 0,
      \end{align*}
      于是
      \begin{align*}
        \lim_{\substack{A \to 0 \\ B \to +\infty}} \int_{A}^{B} \frac{f(ax) - f(bx)}{x} \dx* 
        & = \limit{A}{0} \int_{aA}^{bA} \frac{f(x)}{x} \dx* - \limit{B}{+\infty} \int_{aB}^{bB} \frac{f(x)}{x} \dx* \\
        & = \limit{A}{0} f(\xi_{A}) \ln \frac{b}{a} = f(0) \ln \frac{b}{a}
      \end{align*}
      \item 与 (2) 类似, 瑕积分收敛可以得到
      \begin{align*}
        \limit{A}{0} \int_{aA}^{bA} \frac{f(x)}{x} \dx* = 0,
      \end{align*}
      其余步骤与 (1) 相同.
    \end{answersheet}
  \end{answer}
  \item 计算积分
  \begin{align*}
      I(r) = \int_{0}^{\pi} \ln(1 - 2 r \cos x + r^{2}) \dx*.
  \end{align*}
  \begin{hint}
      分别考虑 $ r = 0 $, $ \abs{r} < 1 $, $ \abs{r} = 1 $ 与 $ \abs{r} > 1 $, 其中 $ \abs{r} > 1 $ 的情况可以令 $ \rho = 1/r $, 再利用 $ \abs{r} < 1 $ 的情况即可.
      利用不定积分
      \begin{align*}
          \int \frac{1}{a + b\cos x} \dx* = \frac{2}{\sqrt{a^{2} - b^{2}}}\arctan\sqrt{\frac{a - b}{a + b}}t + C,
      \end{align*}
      计算 $ I'(r) $, 与利用
      \begin{align*}
          \int_{0}^{\pi/2} \ln \sin x \dx* = \int_{0}^{\pi/2} \ln \cos x\dx* = -\frac{\pi}{2}\ln 2.
      \end{align*}
      计算 $ I(\pm 1) $ 即可求得答案.
  \end{hint}
  \begin{answer}
    先考虑 $ \abs{r} < 1 $ 的情况, 将 $ r $ 看成参变量, 并对 $ r $ 求导, 可以得到
    \begin{align*}
      I'(r) = \int_{0}^{\pi} \frac{-2\cos x + 2r}{1 - 2r \cos x + r^{2}} \dx*.
    \end{align*}
    于是
    \begin{align}\label{eq:I'0=0}
      I'(0) = -2 \int_{0}^{\pi} \cos x \dx* = 0.
    \end{align}
    现在设 $ r \ne 0 $, 就有
    \begin{align}
      I'(r) & = \frac{1}{r} \int_{0}^{\pi} \qty(1 - \frac{1 - r^{2}}{1 - 2r \cos x + r^{2}}) \dx* \notag \\
      & = \frac{\pi}{r} - \frac{1 - r^{2}}{r} \int_{0}^{\pi} \frac{\dx}{1 - 2r \cos x + r^{2}}, \label{eq:I'r=pi/r+..}
    \end{align}
    利用万能公式 $ t = \tan(x/2) $ 换元, 可得
    \begin{align*}
      x = 2\arctan t,\quad \dx* = \frac{2}{1 + t^{2}} \dd{t}
    \end{align*}
    \begin{align*}
      \int_{0}^{\pi} \frac{\dx}{1 - 2r \cos x + r^{2}} 
      & = \int_{0}^{+\infty} \frac{2(1 + t^{2})^{-1}\dd{t}}{(1 + r^{2}) - 2r(1 - t^{2})(1 + t^{2})^{-1}} \\
      & = \int_{0}^{+\infty} \frac{2\dd{t}}{(1 - r)^{2} + t^{2}(1 + r)^{2}} \\
      & = \frac{2}{(1 - r)^{2}} \int_{0}^{+\infty} \frac{\dd t}{1 + \qty(\frac{1 + r}{1 - r}t)^{2}} \\
      & = \frac{2}{(1 - r)^{2}}\cdot \frac{1 - r}{1 + r} \arctan\qty(\frac{1 + r}{1 - r}t) \Big|_{0}^{+\infty} \\
      & = \frac{2}{1 - r^{2}} \cdot \frac{\pi}{2} = \frac{\pi}{1 - r^{2}}
    \end{align*}
    这里用到了 $ \abs{r} < 1 $ 的条件. 将结果代入等式 \eqref{eq:I'r=pi/r+..}, 并注意到等式 \eqref{eq:I'0=0}, 就可以知道 $ I'(r) = 0 $ 恒成立. 因此 $ I(r) $ 为常数, 又因为 $ I(0) = 0 $, 所以
    \begin{align*}
      \int_{0}^{\pi} \ln(1 - 2r \cos x + r^{2}) \dx* = 0,\quad \abs{r} < 1.
    \end{align*}

    再考虑 $ \abs{r} > 1 $ 的情况, 令 $ \rho = 1/r $, 则 $ \abs{\rho} < 1 $. 于是
    \begin{align*}
      I(r) = I(\rho^{-1}) & = \int_{0}^{\pi} \ln(1 - 2\rho \cos x + \rho^{2}) \dx* - \int_{0}^{\pi} \ln \rho^{2} \dx* \\
      & =  -2 \ln \abs{\rho} = 2\pi \ln \abs{r}.
    \end{align*}

    当 $ r = 1 $ 时, 就有
    \begin{align*}
      I(1) & = \int_{0}^{\pi} \ln(2 - 2\cos x) \dx* = \int_{0}^{\pi}  \ln(4\sin^{2}(x)) \dx* \\
      & = 2\pi \ln 2 + 4\int_{0}^{\pi/2} \ln \sin x \dx* = 0.
    \end{align*}
    这是因为 $ \int_{0}^{\pi/2} \ln \sin x \dx* = -\pi\ln 2/2 $. 同理可得 $ I(-1) = 0 $.

    综上所述:
    \begin{align*}
        I(r) = \begin{cases}
            0 & , \abs{r} \le 1;\\
            2\pi \ln \abs{r} & , \abs{r} > 1.
        \end{cases}
    \end{align*}
  \end{answer}
  \item 设 $ p > 0 $, 讨论积分
  \begin{align*}
      \int_{0}^{+\infty} \frac{\sin(1/x)}{x^{p}} \dx*
  \end{align*}
  的敛散性.
  \begin{hint}
      做换元 $ t = 1/x $, 则有
      \begin{align*}
          \int_{0}^{+\infty} \frac{\sin(1/x)}{x^{p}} \dx* = \int_{0}^{+\infty} \frac{\sin t}{t^{2 - p}} \dd{t}
      \end{align*}
      再分成 $ [0, 1]\cup[1, +\infty) $ 进行讨论.
  \end{hint}
  \begin{answer}
    先做换元 $ t = 1/x $ 把 $ \sin(1/x) $ 变成 $ \sin t $, 于是
    \begin{align*}
        x =\frac{1}{t}, \quad \dx* = - \frac{1}{t^{2}} \dd{t},
    \end{align*}
    那么就有
    \begin{align*}
        \int_{0}^{+\infty} \frac{\sin(1/x)}{x^{p}} \dx* = \int_{0}^{+\infty} \frac{\sin t}{t^{2 - p}} \dd{t}
    \end{align*}
    这是一个无穷积分, 也可能是一个瑕积分, 我们分段来看. 
    先考虑
    \begin{align*}
        I_{1} = \int_{0}^{1} \frac{\sin t}{t^{2 - p}} \dd{t}
    \end{align*}
    当 $ p \ge 2 $ 的时候, 这是一个常义积分, 当然收敛. 当 $ 1 \le p < 2 $ 时, 由于
    \begin{align*}
        \limit{t}{0}\frac{\sin t}{t^{2 - p}} = 0,
    \end{align*}
    故它还是一个常义积分, 也收敛. 当 $ 0 < p < 1 $ 时, 有
    \begin{align*}
        \limit{t}{0} \frac{t^{p - 2} \sin t }{t^{p - 1}} = \limit{t}{0} \frac{\sin t}{t} = 1,
    \end{align*}
    故 $ I_{1} $ 与 $ \int_{0}^{1} t^{p - 1} \dd{t} $ 同敛散 那么可知当 $ p > 0 $ 时 $ I_{1} $ 恒收敛, 并且恒绝对收敛,
    这是因为在 $ [0, 1] $ 上, $ t^{p - 2} \sin t \ge 0 $.

    再考虑
    \begin{align*}
        I_{2} = \int_{1}^{+\infty} \frac{\sin t}{t^{2 - p}} \dd{t},
    \end{align*} 
    首先可以看出当 $ p \ge 2 $ 时, $ I_{2} $ 发散, 否则若 $ I_{2} $ 收敛, 就应该存在充分大的 $ A', A'' $, 使得
    \begin{align}\label{eq:absintA'A''le1}
        \abs{\int_{A'}^{A''} t^{p - 2} \sin t \dd{t}} \le 1
    \end{align}
    那我们令 $ A' = 2k \pi,\ A'' = (2k + 1) \pi $, 由于 $ \sin t $ 在 $ [2k\pi, (2k + 1)\pi] $ 上非负, 
    那么当 $ k $ 充分大的时候就有
    \begin{align*}
        \abs{\int_{2k\pi}^{(2k + 1)\pi} t^{p - 2} \sin t \dd{t}} & \ge (2k\pi)^{p - 2} \int_{0}^{\pi} \sin t \dd{t}  \\
        & = 2(2k \pi)^{p - 2} \ge 2,
    \end{align*} 
    这与不等式 \eqref{eq:absintA'A''le1} 矛盾. 又由于
    \begin{align*}
      \abs{\frac{\sin t}{t^{2 - p}}} \le \frac{1}{t^{2 - p}},
    \end{align*}
    于是有 $ 2 - p < 1 $ 时, 即 $ 0 < p < 1 $ 时, $ I_{2} $ 绝对收敛. 而当 $ 2 - p > 0 $ 时, 即 $ 0 < p < 2 $ 时, 由于 $ t \to +\infty $ 时, $ t^{p - 2} $ 单调递减趋于 $ 0 $, 并且对于任意 $ M > 1 $, 总有
    \begin{align*}
      \abs{\int_{1}^{M} \sin t \dd{t}} = \abs{\cos M - \cos 1} \le 2,
    \end{align*}
    于是由 Dirichlet 判别法可知此时 $ I_{2} $ 收敛, 下面判断 $ 1 \le p < 2 $ 时 $ I_{2} $ 是否绝对收敛. 
    由于 $ \abs{\sin t} \le 1 $, 于是
    \begin{align*}
      \abs{\frac{\sin t}{t^{2 - p}}} & \ge \frac{\sin^{2}(t)}{t^{2 - p}} = \frac{1}{2t^{2 - p}} - \frac{\cos(2t)}{2t^{2 - p}},
    \end{align*}
    同样由 Dirichlet 判别法可知无穷积分 $ \int_{1}^{+\infty} t^{p - 2} \cos (2t)/2 \dx* $ 收敛, 而 $ 2 - p < 1 $, 于是 $ \int_{1}^{+\infty} t^{p - 2}/2 \dx* $ 发散, 故 $ I_{2} $ 不绝对收敛. 

    综上所述, 当 $ 0 < p < 1 $ 时原积分绝对收敛, $ 1 \le p < 2 $ 时原积分条件收敛, $ p \ge 2 $ 时原发散.
  \end{answer}
  \item 求积分 $ \int_{0}^{+\infty} \sin(x)/x \dx* $.
  \begin{hint}
    用积分因子构造含参变量反常积分:
    \begin{align*}
      H(t) = \int_{0}^{+\infty} \me^{-tx}\frac{\sin x}{x} \dx*, \quad t\in [0, A],
    \end{align*}
    求出 $ H'(t) $, 再利用 $ \int_{0}^{+\infty}H'(t)\dd{t} $ 来计算 $ H(0) $.
  \end{hint}
  \begin{answer}
    这里先回忆一下含参变量反常积分求导的定理:

    如果函数 $ f $ 和 $ \pdv{f}{t} $ 都在 $ [a, +\infty) \times [\alpha, \beta] $ 上连续, 且积分 $ \int_{a}^{+\infty} \pdv{f(x, t)}{t} \dx* $ 在 $ [\alpha, \beta] $ 上一致收敛, 那么 $ H(t) = \int_{a}^{+\infty}f(x, t)\dx* $ 在 $ [\alpha, \beta] $ 上可微, 且
    \begin{align*}
      H'(t) = \int_{a}^{+\infty}\pdv{f(x, t)}{x}\dx*,\quad \alpha \le u \le \beta.
    \end{align*}

    对于任意 $ A > 0 $, 我们利用积分因子构造含参变量反常积分:
    \begin{align*}
      H(t) = \int_{0}^{+\infty} \me^{-tx}\frac{\sin x}{x} \dx* = \int_{0}^{+\infty} f(x, t) \dx*,\quad t \in [0, A],
    \end{align*}
    然后我们依次来验证条件来求得 $ H'(t) $, 首先
    \begin{align*}
      f(x, t) = \me^{-tx}\frac{\sin x}{x}, \quad \pdv{f(x, t)}{t} = -\me^{-tx}\sin x
    \end{align*}
    在 $ (x, t) \in [0, +\infty)\times [0, A] $ 上连续, 其次可以用 Weierstrass 判别法验证
    \begin{align*}
      \int_{0}^{+\infty} \pdv{f(x, t)}{t} \dx* = - \int_{0}^{+\infty} \me^{-tx}\sin x \dx*
    \end{align*}
    在任意 $ t \in [\alpha, \beta] \subset (0, A) $, 那么 $ H(t) $ 在 $ [\alpha, \beta] $ 上可微, 并且
    \begin{align*}
      H'(t) = - \int_{0}^{+\infty} \me^{-tx}\sin x \dx*, \quad t \in [\alpha, \beta] \subset (0, A)
    \end{align*}
    那么由 $ \alpha $ 和 $ \beta $ 的任意性可知该结果在 $ t\in (0, A) $ 上均成立, 而对于这个反常积分的计算, 可以使用分部积分:
    \begin{align*}
      I(t) & = \int_{0}^{+\infty} \me^{-tx}\sin x \dx* = -\frac{1}{t}\int_{0}^{+\infty} \sin x\dd{\me^{-tx}}\\
      & = -\frac{1}{t}\sin x\me^{-tx}\Big|_{x = 0}^{+\infty} + \frac{1}{t} \int_{0}^{+\infty} \me^{-tx}\cos x \dx* = -\frac{1}{t^{2}} \cos x\dd{\me^{-tx}}\\
      & = -\frac{1}{t^{2}}\cos x\me^{-tx}\Big|_{x = 0}^{+\infty} + \frac{1}{t^{2}}\int_{0}^{+\infty} -\me^{-tx}\sin x \dx* = \frac{1}{t^{2}} - \frac{1}{t^{2}} I(t),
    \end{align*}
    可以解得 $ I(t) = (1 + t^{2})^{-1} $, 所以 $ H'(t) = -(1 + t^{-2})^{-1} $. 或者用\footnote{这种写法我不确定是否标准, 或者是否有足够的理论依据} $ \sin x = (\me^{\mi x} - \me^{-\mi x})/(2\mi) $. 然后我们要说明 $ H(t) $ 在 $ t = 0 $ 与 $ t = A $ 处均是连续的, 因为这样才可以利用 Newton -- Leibniz 公式来求得 $ H(0) $.

    首先由 $ \limit{x}{0+}\sin x/x = 1 $ 可知 $ x = 0 $ 不是 $ \int_{0}^{+\infty}\sin x/x \dx* $ 的瑕点, 又因为 $ \int_{0}^{M}\sin x\dx* $ 有界, 且 $ x^{-1} $ 在 $ x \to +\infty $ 的过程中单调递减趋于 $ 0 $, 所以由 Dirichlet 判别法可知 $ \int_{0}^{+\infty} \sin x/x \dx* $ 在收敛, 由于它不含有 $ t $, 那么它就关于 $ t\in [0, A] $ 一致收敛, 并且对于每个固定的 $ x \in [0, +\infty) $, $ \me^{-tx} $ 在 $ t\in[0, A] $ 是单调函数, 且 $ \abs{\me^{-tx}} \le 1 $ 对 $ t \in [0, A] $ 恒成立, 那么可以用 Abel 判别法说明 $ H(t) $ 关于 $ t \in [0, A] $ 一致收敛, 则有 $ H(t) \in C[0, A] $. 那么我们就有
    \begin{align}\label{eq:sinx/xarctanA}
      H(A) - H(0) = \int_{0}^{A} -\frac{1}{1 + t^{2}} \dd{t} = -\arctan A,
    \end{align}
    而我们又有
    \begin{align*}
      0 \le H(t) \le \int_{0}^{+\infty} \me^{-tx}\abs{\frac{\sin x}{x}} \dx* \le \int_{0}^{+\infty} \me^{-tx} \dx* = \frac{1}{t},
    \end{align*}
    于是 $ \limit{A}{+\infty} H(A) = 0 $, 那么对等式 \eqref{eq:sinx/xarctanA} 两侧同时令 $ A \to +\infty $, 就有 $ H(0) = \pi/2 $.
  \end{answer}
  \item 设 $ f(x) \in C[0, 1] $, 证明
  \begin{align*}
    \limit{n}{\infty} \frac{1}{n} \sum_{k=1}^{n} (-1)^{k+1} f\qty(\frac{k}{n}) = 0.
  \end{align*}
  \begin{hint}
    由 $ f \in C[0, 1] $ 可知 $ f $ 在 $ [0, 1] $ 上一致连续, 则当 $ \abs{x - y} $ 足够小的时候 $ \abs{f(x) - f(y)} $ 也足够小, 再考虑 $ n $ 为偶数时
    \begin{align*}
      \frac{1}{n} \abs{\sum_{k=1}^{n} (-1)^{k+1} f\qty(\frac{k}{n})} & = \frac{1}{n}\abs{\sum_{k = 1}^{n/2}f\qty(\frac{2k-1}{n}) - f\qty(\frac{2k}{n})} \le \frac{1}{n}\cdot\frac{n}{2}\varepsilon.
    \end{align*}
    $ n $ 为奇数时只需要增加一项 $ f(1)/n $.
  \end{hint}
  \begin{answer}
    首先由 Cantor 定理可知 $ f(x) $ 在 $ [0, 1] $ 上一致连续, 于是对于任意 $ \varepsilon > 0 $, 都存在 $ N \in \N $, 使得对于 $ n > N $, 当 $ \abs{x - y} \le 1/n $ 时, 就有 $ \abs{f(x) - f(y)} < \varepsilon $. 那么考虑 $ n $ 为偶数时:
    \begin{align*}
      \frac{1}{n} \abs{\sum_{k=1}^{n} (-1)^{k+1} f\qty(\frac{k}{n})} 
      & = \frac{1}{n}\abs{\sum_{k = 1}^{n/2}f\qty(\frac{2k-1}{n}) - f\qty(\frac{2k}{n})} \\
      & \le \frac{1}{n}\sum_{k = 1}^{n/2}\abs{f\qty(\frac{2k-1}{n}) - f\qty(\frac{2k}{n})} \\
      & \le \frac{1}{n}\cdot\frac{n}{2}\varepsilon \le \frac{\varepsilon}{2}.
    \end{align*}
    再考虑 $ n $ 为奇数时:
    \begin{align*}
        \frac{1}{n} \abs{\sum_{k=1}^{n} (-1)^{k+1} f\qty(\frac{k}{n})} 
        & = \frac{1}{n}\abs{\sum_{k = 1}^{(n - 1)/2}f\qty(\frac{2k-1}{n}) - f\qty(\frac{2k}{n}) + f(1)} \\
        & \le \frac{1}{n}\sum_{k = 1}^{(n - 1)/2}\abs{f\qty(\frac{2k-1}{n}) - f\qty(\frac{2k}{n})} + \frac{f(1)}{n} \\
        & \le \frac{1}{n} \frac{n - 1}{2} \varepsilon + \frac{f(1)}{n},
    \end{align*}
    而对上述 $ \varepsilon $, 存在 $ N' > N $, 使得 $ n > N' $ 时, $ \abs{f(1)/n} < \varepsilon $, 这样就有 $ n > N' $ 时
    \begin{align*}
        \frac{1}{n} \abs{\sum_{k=1}^{n} (-1)^{k+1} f\qty(\frac{k}{n})} \le \qty(\frac{n - 1}{2n} + 1) \varepsilon < 2 \varepsilon,
    \end{align*}
    综上所述, 有
    \begin{align*}
        \limit{n}{\infty} \frac{1}{n} \sum_{k=1}^{n} (-1)^{k+1} f\qty(\frac{k}{n}) = 0.
    \end{align*}
  \end{answer}
  \item 证明: 积分
  \begin{align*}
      \int_{0}^{+\infty} \me^{-(\alpha + u^{2})t} \sin(t) \dd{t},\quad \alpha > 0
  \end{align*}
  关于 $ u $ 在 $ [0, +\infty) $ 上一致收敛.
  \begin{hint}
      $ \abs{\me^{-(\alpha + u^{2})t} \sin(t)} \le \me^{-\alpha t} $
  \end{hint}
  \begin{answer}
      直接放缩:
      \begin{align*}
          \abs{\me^{-(\alpha + u^{2})t} \sin(t)} \le \me^{-\alpha t},
      \end{align*}
      由于 $ \alpha > 0 $, 所以 $ \int_{0}^{+\infty} \me^{-\alpha t} \dd{t} $ 收敛, 于是由 Weierstrass 判别法可知
      \begin{align*}
          \int_{0}^{+\infty} \me^{-(\alpha + u^{2}) t} \sin(t) \dd{t}
      \end{align*}
      关于 $ u $ 在 $ [0, +\infty] $ 上一致收敛.
  \end{answer}
  \item 证明: 积分
  \begin{align*}
      \int_{1}^{+\infty} \me^{-(\alpha + u^{2})t} \sin(t) \dd{u},\quad \alpha > 0
  \end{align*}
  关于 $ t $ 在 $ [0, +\infty) $ 上一致收敛.
  \begin{hint}
    注意到 $ u = 0 $ 不是被积函数的瑕点, 所以只要考虑 $ \int_{1}^{+\infty} $ 关于 $ t $ 在 $ [0, +\infty) $ 上的一直收敛性. 
      \begin{align*}
          \abs{\me^{-(\alpha + u^{2})t} \sin(t)} \le \abs{\frac{t}{\me^{(\alpha + u^{2})t}}} \le \frac{t}{1 + u^{2}t} \le \frac{1}{u^{2}}
      \end{align*}
  \end{hint}

  \hitem 计算 Fresnel 积分 $ \int_{0}^{+\infty} \sin(x^{2}) \dx* $.
  \item 计算积分 $ \int_{0}^{+\infty} \exp(-a x^{2})\cos(bx) \dx* $, 其中 $ a > 0,\ b\in \R $.
  \begin{hint}
      记
      \begin{align*}
          I(b) = \int_{0}^{+\infty} \exp(-a x^{2})\cos(bx) \dx*
      \end{align*}
      则又 Weierstrass 判别法可知
      \begin{align*}
          \int_{0}^{+\infty} \pdv{b}\qty(\exp(-a x^{2})\cos(bx)) \dx*
      \end{align*}
      在 $ \R $ 上一致收敛, 于是可求得 $ I'(b) $, 再对 $ I(b) $ 利用分部积分, 可得
      \begin{align*}
          I'(b) = -\frac{b}{2a}I(b)
      \end{align*}
      即 $ \qty(\ln I(b))' = -b/(2a) $, 再利用 $ I(0) $ 的值可以计算得 $ I(b) $.
  \end{hint}
  \begin{answer}
      \begin{align*}
          I(b) = \frac{1}{2}\sqrt{\frac{\pi}{a}}\exp\qty(-\frac{b^{2}}{4a}).
      \end{align*}
  \end{answer}
  \item 计算积分 $ \int_{0}^{\pi/2} \tan^{\alpha}(x) \dx* $, 其中 $ \abs{\alpha} < 1 $.
  \begin{hint}
      可以直接利用 $ \Beta $ 函数进行求解.
  \end{hint}
  \begin{answer}
      这里我们先给出 $ \Beta $ 函数的几种形式:
      \begin{align*}
          \Beta(p, q) = \int_{0}^{1} x^{p - 1}(1 - x)^{q - 1} \dx*,\quad p > 0, q > 0
      \end{align*}
      如果令 $ x = \cos^{2} t $ 进行换元, 就可以得到
      \begin{align*}
          \Beta(p, q) = 2 \int_{0}^{\pi/2} \cos^{2p - 1}t \sin^{2q-1} t \dd{t},
      \end{align*}
      如果令 $ x = y/(1 + y) $ 换元, 就可以得到
      \begin{align*}
          \Beta(p, q) = \int_{0}^{+\infty} \frac{y^{p - 1}}{(1 + y)^{p + q}} \dd{y},
      \end{align*}
      如果把积分趋于拆成 $ (0, 1) $ 与 $ (1, +\infty) $, 那么在其中一个区间上令 $ y = 1/u $ 就有
      \begin{align*}
          \Beta(p, q) = \int_{0}^{1} \frac{u^{p - 1} + u^{q - 1}}{(1 + u)^{p + q}} \dd{u} = \int_{1}^{+\infty} \frac{u^{p - 1} + u^{q - 1}}{(1 + u)^{p + q}} \dd{u}
      \end{align*}
      并且余元公式
      \begin{align*}
          \Gamma(p)\Gamma(1 - p) = \Beta(p, 1 - p) = \frac{\pi}{\sin p\pi}
      \end{align*}
      于是就有
      \begin{align*}
          \int_{0}^{\pi/2} \tan^{\alpha}x \dx* & = \int_{0}^{\pi/2} \sin^{\alpha}x\cos^{-\alpha}x \dx* = \frac{1}{2}\Beta\qty(\frac{\alpha + 1}{2}, \frac{1 - \alpha}{2}) \\
                                                  & = \frac{\pi}{2\sin\qty(\frac{\alpha + 1}{2}\pi)} = \frac{\pi}{2\cos(\alpha\pi/2)},
      \end{align*}
  \end{answer}
  \sitem\label{item:AX=XB} 设 $ \K $ 是数域, $ A \in \MM[\K][m] $, $ B \in \MM[\K][n] $, 且 $ A, B $ 没有相同的特征值, 证明矩阵方程 $ AX = XB $ 只有零解.
  \begin{hint}
      由题可知对任意多项式 $ f(x) $, 都有 $ f(A)X = Xf(B) $, 特别地令 $ f(x) $ 为 $ A $ 的特征多项式即可.
  \end{hint}
  \begin{answer}
      由题目可知
      \begin{align*}
          A^{2}X = AAX = A(XB) = (AX)B = XB^{2},
      \end{align*}
      由数学归纳法可知对任意的 $ k \in \N $, 都有 $ A^{k}X = XB^{k} $, 也就是说对于任意的多项式 $ f(x) $, 总有 $ f(A)X = Xf(B) $, 那么特别地, 我们取 $ f(x) $ 为 $ A $ 的特征多项式, 那么由 Cayley -- Hamilton 定理可知 $ f(A) = 0 $, 那么就有 $ Xf(B) = 0 $, 下面我们说明 $ f(B) $ 可逆.

      如果 $ f(B) $ 不可逆, 那么 $ f(B) $ 有 $ 0 $ 特征值, 而对 $ B $ 的特征值 $ \lambda $, 有 $ f(B) $ 的特征值是 $ f(\lambda) $, 那么这就说明存在 $ B $ 的某个特征值 $ \lambda_{B} $ 满足 $ f(\lambda_{B}) = 0 $, 也就说 $ \lambda_{B} $ 是 $ f(x) $ 的一个根, 而 $ f(x) $ 的全体根都是 $ A $ 的特征值, 这说明 $ \lambda_{B} $ 也是 $ A $ 的特征值, 这与 $ A, B $ 没有公共特征值矛盾, 故 $ f(B) $ 可逆. 这样在 $ Xf(B) = 0 $ 两侧同时右乘 $ (f(B))^{-1} $ 即可得到 $ X = 0 $, 也就是说矩阵方程 $ AX = XB $ 只有零解.
  \end{answer}
  \item 如 \ref{item:AX=XB} 题, 可以进一步证明逆命题也成立, 即: 如果 $ AX = XB $ 只有零解, 则 $ A, B $ 无公共特征值.
  \begin{hint}
      反证 $ A, B' $ 有公共特征值 $ \lambda $, 设特征向量分别为 $ \alpha, \beta $, 那么 $ X = \alpha\beta' $ 即为 $ AX = XB $ 的一个非零解.
  \end{hint}
  \begin{answer}
      先说明 $ B $ 与 $ B' $ 有相同的特征值: 对任意 $ B $ 的特征值 $ \lambda $ 都有 $ \abs{\lambda I - B} = 0 $, 取转置就有 $ \abs{\lambda I - B'} = 0 $, 即 $ \lambda $ 都是 $ B' $ 的特征值, 反过来同理.

      那么我们用反证法, 设 $ A, B' $ 有公共特征值 $ \lambda $, 对应的特征向量分别为 $ \alpha, \beta $, 也就是有
      \begin{align*}
          A\alpha = \lambda\alpha, B'\beta = \lambda\beta,
      \end{align*}
      于是我们令 $ X = \alpha\beta' \ne 0 $, 那么就有
      \begin{align*}
          AX & = A\alpha\beta' = \lambda\alpha\beta',\\
          XB & = \alpha\beta B' = \alpha(B'\beta)' = \lambda\alpha\beta'
      \end{align*}
      这说明 $ AX = XB $ 有非零解, 矛盾. 故 $ A, B $ 没有公共特征值.
  \end{answer}
  \item 设 $ A $ 是 4 阶方阵, 满足 $ \tr(A^{i}) = i\,(i = 1, 2, 3, 4) $, 求 $ \abs{A} $.
  \begin{hint}
      用 Newton 公式, 题目条件为 $ s_{i} = i $, 要求 $ \sigma_{4} $.
  \end{hint}
  \begin{answer}
      首先来熟悉一下对称多项式与 Newton 公式: 设 $ x_{1}, x_{2}, \dots, x_{n} $ 为 $ n $ 个不定元, 那么初等对称多项式 $ \sigma_{k}\,(k \le n) $ 为:
      \begin{align*}
          \sigma_{k} = \sum_{\mathclap{1 \le r_{1} < r_{2} < \dots < r_{k} \le n}} x_{r_{1}}x_{r_{2}}\dotsm x_{r_{k}},
      \end{align*}
      而对于任意的 $ k \in \N $, 有
      \begin{align*}
          s_{k} = \sum_{i = 1}^{n}x_{i}^{k},
      \end{align*}
      那么我们有 Newton 公式:
      \begin{align*}
          0 = \begin{cases}
              s_{k} - s_{k - 1}\sigma_{1} + s_{k - 2}\sigma_{2} + \dots + (-1)^{k}k\sigma_{k} & , k \le n \\
              s_{k} - s_{k - 1}\sigma_{1} + s_{k - 2}\sigma_{2} + \dots + (-1)^{n}s_{k - n}\sigma_{n} & , k > n\\
          \end{cases}
      \end{align*}
      而在这道题中, 我们知道 $ \tr(A) $ 是矩阵 $ A $ 的特征值之和, 于是我们令 $ A $ 的 $ 4 $ 个特征值为 $ \lambda_{1}, \lambda_{2}, \lambda_{3}, \lambda_{4} $, 那么题目条件就可以转化为: 已知 $ s_{i} = i\,(i = 1, 2, 3, 4) $, 结论可以转化为: 求 $ \abs{A} = \lambda_{1}\lambda_{2}\lambda_{3}\lambda_{4} = \sigma_{4} $.

      那么我们依次来计算 $ \sigma_{1}, \sigma_{2}, \sigma_{3}, \sigma_{4} $, 注意到 $ s_{1} = \sigma_{1} = 1 $, 那么就有
      \begin{align*}
          s_{2} - s_{1}\sigma_{1} + 2\sigma_{2} = 0 & \implies \sigma_{2} = -\frac{1}{2}\\
          s_{3} - s_{2}\sigma_{1} + s_{1}\sigma_{2} - 3\sigma_{3} = 0 & \implies \sigma_{3} = \frac{1}{6}\\
          s_{4} - s_{3}\sigma_{1} + s_{2}\sigma_{2} - s_{1}\sigma_{3} + 4\sigma_{4} = 0 & \implies \sigma_{4} = \frac{1}{24}
      \end{align*}
  \end{answer}
  \item $ n $ 阶方阵可对角化的充分必要条件.
  \begin{hint}
      \begin{hintsheet}
          \item $ A $ 有 $ n $ 个线性无关的特征向量
          \item 复数域上的列向量空间 $ \C^{n} $ 可以分解为 $ A $ 的特征子空间的直和
          \item $ A $ 有完全的特征向量系, 即对 $ A $ 的任一特征值, 其几何重数等于其代数重数
          \item $ A $ 的极小多项式无重根
          \item $ A $ 的 Jordan 块都是一阶的 (或 $ A $ 的初等因子都是一次多项式)
      \end{hintsheet}
  \end{hint}
  \hitem 设 $ f(x) $ 在 $ [0, 1] $ 上可积, 在 $ x = 1 $ 处左连续, 证明:
  \begin{align*}
      \limit{n}{\infty}\frac{\int_{0}^{1} x^{n} f(x) \dx*}{\int_{0}^{1} x^{n} \dx*} = f(1).
  \end{align*}
  \begin{hint}
      注意到
      \begin{align*}
          f(1) = (n + 1) \int_{0}^{1} x^{n} f(1) \dx*,
      \end{align*}
      然后使用拟合法在 $  x = 1 $ 附近分界.
  \end{hint}
  \begin{answer}
      因为
      \begin{align*}
          f(1) = \frac{\int_{0}^{1} x^{n} f(1) \dx*}{\int_{0}^{1} x^{n} \dx*},
      \end{align*}
      于是我们只要证明
      \begin{align*}
          \limit{n}{\infty} \frac{\int_{0}^{1} x^{n} (f(x) - f(1)) \dx*}{\int_{0}^{1} x^{n} \dx*} = \limit{n}{\infty} (n + 1) \int_{0}^{1} x^{n} (f(x) - f(1)) \dx* = 0,
      \end{align*}
      由于 $ f(x) $ 在 $ x = 1 $ 左连续, 那么对任意 $ \varepsilon > 0 $, 存在 $ \delta > 0 $, 使得 $ x \in (1 - \delta, 1) $ 时, 有
      \begin{align*}
          \abs{f(x) - f(1)} < \varepsilon,
      \end{align*}
      又因为 $ f(x) $ 在 $ (0, 1) $ 上可积, 那么 $ f(x) $ 在 $ (0, 1) $ 上有界, 设 $ \abs{f(x)} \le M $, 那么就有
      \begin{align*}
          \abs{f(x) - f(1)} \le \abs{f(x)} + \abs{f(1)} \le 2M
      \end{align*}
      \begin{align*}
          (n + 1) \abs{\int_{0}^{1} x^{n} (f(x) - f(1)) \dx*} & \le (n + 1) \abs{\int_{0}^{1 - \delta} x^{n} (f(x) - f(1)) \dx*} + (n + 1) \abs{\int_{1 - \delta}^{1} x^{n} (f(x) - f(1)) \dx*}\\
          & \le (n + 1) \int_{0}^{1 - \delta} x^{n} \abs{f(x) - f(1)} \dx* + (n + 1) \int_{1 - \delta}^{1} x^{n} \abs{f(x) - f(1)} \dx*\\
          & \le 2M(n + 1)\int_{0}^{1 - \delta} x^{n} \dx* + \varepsilon (n + 1) \int_{0}^{1} x^{n} \dx*\\
          & \le 2M(1 - \delta)^{n + 1} + \varepsilon,
      \end{align*}
      由于 $ 1 - \delta < 1 $, 那么对上述 $ \varepsilon $, 存在 $ N \in \N $, 使得当 $ n > N $ 时, $ (1 - \delta)^{n + 1} < \varepsilon/(2M) $, 于是就有
      \begin{align*}
          (n + 1) \abs{\int_{0}^{1} x^{n} (f(x) - f(1)) \dx*} \le 2M(1 - \delta)^{n + 1} + \varepsilon \le 2\varepsilon,
      \end{align*}
      这就说明了 $ \limit{n}{\infty} (n + 1) \int_{0}^{1} x^{n} (f(x) - f(1)) \dx* = 0 $.
  \end{answer}
  \item\label{item:A2=AA'} 设 $ A \in \MM $, 若 $ A^{2} = AA' $, 证明 $ A $ 为实对称阵.
  \begin{hint}
      对于 $ B \in \MM $, 如果有 $ \tr(BB') = 0 $, 就有 $ B = 0 $, 那么这里可以考虑用这种方法来证明 $ A - A' = 0 $.
  \end{hint}
  \begin{answer}
      我们先说明对于 $ B = (b_{ij}) \in \MM $, 如果 $ \tr(BB') = 0 $, 则 $ B = 0 $: 由矩阵乘法可知 $ BB' $ 的 $ (i, i) $ 元为
      \begin{align*}
          BB'(i, i) = b_{i1}^{2} + b_{i2}^{2} + \dots + b_{in}^{2}
      \end{align*}
      那么
      \begin{align*}
          \tr(BB') = \sum_{i = 1}^{n} BB'(i, i) = \sum_{i = 1}^{n}\sum_{j = 1}^{n} b_{ij}^{2} = 0
      \end{align*}
      这就说明 $ b_{ij} = 0 $, 即 $ B = 0 $. 那么考虑 $ \tr((A - A')(A - A')') $:
      \begin{align*}
          \tr((A - A')(A - A')') = \tr(A^{2} - AA' - A'A + (A')^{2}) = \tr(0) = 0,
      \end{align*}
      即说明 $ A - A' = 0 $, 也就是 $ A $ 为实对称阵.
  \end{answer}
  \item 设 $ A, B \in \MM $, 若 $ A^{2} = A $, $ B^{2} = B $ 以及 $ (A + B)^{2} = A + B $, 证明 $ AB = BA = 0 $.
  \begin{hint}
      这里要利用一个经典的递推: 若 $ AB = BA $, 则 $ A^{k}B = BA^{k}\,(k \in \N) $, 只是要进行一些小的改动.
  \end{hint}
  \begin{answer}
      首先由 $ A^{2} = A, B^{2} = B $ 可以得到
      \begin{align*}
          (A + B)^{2} = A^{2} + AB + BA + B^{2} = A + B \implies AB + BA = 0 \implies AB = -BA,
      \end{align*}
      那么就有
      \begin{align*}
          A^{2}B = AAB = -ABA = -(AB)A = -(-BA)A = BA^{2} \implies AB = BA,
      \end{align*}
      上下对比即可得到 $ AB = BA = 0 $.
  \end{answer}
  \item 设 $ A, B $ 都是 $ n $ 阶矩阵, 若 $ A^{k} = 0 $, 且 $ AB + BA = B $, 证明 $ B = 0 $.
  \begin{hint}
      利用 $ A^{\ell}B = B(I - A)^{\ell} $, 与 $ A $ 的特征值全 $ 0 $ 可以得到 $ B = 0 $.
  \end{hint}
  \begin{answer}
      将 $ AB + BA = B $ 移项可得 $ AB = B(I - A) $, 那么就有
      \begin{align*}
          A^{2}B = AAB = AB(I - A) = B(I - A)(I - A) = B(I - A)^{2}
      \end{align*}
      同理可得对于任意的 $ \ell \in \N $, 总有 $ A^{\ell}B = B(I - A)^{\ell} $, 特别地, 令 $ \ell = k $, 于是就有 $ B(I - A)^{k} = 0 $. 再注意到 $ A^{k} = 0 $, 那么 $ 1 $ 不是 $ A $ 的特征值, 故 $ \abs{I - A} \ne 0 $, 也即 $ (I - A)^{k} $ 可逆. 那么就有 $ B = 0 $.
  \end{answer}
  \item $ A, B $ 是 $ n $ 阶方阵, $ A + B = AB $, 求证
  \begin{exercise}
      \item $ AB = BA $,
      \item $ \rank(A) = \rank(B) $,
      \item $ A $ 可相似对角化当且仅当 $ B $ 可相似对角化.
  \end{exercise}
  \begin{hint}
      \begin{hintsheet}
          \item 使用 $ (I = A)(I - B) = I $ 来验证.
          \item 移项得 $ A = (A - I)B $, 利用 $ A - I $ 可逆可得结论.
          \item 注意到如果 $ A $ 可对角化, 则 $ A - I $ 可对角化, $ B $ 同理.
      \end{hintsheet}
  \end{hint}
  \begin{answer}
      \begin{answersheet}
          \item 由题可知 $ I - A - B + AB = I $, 也就是 $ (I - A)(I - B) = I $, 这说明 $ I - A $ 与 $ I - B $ 互为逆矩阵, 那么就又有 $ (I - B)(I - A) = I $, 展开后即可得 $ A + B = BA $, 这就说明 $ AB = BA $.
          \item 移项可得 $ A = (A - I)B $, 因为任意矩阵与可逆阵相乘都不改变秩, 又由上一问知道 $ I - A $ 可逆, 于是 $ \rank A = \rank B $.
          \item 假设 $ A $ 可对角化, 即存在可逆阵 $ P $, 使得 $ P^{-1}AP $ 为对角阵, 于是 $ P^{-1}(A - I)P = P^{-1}AP - I $ 也为对角阵, 又因为 $ (A - I) $ 可逆, 所以 $ P^{-1}(A - I)^{-1}P $ 也为对角阵. 由上一问可知 $ (A - I)^{-1}A = B $, 那么
          \begin{align*}
              P^{-1}BP = P^{-1}(A - I)^{-1}PP^{-1}AP
          \end{align*}
          也为对角阵, 故 $ B $ 可相似对角化. 如果 $ B $ 可相似对角化, 讨论同上.
      \end{answersheet}
  \end{answer}
  \sitem 设 $ f(x), g(x) $ 为多项式, 且 $ (f(x), g(x)) = 1 $, $ A $ 是 $ n $ 阶方阵, 求证: $ f(A)g(A) = 0 $ 的充分必要条件为 $ \rank(f(A)) + \rank(g(A)) = n $.
  \begin{hint}
      由互素可以得到
      \begin{align*}
          f(A)u(A) + g(A)v(A) = I,
      \end{align*}
      然后再对 $ \smqty(f(A) & \\ & g(A)) $ 进行初等变换得到秩的关系.
  \end{hint}
  \begin{answer}
      由 $ (f(x), g(x)) = 1 $ 互素可知存在多项式 $ u(x), v(x) $, 使得
      \begin{align*}
          f(x)u(x) + g(x)v(x) = 1,
      \end{align*}
      用矩阵 $ A $ 代入, 可得
      \begin{align*}
          f(A)u(A) + g(A)v(A) = I,
      \end{align*}
      注意到矩阵 $ A $ 的多项式之间的乘法是可交换的, 于是考虑如下分块初等变换
      \begin{align*}
          \mqty(f(A) & \\ & g(A)) & \rightarrow \mqty(f(A) & g(A)v(A) \\ & g(A)) \rightarrow \mqty(f(A) & f(A)u(A) + g(A)v(A) \\ & g(A)) \\
                                  & \rightarrow \mqty( & I \\ -f(A)g(A) & g(A)) \rightarrow \mqty( 0 & I \\ -f(A)g(A) & 0 )
      \end{align*}
      那么对两端的矩阵同时取秩, 可得
      \begin{align*}
          \rank(f(A)) + \rank(g(A)) = n + \rank(f(A)g(A)),
      \end{align*}
      那么显然有 $ f(A)g(A) = 0 $ 的充分必要条件为 $ \rank(f(A)) + \rank(g(A)) = n $.
  \end{answer}
  \item 设 $ A, B $ 为实对称阵, 求证:
  \begin{exercise}
      \item 若 $ A $ 正定, 则存在实可逆阵 $ P $ 使得 $ P'AP $ 和 $ P'BP $ 同时为对角阵;
      \item 若 $ A, B $ 半正定, 则 $ \tr(AB) \ge 0 $, 并且等号成立当且仅当 $ AB = 0 $.
  \end{exercise}
  \begin{hint}
      \begin{hintsheet}
          \item $ C'AC = I $, 而 $ C'BC $ 为对称阵, 可以正交相似于对角阵.
          \item 利用 $ A = C'C,\ B = D'D $ 的分解来处理.
      \end{hintsheet}
  \end{hint}
  \begin{answer}
      \begin{answersheet}
          \item 由于 $ A $ 正定, 故存在可逆阵 $ C $, 使得 $ C'AC = I $, 同时可知 $ C'BC $ 也是对称阵, 那么它可以正交相似于一个对角阵, 即存在正交阵 $ Q $, 使得
          \begin{align*}
              Q'C'BCQ = (CQ)'B(CQ) = \mqty(\dmat{\lambda_{1}, \lambda_{2}, \ddots, \lambda_{n}}).
          \end{align*}
          为对角阵, 同时由正交阵的性质可知
          \begin{align*}
              (CQ)'A(CQ) = Q'C'ACQ = Q'IQ = I.
          \end{align*}
          令 $ P = CQ $ 即可.

          更进一步, $ \lambda_{1}, \lambda_{2}, \dots, \lambda_{n} $ 是 $ P'BP $ 的特征值, 也就是说
          \begin{align*}
              \abs{\lambda_{i}I - P'BP} = \abs{P'(\lambda_{i}A)P - P'BP} = \abs{P'}\abs{A}\abs{\lambda_{i} I - A^{-1}B}\abs{P} = 0, \quad i = 1, 2, \dots, n,
          \end{align*}
          这说明对任意 $ i = 1, 2, \dots, n $, 都有 $ \abs{\lambda_{i} I - A^{-1}B} = 0 $, 也就是说 $ \lambda_{i} $ 均为 $ A^{-1}B $ 的特征值.
          \item 由于 $ A, B $ 都是实半正定阵, 于是存在实矩阵 $ C $, $ D $, 使得 $ A = C'C,\ B = D'D $, 那么由迹的性质可知
          \begin{align*}
              \tr(AB) = \tr(C'CD'D) = \tr(CD'DC') = \tr((CD')(CD')') \ge 0,
          \end{align*}
          最后一个不等号可由 \ref{item:A2=AA'} 题的答案得知. 同时等号成立时 $ CD' = 0 $, 这可以得到 $ AB = C'CD'D = 0 $, 又当 $ AB = 0 $ 时, 显然有 $ \tr(AB) = 0 $, 故等号成立的充分必要条件为 $ AB = 0 $.
      \end{answersheet}
  \end{answer}
  % TODO Duplicate of 83
  \item 设 $ f(x) $ 在 $ [a, b] $ 上二阶可导, 且 $ f''(x) > 0 $, 证明
  \begin{align*}
      f\qty(\frac{a + b}{2}) \le \frac{1}{b - a} \int_{a}^{b} f(x) \dx*.
  \end{align*}
  \begin{hint}
      用两种方法对 $ \int_{a}^{b} f(x) \dx* $ 进行换元: $ x = a + \lambda(b - a) $ 与 $ x = b - \lambda(b - a) $,
  \end{hint}
  \begin{answer}
      由 $ f''(x) > 0 $ 可知 $ f(x) $ 为 $ [a, b] $ 上的凸函数. 首先令 $ x = a + \lambda(b - a)\,(\lambda \in (0, 1)) $, 则
      \begin{align*}
          \frac{1}{b - a}\int_{a}^{b} f(x) \dx* = \int_{0}^{1} f(a + \lambda(b - a)) \dd{\lambda},
      \end{align*}
      同理令 $ x = b - \lambda(b - a) $, 又有
      \begin{align*}
          \frac{1}{b - a}\int_{a}^{b} f(x) \dx* = \int_{0}^{1} f(b - \lambda(b - a)) \dd{\lambda},
      \end{align*}
      从而
      \begin{align*}
          \frac{1}{b - a}\int_{a}^{b} f(x) \dx* = \frac{1}{2} \int_{0}^{1} f(a + \lambda(b - a)) + f(b - \lambda(b - a)) \dd{\lambda}
      \end{align*}
      注意到 $ a + \lambda(b - a) $ 与 $ b - \lambda(b - a) $ 关于 $ (a + b)/2 $ 对称, 由 $ f(x) $ 是凸函数, 则有
      \begin{align*}
          \frac{1}{2}(f(a + \lambda(b - a)) + f(b - \lambda(b - a))) \ge f\qty(\frac{a + b}{2}),
      \end{align*}
      这样就可以知道
      \begin{align*}
          f\qty(\frac{a + b}{2}) \le \frac{1}{b - a} \int_{a}^{b} f(x) \dx*.
      \end{align*}
      \textcolor{red}{*}本题亦可采用换元 $x \to a+b-x$, 于是
      \begin{align*}
        \int_{a}^{b} f(x) \dx* &= \int_{a}^{b} \frac{f(x)+f(a+b-x)}{2} \dx* \\
        &\ge \int_{a}^{b} f\qty(\frac{a+b}{2}) \dx*\\
        &=(b-a) f\qty(\frac{a+b}{2})
      \end{align*}
  \end{answer}
  \item 求极限 $ \limit{n}{\infty} \sin^{2}\qty(\pi \sqrt{n^{2} + n}) $.
  \begin{hint}
      注意 $ \sin x = (-1)^n \sin (x - n \pi) $.
  \end{hint}
  \begin{answer}
      首先可以得到
      \begin{align*}
          \sin^{2}\qty(\pi \sqrt{n^{2} + n}) = \sin^{2}\qty(\pi \qty(\sqrt{n^{2} + n} - n)) = \sin^{2}\qty(\frac{n\pi}{\sqrt{n^{2} + n} + n})
      \end{align*}
      于是
      \begin{align*}
          \limit{n}{\infty}\sin^{2}\qty(\pi \sqrt{n^{2} + n}) = \sin^{2}\qty(\limit{n}{\infty}\frac{n\pi}{\sqrt{n^{2} + n} + n}) = \sin^{2}\qty(\frac{\pi}{2}) = 1.
      \end{align*}
  \end{answer}
  \item $ f(x) $ 在 $ [a, b] $ 上二阶可导, 证明存在 $ \xi \in (a, b) $, 使得
  \begin{align*}
      f(b) - 2f\qty(\frac{a + b}{2}) + f(a) = \frac{1}{4} (b - a)^{2} f''(\xi),
  \end{align*}
  \item 设 $ f(x) $ 在 $ [a, b] $ 上二阶可导, 且 $ f(a) = f(b)  = 0 $, 证明对每个 $ x \in (a, b) $, 都存在对应的 $ \xi \in (a, b) $, 使得
  \begin{align*}
      f(x) = \frac{f''(\xi)}{2} (x - a) (x - b).
  \end{align*}
  \item 设 $ f(x) $ 在 $ [a, b] $ 上三阶可导, 证明存在 $ \xi \in (a, b) $, 使得
  \begin{align*}
      f(b) = f(a) + \frac{1}{2} (b - a) [f'(a) + f'(b)] - \frac{1}{12} (b - a)^{3} f'''(\xi).
  \end{align*}
  \item 求 $ \limit{n}{\infty} n \qty(\pi/4 - x_{n}) $, 其中:
  \begin{align*}
      x_{n} = \frac{n}{n^{2} + 1} + \frac{n}{n^{2} + 2^{2}} + \dots + \frac{n}{n^{2} + n^{2}}.
  \end{align*}
  \begin{hint}
    将积分写成Riemann和的形式, 然后使用中值定理即可
  \end{hint}
  \begin{answer}
    更加一般地,我们令
    \begin{align*}
      x_{n}=\frac{1}{n} \sum_{k=1}^{n} f\left(\frac{k}{n}\right)  
    \end{align*}
   计算(其中 $f(x)$ 在 $[a,b]$ 上有连续的导数)
  \begin{align*}
    \limit{n}{\infty} n\left(\int_{0}^{1} f(x) \dx*-\frac{1}{n} \sum_{k=1}^{n} f\left(\frac{k}{n}\right)\right)
  \end{align*}
  我们有
  \begin{align*}
    \int_{0}^{1} f(x) \dx* &=\sum_{k=1}^{n} \int_{\frac{k-1}{n}}^{\frac{k}{n}} f(x) \dx* \\
    \frac{1}{n} \sum_{k=1}^{n} f\left(\frac{k}{n}\right) &=\sum_{k=1}^{n} \int_{\frac{k-1}{n}}^{\frac{k}{n}} f\left(\frac{k}{n}\right) \dx*
  \end{align*}
  于是
  \begin{align*}
    \int_{0}^{1} f(x) \dx*-\frac{1}{n} \sum_{k=1}^{n} f\left(\frac{k}{n}\right) 
    &=\sum_{k=1}^{n} \int_{\frac{k-1}{n}}^{\frac{k}{n}}\left[f(x)-f\left(\frac{k}{n}\right)\right] \dx* \\
    &=-\sum_{k=1}^{n} \int_{\frac{k-1}{n}}^{\frac{k}{n}} f^{\prime}\left(\xi_{k}\right)\left(\frac{k}{n}-x\right) \dx* \quad \xi_{k} \in\left(x, \frac{k}{n}\right) 
  \end{align*}
  再由 $f'(x)$ 的连续性可知, 存在 $m_k,M_k$ 使得 
  \begin{align*}
    m_k \le f'(x) \le M_k 
  \end{align*}
  那么 
  \begin{align*}
    \int_{\frac{k-1}{n}}^{\frac{k}{n}} m_k \left(\frac{k}{n}-x\right) \dx* \le \int_{\frac{k-1}{n}}^{\frac{k}{n}} f^{\prime}\left(\xi_{k}\right)\left(\frac{k}{n}-x\right) \dx* \le  \int_{\frac{k-1}{n}}^{\frac{k}{n}} M_k \left(\frac{k}{n}-x\right) \dx*
  \end{align*}
  也即
  \begin{align*}
    m_k \le \frac{\int_{\frac{k-1}{n}}^{\frac{k}{n}} f^{\prime}\left(\xi_{k}\right)\left(\frac{k}{n}-x\right) \dx*}{\frac{1}{2n^2}} \le M_k
  \end{align*}
  存在 $\eta_k$ 使得 
  \begin{align*}
    &-\sum_{k=1}^{n} \int_{\frac{k-1}{n}}^{\frac{k}{n}} f^{\prime}\left(\xi_{k}\right)\left(\frac{k}{n}-x\right) \dx* \quad \xi_{k} \in\left(x, \frac{k}{n}\right) \\
    =&-\sum_{k=1}^{n} f^{\prime}\left(\eta_{k}\right) \int_{\frac{k-1}{n}}^{\frac{k}{n}}\left(\frac{k}{n}-x\right) \dx* \quad \eta_{k} \in\left(\frac{k-1}{n}, \frac{k}{n}\right) \\
    =&-\sum_{k=1}^{n} f^{\prime}\left(\eta_{k}\right)\left(\frac{1}{2 n^{2}}\right)
  \end{align*}
  再有
  \begin{align*}
    \limit{n}{\infty} n\left(\int_{0}^{1} f(x) \dx*-\frac{1}{n} \sum_{k=1}^{n} f\left(\frac{k}{n}\right)\right) 
    &=-\frac{1}{2 n} \sum_{k=1}^{n} f^{\prime}\left(\eta_{k}\right) \\
    & \rightarrow-\frac{1}{2} \int_{0}^{1} f^{\prime}(x) \dx* \quad(n \rightarrow \infty) \\
    &=\frac{1}{2}(f(0)-(1))
  \end{align*}
  本题只需令
  \begin{align*}
    f(x)=\frac{1}{1+x^{2}}
  \end{align*}
  即可得到
  \begin{align*}
    n\left(\frac{\pi}{4}-x_{n}\right) \to \frac{1}{4} \quad(n \to \infty)
  \end{align*}
  \end{answer}
  \item 如果级数 $ \sum_{n = 1}^{\infty} a_{n} $ 收敛, $ \limit{n}{\infty} p_{n} = \infty $, 证明极限
  \begin{align*}
      \limit{n}{\infty}\frac{a_{1} p_{1} + a_{2} p_{2} + \dots + a_{n} p_{n}}{p_{n}} = 0.
  \end{align*}
  \begin{hint}
      利用 Abel 变换与 Stolz 公式.
  \end{hint}
  \begin{answer}
      由 Abel 变换可得
      \begin{align*}
          \frac{\sum_{k = 1}^{n} a_{k}p_{k}}{p_{n}} = \frac{S_{n}p_{n} - \sum_{k = 1}^{n - 1}S_{k}(p_{k + 1} - p_{k})}{p_{n}} = S_{n} - \frac{\sum_{k = 1}^{n - 1}S_{k}(p_{k + 1} - p_{k})}{p_{n}}
      \end{align*}
      其中 $ S_{k} = \sum_{i = 1}^{k}a_{k} $, 由于 $ \sum_{n = 1}^{\infty}a_{n} $ 收敛, 那么由 Stolz 定理可得
      \begin{align*}
          \limit{n}{\infty}\frac{\sum_{k = 1}^{n} a_{k}p_{k}}{p_{n}} = \limit{n}{\infty}S_{n} - \limit{n}{\infty}\frac{S_{n - 1}(p_{n} - p_{n - 1})}{p_{n} - p_{n - 1}} = \limit{n}{\infty}S_{n} - \limit{n}{\infty}S_{n - 1} = 0.
      \end{align*}
  \end{answer}
  \item 如果级数 $ \sum_{n = 1}^{\infty} a_{n} $ 收敛, 证明极限
  \begin{align*}
      \limit{n}{\infty}\qty(n! a_{1} a_{2} \dots a_{n})^{1/n} = 0.
  \end{align*}
  \begin{hint}
    先证明 $na_n$ 是无穷小, 再取对数 stolz 即可
  \end{hint}
  \begin{answer}
    由于 $ \sum_{n = 1}^{\infty} a_{n} $ 收敛, 易知 $na_n $ 极限为零, 理由如下

    反证法, 若 $\limit{n}{\infty}na_n = a(a\neq 0)$, 那么 
    \begin{align*}
        a_n =O\qty(\frac{1}{n}) \quad n\to \infty
    \end{align*}
    由调和级数的敛散性可知矛盾.

    设 
    \begin{align*}
        x_n = \ln \qty(n! a_{1} a_{2} \dots a_{n})^{1/n} = \frac{\ln{a_1}+\ln{2a_2}+\cdots+\ln{na_n}}{n} \to -\infty \quad n\to \infty
    \end{align*}
    也即
    \begin{align*}
        \limit{n}{\infty}\qty(n! a_{1} a_{2} \dots a_{n})^{1/n} = \limit{n}{\infty}\exp{(x_n)} = 0
    \end{align*}
  \end{answer}
  \item \emph{面积原理}
  \begin{exercise}
      \item 设 $ f $ 是一个非负的递增函数, 则当 $ \xi \ge m $ 时有
      \begin{align*}
          \abs{\sum_{k = m}^{[\xi]} f(k) - \int_{m}^{\xi} f(x) \dx*} \le f(\xi).
      \end{align*}
      \item 设 $ f $ 是一个非负的递减函数, 则极限
      \begin{align*}
          \limit{\xi}{\infty} \qty(\sum_{k = m}^{[\xi]} f(k) - \int_{m}^{\xi} f(x) \dx*) = \alpha
      \end{align*}
      存在, 且 $ 0 \le \alpha \le f(m) $. 更进一步, 如果 $ \limit{x}{+\infty} f(x) = 0 $, 那么
      \begin{align*}
          \abs{\sum_{k = m}^{[\xi]} f(k) - \int_{m}^{\xi} f(x) \dx* - \alpha} \le f(\xi - 1),
      \end{align*}
      这里 $ \xi \ge m + 1 $.
  \end{exercise}
  \item 设 $ f(x) $ 在 $ [a, b] $ 上二次可微, 且 $ f(a)f(b) < 0 $, 对任意 $ x \in [a, b] $ 都有 $ f'(x) > 0 $, $ f''(x) > 0 $. 证明序列 $ \set{x_{n}} $ 极限存在, 其中 $ x_{1} \in [a, b] $, $ x_{n + 1} = x_{n} - f(x_{n})/f'(x_{n})\,(n = 1, 2, \dots) $, 进而可以证明此极限为方程 $ f(x) = 0 $ 的根.
  \item 设正项级数 $ \sum_{n = 1}^{\infty} a_{n} $ 收敛, 数列 $ \set{y_{n}} : y_{1} = 1,\ 2y_{n + 1} = y_{n} + \sqrt{y_{n}^{2} + a_{n}} $\, $ (n = 1, 2, \dots) $. 证明 $ \set{y_{n}} $ 是单调递增的收敛数列.
  \item 设数列 $ \set{x_{n}} $ 满足: 当 $ n < m $ 时, $ \abs{x_{n} - x_{m}} > 1/n $. 证明数列 $ \set{x_{n}} $ 无界.
  \item 设 $ f(x) $ 在闭区间 $ [0, 1] $ 上具有二阶导数, 且 $ f(0) = f'(0) = f(1) = 0 $, 证明: 存在 $ \xi \in (0, 1) $, 使得 $ f''(\xi) = f(\xi) $.
  \begin{hint}
      从结果入手, 即需要找 $ g(x) = \me^{x}(f'(x) - f(x)) $ 导数的零点, 那么由 Rolle 定理, 我们就需要找 $ g(x_{1}) = g(x_{2}) $, 我们已经有了 $ g(0) = 0 $, 那么只要找另一个 $ g(\xi) = 0 $ 即可, 也就等价于要找 $ \xi $ 使得 $ f'(\xi) = f(\xi) $.
  \end{hint}
  \begin{answer}
      答案这里我们依然从结果入手, 提供一种思考的方式. 我们要说明 $ f''(x) - f(x) $ 存在零点, 那么就是要找
      \begin{align*}
          f''(x) - f'(x) + f'(x) - f(x) = (f'(x) - f(x))' + (f'(x) - f(x))
      \end{align*}
      的零点, 也就是找到 $ g(x) = \me^{x}(f'(x) - f(x)) $ 导数的零点, 那么只要找到不同的 $ x_{1} $ 和 $ x_{2} $ 使得 $ g(x_{1}) = g(x_{2}) $ 即可. 而我们又知道 $ g(0) = 0 $, 那么我们只要找 $ g(x) $ 的另外一个零点即可, 也就是要找 $ f'(x) - f(x) $ 的零点, 也就是要找 $ h(x) = f(x)/\me^{x} $ 的导数的零点, 而由题知 $ h(0) = h(1) = 0 $, 也就是存在 $ \eta\in(0, 1) $ 使得 $ h'(\eta) = 0 $, 即 $ g(\eta) = g(0) = 0 $, 那么就存在 $ \xi \in (0, \pi) $, 使得 $ g'(\xi) = 0 $, 即 $ f''(\xi) = f(\xi) $.
  \end{answer}
  \item $ n $ 阶方阵的每行之和与每列之和均为 0, 证明其所有代数余子式全相等.
  \item 设函数 $ f(x) $ 定义在 $ (a, +\infty) $, 且 $ f(x) $ 在每个有限区间 $ (a, b) $ 内都有界, 并满足
  \begin{align*}
      \limit{x}{+\infty} \qty(f(x + 1) - f(x)) = A.
  \end{align*}
  证明 $ \limit{x}{+\infty} (f(x) / x) = A $.
  \begin{hint}
      由极限的定义可知 $ f(x + 1) - f(x) $ 的范围, 那么就可以得到 $ f(x) - f(x - k) $ 的范围, 再利用有界性可得 $ f(x)/x $ 的范围.
  \end{hint}
  \begin{answer}
      由 $ \limit{x}{+\infty} (f(x + 1) - f(x)) = A $ 可知, 对任意 $ \varepsilon > 0 $, 都存在 $ N \in \N $, 使得当 $ x > N $ 时
      \begin{align*}
          A - \varepsilon \le f(x + 1) - f(x) \le A + \varepsilon,
      \end{align*}
      设 $ k(x) = [x] - N $, 这样 $ x - k(x) \in [N, N + 1] $, 那么由
      \begin{align*}
          f(x) - f(x - k(x)) = \sum_{i = 1}^{k(x)}(f(x - k(x) + i) - f(x - k(x) + i - 1))
      \end{align*}
      可知
      \begin{align*}
          k(x)(A - \varepsilon) \le f(x) - f(x - k(x)) \le k(x)(A + \varepsilon),
      \end{align*}
      设存在 $ M > 0 $, 对于任意 $ x \in [N, N + 1] $, 都有 $ \abs{f(x)} \le M $, 于是就有 $ \abs{f(x - k(x))} \le M $, 于是就有
      \begin{align*}
          k(x)(A - \varepsilon) - M \le f(x) \le k(x)(A + \varepsilon) + M,
      \end{align*}
      不等式中同时除以 $ x $, 可得
      \begin{align}\label{eq:kx/xA-e}
          \frac{k(x)}{x}(A - \varepsilon) - \frac{M}{x} & \le \frac{f(x)}{x} \le \frac{k(x)}{x}(A + \varepsilon) + \frac{M}{x}
      \end{align}
      而由 $ k(x) $ 的定义可知 $ \limit{x}{+\infty}k(x)/x = 1 $, 于是在不等式 \eqref{eq:kx/xA-e} 中同时对 $ x \to +\infty $ 取上下极限:
      \begin{align*}
          A - \varepsilon \le \liminf_{x \to +\infty} \frac{f(x)}{x} \le \limsup_{x \to +\infty} \frac{f(x)}{x} \le A + \varepsilon,
      \end{align*}
      而由 $ \varepsilon $ 的任意性可知 $ \limit{x}{+\infty} f(x)/x = A $.
  \end{answer}
  \item 证明: \begin{exercise}
      \item 关于 $ x $ 的方程 $ \sum_{k=1}^{n} \me^{kx} = n + 1 $ 在 $ (0, 1) $ 上存在唯一的实根 $ a_{n} $;
      \item 数列 $ \set{a_{n}} $ 收敛, 并求其极限.
  \end{exercise}
  \item 设 $ a > 0 $, 求积分
  \begin{align*}
      \int_{0}^{\pi/2} \frac{1}{\sqrt{x}} \dx* \int_{\sqrt{x}}^{\sqrt{\pi/2}} \frac{1}{1 + \tan^{a}y^{2}} \dd{y}
  \end{align*}
  \item 设 $ \alpha, \beta $ 是 $ n $ 维列向量, $ A $ 是 $ n $ 阶方阵, 求证: $ \abs{A + \alpha\beta'} = \abs{A} + \beta'A^{*}\alpha $ .
  \item 设 $ A \in \MM[\R][3\times 2] $, $ B \in \MM[\R][2 \times 3] $, 且
  \begin{align*}
      AB = \begin{pmatrix}
          8 & 2 & -2 \\
          2 & 5 & 4 \\
          -2 & 4 & 5
      \end{pmatrix},
  \end{align*}
  求证 $ BA = 9 I $.
  \item $ A\in \MM $, 假设 $ A^{2} = A $, 且对任意的 $ x \in \R^{n} $, 有 $ x'A'Ax \le x'x $, 证明 $ A $ 为对称阵.
  \begin{hint}
      移项后可得 $ I - A'A $ 为半正定阵, 故存在 $ C \in \MM $, 使得 $ I - A'A = C'C $, 可得出 $ CA = 0 $, 即 $ C'CA = 0 $. 再利用 $ A^{2} = A $ 即可得出结论.
  \end{hint}
  \begin{answer}
      将不等式移项, 可得出对任意 $ x \in \R^{n} $, 都有
      \begin{align*}
          x'(I - A'A)x \ge 0,
      \end{align*}
      又因为 $ I - A'A $ 为对称阵, 所以 $ I - A'A $ 为半正定矩阵, 那么存在 $ C \in \MM $, 使得 $ I - A'A = C'C $, 在等式两侧同时左乘 $ A' $, 右乘 $ A $, 就有
      \begin{align*}
          A'(I - A'A)A = A'C'CA \implies (CA)'(CA) = A'A - (A')^{2}A^{2} = A'A - A'A = 0,
      \end{align*}
      那么由 $ \tr((CA)'(CA)) = \tr(0) = 0 $ 可得 $ CA = 0 $, 也就是
      \begin{align*}
          0 = CA = C'CA = (I - A'A)A = A - A'A \implies A = A'A,
      \end{align*}
      再两侧同时取转置, 就有 $ A' = A'A = A $, 即 $ A $ 是对称阵.
  \end{answer}
  \item 设 $ A \in \MM[\R][m \times n] $, 都有 $ \rank(AA') = \rank(A) $.
  \item $ f(x) \in C[a, b] $, 证明函数 $ m(x) = \min\limits_{a \le \xi \le x}f(\xi) $ 在 $ [a, b] $ 连续.
  \item 设 $ f(x) $ 是 $ [0, 1] $ 上的单调增函数, 且$ f(0) > 0 $, $ f(1) < 1 $, $ k \in \N $, 证明存在 $ x_{0} \in (0, 1) $, 使得 $ f(x_{0}) = x_{0}^{k} $.
  \begin{hint}
      注意这里没有 $ f(x) $ 连续的条件, 所以不可以用介值定理, 构造 $ g(x) = f(x) - x^{k} $, 然后对 $ [0, 1] $ 不断二分, 利用闭区间套以及 $ f $ 的单调性即可得出结论.
  \end{hint}
  \begin{answer}
      构造 $ g(x) = f(x) - x^{k} $, 那么我们的目标就变成了证明 $ g(x) $ 存在零点.

      于是我们记 $ a_{1} = 0 $, $ b_{1} = 1 $, 那么有 $ g(a_{1}) > 0,\ g(b_{1}) < 0 $, 于是我们取 $ c_{1} = (a_{1} + b_{1})/2 $, 如果 $ g(c_{1}) = 0 $, 那么结论成立. 否则如果 $ g(c_{1}) > 0 $, 则令 $ [a_{2}, b_{2}] = [c_{1}, b_{1}] $, 否则 $ [a_{2}, b_{2}] = [a_{1}, c_{1}] $, 再取 $ c_{2} = (a_{2} + b_{2})/2 $, 如此做下去, 如果在某个 $ c_{m} $ 处使得 $ g(c_{m}) = 0 $, 那么结论成立, 否则我们可以得到一列闭区间 $ \set{[a_{n}, b_{n}]} $, 满足
      \begin{enumerate}
          \item $ [a_{n + 1}, b_{n + 1}] \subset [a_{n}, b_{n}] $,
          \item $ g(a_{n}) > 0 $, $ g(b_{n}) < 0 $,
          \item $ \limit{n}{\infty}(b_{n} - a_{n}) = 0 $.
      \end{enumerate}
      那么由区间套定理可知存在唯一的 $ c \in \bigcap_{n = 1}^{\infty} [a_{n}, b_{n}] $, 下面说明 $ g(c) = 0 $, 如果 $ g(c) \ne 0 $, 那么由 $ f(x) $ 的单调性可知
      \begin{align*}
          a_{n}^{k} < f(a_{n}) < f(c) < f(b_{n}) < b_{n}^{k},
      \end{align*}
      由于 $ a_{n}^{k}, b_{n}^{k} \to c^{k}\,(n \to \infty) $, 于是对不等式两侧同时取极限 $ n \to \infty $, 就有 $ f(c) = c^{k} $, 结论成立.
  \end{answer}
  \item $ f(x) $ 在 $ \R $ 上三阶连续可导, 且对任意的 $ h > 0 $, 有
  \begin{align*}
      \frac{f(x + h) - f(x)}{h} = f'\qty(x + \frac{h}{2})
  \end{align*}
  求证: $ f(x) $ 为次数至多为 2 的多项式.
  \item 设实矩阵 $ A = A' $, 证明 $ A $ 可逆当且仅当存在实矩阵 $ B $ 使得 $ AB + B'A $ 正定.
  \begin{hint}
      $ (\implies) $ 取 $ B = A^{-1} $ 即可

      $ (\impliedby) $ 反证 $ A $ 不可逆, 则考虑 $ x'(AB + B'A)x $, 其中 $ x $ 满足 $ Ax = 0 $.
  \end{hint}
  \begin{answer}
      $ (\implies) $ 若 $ A $ 可逆, 直接取 $ B = A^{-1} $, 就有 $ AB + B'A = 2I $ 为正定阵.

      $ (\impliedby) $ 用反证法. 若 $ A $ 不可逆, 则存在非零向量 $ x $ 使得 $ Ax = 0 $, 那么就有
      \begin{align*}
          x'(AB + B'A)x = (A'x)'B + B'(Ax) = 0,
      \end{align*}
      这与 $ AB + B'A $ 正定矛盾, 故 $ A $ 可逆.
  \end{answer}
  % TODO Duplicate of 75
  \item
  \item 设 $ f(x) $ 在 $ [1, +\infty) $ 上一阶连续可导, 且
  \begin{align*}
      f'(x) = \frac{1}{1 + f^{2}(x)}\qty(\frac{1}{\sqrt{x}} - \sqrt{\ln\qty(1 + \frac{1}{x})})
  \end{align*}
  证明: $ \limit{x}{+\infty}f(x) $ 存在.
  \begin{hint}
      这里给出两种方法:
      \begin{method}
      \item 把 $ f(x) $ 和 $ f'(x) $ 移到一侧, 并且对两侧同时在 $ [1, +\infty) $ 积分. 再注意到 $ g(x) = x^{3}/3 + x $ 导数的性质, 即可得到结论.
      \item 由等式知 $ f'(x) > 0 $, 而右侧可以把 $ (1 + f^{2}(x))^{-1} $ 放掉, 再利用比较判别法可以知道.
      \end{method}
  \end{hint}
  \begin{answer}
      这里给出两种方法:
      \begin{method}
          \item\label{method:移项积分} 将等式移项, 可得
          \begin{align}\label{eq:f2x+1f'x}
              (f^{2}(x) + 1)f'(x) = \frac{1}{\sqrt{x}} - \sqrt{\ln\qty(1 + \frac{1}{x})}
          \end{align}
          我们把右侧记为 $ h(x) $. 由于 $ 1/x > \ln(1 + 1/x) $, 所以 $ h(x) > 0 $, 下面来确定 $ h(x) $ 当 $ x \to +\infty $ 时的阶数, 由 Taylor 定理可知
          \begin{align*}
              \frac{1}{\sqrt{x}} - \sqrt{\ln\qty(1 + \frac{1}{x})} & = \frac{1}{\sqrt{x}} - \sqrt{\frac{1}{x} - \frac{1}{2x^{2}} + o(x^{-2})} \\
              & = \frac{1}{\sqrt{x}}\qty(1 - \sqrt{1 - \frac{1}{2x} + o(x^{-1})})\\
              & = \frac{1}{\sqrt{x}}\qty(1 - \qty(1 - \frac{1}{2} \cdot \frac{1}{2x} + o(x^{-1})))\\
              & \sim x^{-3/2}, \quad x \to +\infty,
          \end{align*}
          所以由比较判别法可知 $ \int_{1}^{+\infty} h(x) \dx* $ 收敛, 于是等式 \eqref{eq:f2x+1f'x} 左侧的无穷积分也收敛, 即
          \begin{align*}
              \int_{1}^{+\infty} (f^{2}(x) + 1)f'(x) \dx* = \limit{x}{+\infty} \qty(\frac{1}{3}f^{3}(x) + f(x)) - \frac{1}{3}f^{3}(1) - f(1)
          \end{align*}
          存在, 也就是说 $ \limit{x}{+\infty} g(f(x)) $ 存在, 其中 $ g(x) = x^{3}/3 + x $. 下面来说明 $ \limit{x}{+\infty} f(x) $ 存在.

          首先由 $ g'(x) = x^{2} + 1 > 2\,(x \ge 1) $ 可知 $ g(x) $ 在 $ [1, +\infty) $ 单调递增趋于 $ +\infty $. 先设 $ \limit{x}{+\infty} g(f(x)) = A $, 由极限的保号性可知 $ A \in [g(1), +\infty) $, 那么就存在 $ a \in [1, +\infty) $, 使得 $ g(a) = A $, 然后用反证法说明 $ \limit{x}{+\infty} f(x) = a $. 如果 $ \limit{x}{+\infty}f(x) $ 不存在或不以 $ a $ 为极限, 那么存在 $ \varepsilon_{0} $, 使得对任意 $ n \in \N $, 都存在 $ x_{n} > n $, 使得 $ \abs{f(x_{n}) - a} > \varepsilon_{0} $, 那么由 Lagrange 中值定理可知存在位于 $ a $ 与 $ f(x_{n}) $ 之间的 $ \xi_{n} $, 使得
          \begin{align*}
              \abs{g(f(x_{n})) - g(a)} = g'(\xi_{n})\abs{f(x_{n}) - a} > 2\varepsilon_{0},
          \end{align*}
          这与 $ \limit{x}{+\infty}g(f(x_{n})) = g(a) = A $ 矛盾, 故 $ \limit{x}{+\infty} f(x) = a $ 存在.
          \item 首先依旧由 $ 1/x > \ln (1 + 1/x) $ 可知等式右侧恒正, 那么就有 $ f'(x) > 0\,(x \in [1, +\infty)) $, 又有
          \begin{align*}
              f'(x) < \frac{1}{\sqrt{x}} - \sqrt{\ln\qty(1 + \frac{1}{x})} := h(x)
          \end{align*}
          而从 \ref{method:移项积分} 中我们已经知道了 $ \int_{1}^{+\infty} h(x) \dx* $ 收敛, 再由比较判别法可知
          \begin{align*}
              \int_{1}^{+\infty} f'(x) \dx* = \limit{x}{+\infty}f(x) - f(1)
          \end{align*}
          收敛, 也即 $ \limit{x}{+\infty} f(x) $ 存在.
      \end{method}
  \end{answer}
  % TODO 两个极限
  \item\label{item:fx+n} 设 $ f(x) $ 在 $ [0, +\infty) $ 上一致连续, 若对于任意 $ x \in \R $, 都有 $ \limit{n}{\infty} f(x + n) = 0 $, 证明 $ \limit{x}{+\infty} f(x) = 0 $.
  \begin{hint}
      将 $ [0, 1] $ 分成一些小区间, 然后对任意 $ x > M $, 考虑 $ x - [x] \in [0, 1] $ 与 $ f(x) $ 在 $ [0, 1] $ 上的一致连续性即可.
  \end{hint}
  \begin{answer}
      由 $ f(x) $ 在 $ [0, +\infty) $ 上一致连续可知对任意 $ \varepsilon > 0 $, 存在 $ \delta > 0 $, 使得当 $ x_{1}, x_{2} > 0 $, 且 $ \abs{x_{1} - x_{2}} < \delta $ 时, 就有 $ \abs{f(x_{1}) - f(x_{2})} < \varepsilon $. 又对固定的 $ x_{0} \in [0, 1] $, 都有 $ \limit{n}{\infty} f(x_{0} + n) = 0 $, 于是对上述 $ \varepsilon $, 存在 $ N(x_{0}) \in \N $, 使得当 $ n \ge N(x_{0}) $ 时, 有 $ \abs{f(x_{0} + n)} < \varepsilon $, 那么搭配一致连续的条件, 可知对任意的 $ x \in (x_{0} - \delta, x_{0} + \delta) $ 都有
      \begin{align*}
          \abs{f(x + n)} \le \abs{f(x + n) - f(x_{0} + n)} + \abs{f(x_{0} + n)} < 2 \varepsilon,
      \end{align*}
      于是令 $ x_{0} $ 跑遍 $ [0, 1] $, 这时 $ \bigcup_{x_{0} \in [0, 1]} (x_{0} - \delta, x_{0} + \delta) $ 构成了 $ [0, 1] $ 的一个开覆盖, 由有限覆盖定理可知存在有限个开区间覆盖\footnote{这里也可以直接将 $ [0, 1] $ 等分成 $ [1/\delta] + 1 $ 个小区间} $ [0, 1] $, 设为
      \begin{align*}
          \bigcup_{i = 1}^{k} (x_{i} - \delta, x_{i} + \delta) \supset [0, 1]
      \end{align*}
      那么取 $ N = \max_{1 \le i \le k} N(x_{i}) $, 则对任意的 $ x \in [0, 1] $, 总存在 $ 1 \le i \le k $, 使得 $ x \in (x_{i} - \delta, x_{i} + \delta) $, 那么当 $ n \ge N \ge N(x_{i}) $ 时, 有
      \begin{align}\label{eq:absfx+n}
          \abs{f(x + n)} < 2\varepsilon,
      \end{align}
      于是我们任取 $ y > N $, 有 $ [y] \in \N $, 这里 $ [y] $ 为取整函数, 并且 $ [y] \ge N $, $ y - [y] \in [0, 1] $, 于是令 \eqref{eq:absfx+n} 中的 $ x = y - [y] $, $ n = [y] $, 就可以得到
      \begin{align*}
          \abs{f(y)} = \abs{f(y - [y] + [y])} < 2\varepsilon,
      \end{align*}
      这说明对任意 $ \varepsilon > 0 $, 都存在 $ N > 0 $, 当 $ y > N $ 时, $ \abs{f(y)} < 2\varepsilon $, 此即 $ \limit{y}{+\infty}f(y) = 0 $.
  \end{answer}
  \item 设 $ f(x) $ 在 $ [0, +\infty) $ 上一致连续, 且对任意 $ \delta > 0 $, 都有 $ \limit{n}{\infty} f(n\delta) = 0 $, 证明 $ \limit{x}{+\infty} f(x) = 0 $.
  \begin{hint}
      与 \ref{item:fx+n} 题类似, 首先由一致连续性可得一个 $ \delta $, 然后将这个 $ \delta $ 应用到 $ \limit{n}{\infty} f(n\delta) = 0 $ 中, 最后再对任意 $ x > M $ 进行说明.
  \end{hint}
  \begin{answer}
      由 $ f(x) $ 在 $ [0, +\infty) $ 上一致连续可知对任意 $ \varepsilon > 0 $, 存在 $ \delta > 0 $, 使得当 $ x_{1}, x_{2} > 0 $, 且 $ \abs{x_{1} - x_{2}} < \delta $ 时, 就有 $ \abs{f(x_{1}) - f(x_{2})} < \varepsilon $. 固定该 $ \delta $, 再由 $ \limit{n}{\infty}f(n\delta) = 0 $ 可知存在 $ N(\delta) \in \N $, 使得当 $ n > N(\delta) $ 时, $ \abs{f(n\delta)} < \varepsilon $.

      那么我们取 $ A = \delta \cdot N(\delta) $, 这样当 $ x > A $ 时, 存在 $ n \ge N(\delta) $, 使得
      \begin{align*}
          n\delta \le x < (n + 1)\delta,
      \end{align*}
      于是有
      \begin{align*}
          \abs{f(x)} \le \abs{f(x) - f(n\delta)} + \abs{f(n\delta)} < 2\varepsilon.
      \end{align*}
      这就说明对任意 $ \varepsilon > 0 $, 存在 $ A > 0 $, 当 $ x > A $ 时, $ \abs{f(x)} < 2\varepsilon $, 即 $ \limit{x}{+\infty} f(x) = 0 $.
  \end{answer}
  \begin{remark}
      本题不需要一致连续的条件也可以证明, 过程略为繁琐, 而 \ref{item:fx+n} 题如果没有一致连续的条件, 则有反例 $ f(x) = \frac{x\sin \pi x}{1 + x^{2} \sin^{2} \pi x} $
  \end{remark}
  \item\label{item:一致连续可以被控制} 设 $ f(x) $ 在 $ \R $ 上一致连续, 则存在正实数 $ a, b $, 使得 $ \abs{f(x)} \le a\abs{x} + b $.
  \begin{hint}
      由一致连续可以得到当 $ \abs{x - y} < \delta $ 时, $ \abs{f(x) - f(y)} < 1 $, 那么对任意 $ x \in \R $, 把 $ x $ 和 $ 0 $ 之间的距离分成 $ n(x) $ 个小区间, 使得每个区间的长度都小于 $ \delta $ 即可.
  \end{hint}
  \begin{answer}
      由 $ f(x) $ 在 $ \R $ 上一致连续可知, 对 $ \varepsilon_{0} = 1 $, 存在 $ \delta > 0 $, 使得当 $ \abs{x - y} < \delta $ 时, $ \abs{f(x) - f(y)} < 1 $, 那么对任意 $ x \in \R $, 将 $ x $ 到 $ 0 $ 的这个区间分为 $ n $ 个小区间, 其中
      \begin{align*}
          n = \frac{[\abs{x}]}{\delta} + 1
      \end{align*}
      这样每个小区间的长度 $ \abs{x}/n < \delta $, 于是就有
      \begin{align*}
          \abs{f(x)} & \le \abs{f(x) - f\qty(\frac{n - 1}{n}x)} + \abs{f\qty(\frac{n - 1}{n}x) - f\qty(\frac{n - 2}{n}x)} + \dots + \abs{f\qty(\frac{1}{n}x) - f(0)} + \abs{f(0)}\\
          & \le n + \abs{f(0)}
      \end{align*}
      而由 $ n $ 的定义可知
      \begin{align*}
          n = \frac{[\abs{x}]}{\delta} + 1 \le \frac{\abs{x}}{\delta} + 1,
      \end{align*}
      于是令 $ a = 1/\delta $, $ b = 1 + \abs{f(0)} $, 就有
      \begin{align*}
          \abs{f(x)} \le n + \abs{f(0)} \le \frac{\abs{x}}{\delta} + 1 + f(0) = a\abs{x} + b.
      \end{align*}
  \end{answer}
  \item 设 $ f(x) $ 在 $ [1, +\infty) $ 上一致连续, 证明 $ \abs{f(x)/x} $ 在 $ [1, +\infty) $ 有界.
  \begin{hint}
      这道题是 \ref{item:一致连续可以被控制} 题的直接推论.
  \end{hint}
  \begin{answer}
      由 \ref{item:一致连续可以被控制} 题可知存在 $ a, b > 0 $, 使得 $ \abs{f(x)} \le a\abs{x} + b $, 那么就有
      \begin{align*}
          \abs{\frac{f(x)}{x}} \le a + \frac{b}{\abs{x}} \le a + b,\quad x > 1.
      \end{align*}
      即 $ \abs{f(x)/x} $ 在 $ [1, +\infty) $ 上有界.
  \end{answer}
  \item 证明欧式空间中两标准正交基的过渡矩阵为正交阵.
  \hitem 设 $ \alpha $ 是欧式空间 $ V $ 中的一个非零向量, $ \alpha_{1}, \alpha_{2}, \dots, \alpha_{p} $ 是 $ V $ 中的 $ p $ 个向量, 满足
  \begin{align*}
      (\alpha_{i}, \alpha_{j}) \le 0,\ (\alpha_{i}, \alpha) > 0, \quad i, j = 1, 2, \dots, p, i \ne j
  \end{align*}
  证明
  \begin{exercise}
      \item\label{item:alphaalphap} $ \alpha_{1}, \alpha_{2}, \dots, \alpha_{p} $ 线性无关;
      \item $ n $ 维欧式空间中最多有 $ n + 1 $ 个向量, 使其两两互成钝角;
      \item $ n $ 维欧式空间中一定存在 $ n + 1 $ 个向量, 使其两两互为钝角.
  \end{exercise}
  \begin{hint}
      \begin{hintsheet}
          \item 假设 $ \alpha_{1}, \alpha_{2}, \dots, \alpha_{p} $ 线性相关, 即 $ \alpha_{p} $ 可以被其余向量线性表示, 然后将系数按正负分类, 然后用 $ (\alpha_{p}, \alpha) $ 导出矛盾.
          \item 设有 $ \alpha_{1}, \alpha_{2}, \dots, \alpha_{m} $ 个向量互为钝角, 那么令 $ \alpha = -\alpha_{m} $, 那么可验证其满足 \ref{item:alphaalphap} 的条件, 即可得到结论.
          \item 用数学归纳法, 先找 $ n - 1 $ 维子空间 $ V_{n - 1} $ 中的互为钝角的向量 $ \alpha_{1}, \alpha_{2}, \dots, \alpha_{n} $, 与 $ V_{n - 1}^{\bot} $ 中的向量 $ \beta $, 考虑
          \begin{align*}
              \beta, \alpha_{1} - t\beta, \alpha_{2} - t\beta, \dots, \alpha_{n} - t\beta
          \end{align*}
          可以求得满足一定条件的 $ t $ 就可以使得它们互成钝角.
      \end{hintsheet}
  \end{hint}
  \begin{answer}
      \begin{answersheet}
          \item 用反证法, 若 $ \alpha_{1}, \alpha_{2}, \dots, \alpha_{p} $ 线性相关, 则必有某一个向量可以被其余向量线性表示, 不妨设为 $ \alpha_{p} $, 即存在不全为 $ 0 $ 的 $ \lambda_{1}, \lambda_{2}, \dots, \lambda_{p - 1} $, 使得
          \begin{align*}
              \alpha_{p} = \lambda_{1}\alpha_{1} + \lambda_{2}\alpha_{2} + \dots + \lambda_{p - 1}\alpha_{p - 1}.
          \end{align*}
          我们将 $ \lambda_{i} $ 按正负进行分类, 设
          \begin{align*}
              \alpha_{p} = \sum_{i}\lambda_{i}\alpha_{i} + \sum_{j}\lambda_{j}\alpha_{j} := \beta + \gamma,\quad \lambda_{i} \le 0, \lambda_{j} > 0,
          \end{align*}
          首先可以知道 $ (\alpha, \alpha_{p}) = (\alpha, \beta) + (\alpha, \gamma) > 0 $, 又知道 $ (\alpha, \beta) = \sum_{i}(\alpha, \alpha_{i}) \le 0 $, 于是一定有 $ (\alpha, \gamma) > 0 $, 即 $ \gamma \ne 0 $, 于是我们用两种方法计算 $ (\gamma, \alpha_{p}) $ 来得到矛盾:
          \begin{align*}
              (\gamma, \alpha_{p}) & = (\gamma, \gamma) + (\gamma, \beta) = (\gamma, \gamma) + \sum_{i}\sum_{j}\lambda_{i}\lambda_{j}(\alpha_{i}, \alpha_{j}) > 0 \\
              (\gamma, \alpha_{p}) & = \sum_{j}\lambda_{j}(\alpha_{j}, \alpha_{p}) \le 0,
          \end{align*}
          矛盾, 这说明 $ \alpha_{1}, \alpha_{2}, \dots, \alpha_{p} $ 线性无关.
          \item 设 $ \alpha_{1}, \alpha_{2}, \dots, \alpha_{m} $ 互为钝角, 那么取 $ \alpha = -\alpha_{m} $, 于是对 $ \alpha_{1}, \alpha_{2}, \dots, \alpha_{m - 1} $ 就有
          \begin{align*}
              (\alpha_{i}, \alpha_{j}) < 0, (\alpha_{i}, a) > 0, \quad i, j = 1, 2, \dots, m - 1, i \ne j
          \end{align*}
          满足第 \ref{item:alphaalphap} 问的条件, 于是 $ \alpha_{1}, \alpha_{2}, \dots, \alpha_{m - 1} $ 线性无关, 又因为 $ V $ 是 $ n $ 维线性空间, 于是 $ m - 1 \le n $, 即 $ m \le n + 1 $.
          \item 用数学归纳法, 当 $ \dim V = 1 $ 时, 任取 $ \alpha_{1} \in V $, 取 $ \alpha_{2} = -\alpha_{1} $, 这样就有了 $ 2 $ 个互成钝角的向量. 假设当 $ \dim V \le n - 1 $ 时结论都成立, 那么当 $ \dim V = n $ 时, 考虑 $ V $ 的维数为 $ n - 1 $ 的子空间 $ V_{n - 1} $, 由归纳假设可知其中存在 $ n $ 个互成钝角的向量, 设为 $ \alpha_{1}, \alpha_{2}, \dots, \alpha_{n} $, 然后我们取 $ \beta \in V_{n - 1}^{\bot} $, 考虑以下 $ n + 1 $ 个向量:
          \begin{align}\label{eq:betaalpha1-tbeta}
              \beta, \alpha_{1} - t\beta, \alpha_{2} - t\beta, \dots, \alpha_{n} - t\beta
          \end{align}
          其中 $ t > 0 $, 那么有
          \begin{align*}
              (\beta, \alpha_{i} - t\beta) & = (\beta, \alpha_{i}) - t(\beta, \beta) < 0\\
              (\alpha_{i} - t\beta, \alpha_{j} - t\beta) & = (\alpha_{i}, \alpha_{j}) + t^{2}(\beta, \beta),
          \end{align*}
          由于 $ \beta, \alpha_{1}, \alpha_{2}, \dots, \alpha_{n} $ 均为确定的向量, 那么只要取满足
          \begin{align*}
              t^{2} < \abs{\frac{\min_{i, j}(\alpha_{i}, \alpha_{j})}{(\beta, \beta)}}
          \end{align*}
          的 $ t $, 就能使 \eqref{eq:betaalpha1-tbeta} 这 $ n + 1 $ 个向量互为钝角.
      \end{answersheet}
  \end{answer}
  \item 设 $ A, B \in \MM[\K] $, 且 $ AB = BA $, 利用线性方程组的知识证明
  \begin{align*}
      \rank(A + B) \le \rank(A) + \rank(B) - \rank(AB)
  \end{align*}
  \begin{hint}
      分别设 $ (A + B)x = 0, Ax = 0, Bx = 0, ABx = 0, BAx = 0 $ 的解空间为 $ V_{A+B}, V_{A}, V_{B}, V_{AB}, V_{BA} $, 注意到 $ V_{AB} = V_{BA} $, 再利用这些解空间的包含关系即可得到结论.
  \end{hint}
  \begin{answer}
      分别设 $ (A + B)x = 0, Ax = 0, Bx = 0, ABx = 0, BAx = 0 $ 的解空间为 $ V_{A+B}, V_{A}, V_{B}, V_{AB}, V_{BA} $, 由线性方程组的关系, 可以得到 $ V_{A} \cap V_{B} \subset V_{A + B} $, $ V_{A} \subset V_{BA}  $, $ V_{B} \subset V_{AB} $, 注意到 $ V_{AB} = V_{BA} $, 那么就有 $ V_{A} + V_{B} \subset V_{AB} $, 于是有
      \begin{align*}
          \dim(V_{A} + V_{B}) & = \dim V_{A} + \dim V_{B} - \dim(V_{A} \cap V_{B})\\
          \dim V_{AB} & \ge  \dim V_{A} + \dim V_{B} - \dim V_{A + B}
      \end{align*}
      再利用解空间和系数矩阵的关系:
      \begin{alignat*}{2}
          & n - \rank (AB) && \le  n - \rank A + n - \rank B - n + \rank (A + B) \\
          \implies & \rank(A + B) && \le \rank(A) + \rank(B) - \rank(AB)
      \end{alignat*}
  \end{answer}
  \item 设 $ B \in \MM[\C][n \times 2] $,
  \begin{align*}
      C = \begin{pmatrix}
          1 & 1 & \cdots & 1\\
          1 & 2 & \cdots & n
      \end{pmatrix}
  \end{align*}
  若 $ A = BC $, 且 $ CB $ 的特征多项式为 $ x^{2} - 2x + 1 $, 求 $ A $ 的特征值, 并求线性方程组 $ Ax = 0 $ 的基础解系.
  \begin{hint}
      由 $ \abs{\lambda I_{n} - BC} = \lambda^{n - 2}\abs{\lambda I_{2} - CB} $ 可求得 $ A $ 的特征值, 又因为 $ B $ 列满秩可知 $ BCx=0 $ 的解就是 $ Cx = 0 $ 的解.
  \end{hint}
  \begin{answer}
      先证明一个引理
      \begin{lemma}
          设 $ m \le n $, $ M $ 是 $ m \times n $ 阶矩阵, $ N $ 是 $ n \times m $ 阶矩阵, $ I_{n}, I_{m} $ 分别为 $ n $ 阶和 $ m $ 阶的单位阵, 则对于 $ \lambda \ne 0 $, 有 $ \abs{\lambda I_{n} - NM} = \lambda^{n - m}\abs{\lambda I_{m} - MN} $.
      \end{lemma}
      \begin{proof}
          对分块行列式做初等变换:
          \begin{align*}
              \begin{vmatrix}
                  \lambda I_{n} - NM & 0_{n \times m} \\ 0_{m \times n} & I_{m}
              \end{vmatrix}
              =
              \begin{vmatrix}
                  \lambda I_{n} - NM & N \\ 0_{m \times n} & I_{m}
              \end{vmatrix}
              =
              \begin{vmatrix}
                  \lambda I_{n} & N \\ M & I_{m}
              \end{vmatrix}
              =
              \begin{vmatrix}
                  \lambda I_{n} & 0_{n \times m} \\ M & I_{m} - \frac{1}{\lambda} MN
              \end{vmatrix},
          \end{align*}
          于是就有 $ \abs{\lambda I_{n} - NM} = \lambda^{n - m}\abs{\lambda I_{m} - MN} $. \qed
      \end{proof}

      将这个结论应用到这道题, 就可以得到
      \begin{align*}
          \abs{\lambda I_{n} - A} = \lambda^{n - 2}\abs{\lambda I_{2} - CB} = \lambda^{n - 2}(\lambda^{2} - 2\lambda + 1),
      \end{align*}
      可以解得 $ A $ 的特征值有 $ 1 $ ( $ 2 $ 重) 与 $ 0 $ ( $ n - 2 $ 重).

      首先由
      \begin{align*}
          2 = \rank(CB) \le \rank B \le 2
      \end{align*}
      可知 $ \rank B = 2 $, 即 $ B $ 是一个列满秩阵, 那么对于任意的 $ y \in \R^{2} $, 线性方程组 $ By = 0 $ 的解空间的维数为 $ 2 - \rank B = 0 $, 即 $ By = 0 $ 只有零解, 那么对线性方程组 $ B(Cx) = 0 $ 来说, 其与 $ Cx = 0 $ 同解, 那么它的解空间的一组基为
      \begin{align*}
          & (-1, 2, 1, 0, 0, \dots, 0)'\\
          & (-2, 3, 0, 1, 0, \dots, 0)'\\
          & \phantom{(-2, 3, 0, } \vdots \\
          & (2 - n, n - 1, 0, \dots, 0, 1)'.
      \end{align*}
  \end{answer}
  \item  计算 $ n $ 阶 $ b $ -- 循环行列式:
  \begin{align*}
      B = \begin{vmatrix}
          a_{1}   & a_{2} & a_{3} & \ldots    & a_{n} \\
          ba_{n}  & a_{1} & a_{2} & \ldots    & a_{n-1} \\
          ba_{n-1}    & ba_{n}    & a_{1} & \ldots & a_{n-1} \\
          \vdots  & \vdots & \vdots & \ddots & \vdots \\
          ba_{2} & ba_{3} & ba_{4} & \ldots & a_{1}
      \end{vmatrix}
  \end{align*}
  \begin{hint}
      利用多项式 $ f(x) = a_{1} + a_{2}x + \dots + a_{n}x^{n-1} $ 与 $ x^{n} = b $ 的根的特性来构造 Vandermonde 行列式.
  \end{hint}
  \begin{answer}
      设多项式 $ f(x) = a_{1} + a_{2}x + \dots + a_{n}x^{n-1} $, 再设 $ x^{n} = b $ 的根分别为 $ \omega_{1}, \omega_{2}, \dots, \omega_{n} $, 它们互不相同. 那么构造 Vandermonde 行列式
      \begin{align*}
          V = \begin{vmatrix}
              1 & 1 & \dots & 1\\
              \omega_{1} & \omega_{2} & \dots & \omega_{n}\\
              \omega_{1}^{2} & \omega_{2}^{2} & \dots & \omega_{n}^{2}\\
              \vdots & \vdots & \ddots & \vdots\\
              \omega_{1}^{n-1} & \omega_{2}^{n-1} & \dots & \omega_{n}^{n-1}
          \end{vmatrix}
      \end{align*}
      那么考虑行列式乘法:
      \begin{align*}
          BV =
          \begin{vmatrix}
              a_{1}   & a_{2} & a_{3} & \ldots    & a_{n} \\
              ba_{n}  & a_{1} & a_{2} & \ldots    & a_{n-1} \\
              ba_{n-1}    & ba_{n}    & a_{1} & \ldots & a_{n-1} \\
              \vdots  & \vdots & \vdots & \ddots & \vdots \\
              ba_{2} & ba_{3} & ba_{4} & \ldots & a_{1}
          \end{vmatrix}
          \begin{vmatrix}
              1 & 1 & \dots & 1\\
              \omega_{1} & \omega_{2} & \dots & \omega_{n}\\
              \omega_{1}^{2} & \omega_{2}^{2} & \dots & \omega_{n}^{2}\\
              \vdots & \vdots & \ddots & \vdots\\
              \omega_{1}^{n-1} & \omega_{2}^{n-1} & \dots & \omega_{n}^{n-1}
          \end{vmatrix}
      \end{align*}
      可以计算得 $ BV $ 的第 $ (i, j) $ 元为:
      \begin{align*}
          BV(i, j) & = (ba_{n-i+2}, ba_{n-i+3}, \dots, ba_{n}, a_{1}, \dots, a_{n-i+1})\begin{pmatrix}
              1 \\ \omega_{j} \\ \omega_{j}^{2} \\ \vdots \\ \omega_{j}^{n-1}
          \end{pmatrix}\\
          & = ba_{n-i+2} + b\omega_{j}a_{n-i+3} + \dots + b\omega_{j}^{i-2}a_{n} + \omega_{j}^{i-1}a_{1} + \dots + \omega_{j}^{n-1}a_{n-i+1}
      \end{align*}
      然后我们将 $ b $ 写为 $ \omega_{j}^{n} $, 并重新整理一下顺序, 就有
      \begin{align*}
          BV(i, j) & = \omega_{j}^{i-1}a_{1} + \omega_{j}^{i}a_{2} + \dots + \omega_{j}^{n-1}a_{n-i+1} + \omega_{j}^{n}a_{n-j+2} + \dots + \omega^{n+i-2}a_{n}\\
          & = \omega_{j}^{i-1}(a_{1} + \omega_{j}a_{2} + \dots + \omega_{j}^{n-1}a_{n}) = \omega_{j}^{i-1}f(\omega_{j})
      \end{align*}
      那么我们可以重新写一份 $ BV $:
      \begin{align*}
          BV = \begin{vmatrix}
              f(\omega_{1}) & f(\omega_{2}) & \dots & f(\omega_{n})\\
              \omega_{1}f(\omega_{1}) & \omega_{2}f(\omega_{2}) & \dots & \omega_{n}f(\omega_{n})\\
              \omega_{1}^{2}f(\omega_{1}) & \omega_{2}^{2}f(\omega_{2}) & \dots & \omega_{n}^{2}f(\omega_{n})\\
              \vdots & \vdots & \ddots & \vdots\\
              \omega_{1}^{n-1}f(\omega_{1}) & \omega_{2}^{n-1}f(\omega_{2}) & \dots & \omega_{n}^{n-1}f(\omega_{n})\\
          \end{vmatrix} = \prod_{i = 1}^{n}f(\omega_{i})
          \begin{vmatrix}
              1 & 1 & \dots & 1\\
              \omega_{1} & \omega_{2} & \dots & \omega_{n}\\
              \omega_{1}^{2} & \omega_{2}^{2} & \dots & \omega_{n}^{2}\\
              \vdots & \vdots & \ddots & \vdots\\
              \omega_{1}^{n-1} & \omega_{2}^{n-1} & \dots & \omega_{n}^{n-1}
          \end{vmatrix}= \prod_{i = 1}^{n}f(\omega_{i})V
      \end{align*}
      由于 $ \omega_{1}, \omega_{2}, \dots, \omega_{n} $ 互不相同, 则 $ V \ne 0 $, 那么就有 $ B = \prod_{i = 1}^{n}f(\omega_{i}) $.
  \end{answer}
  \sitem\label{item:反称加对角} 设 $ A $ 是 $ n $ 阶实反对称阵, $ D = \diag\qty{d_{1}, d_{2}, \dots, d_{n}} $ 是同阶的对角阵, 且 $ d_{i} > 0\,(i = 1, 2, \dots, n) $. 求证 $ \abs{A + D} > 0 $, 特别地, 若 $ B = A + D $, 其中 $ A $ 是反对称阵, $ D $ 是正定对称阵, 则 $ B $ 是行列式为正的非异阵.
  \begin{hint}
      考虑 $ (tA + D)x = 0 $ 的解的情况, 再利用 $ t = 0 $ 来判断符号.
  \end{hint}
  \begin{answer}
      设 $ t \ge 0 $, 记 $ M(t) = (tA + D) $, 考虑 $ x \in \R^{n} $ 使得 $ (tA + D)x = 0 $, 则有 $ x'(tA + D)x = 0 $, 对等式两侧取转置, 注意到 $ tA $ 也是反对称阵, 那么就有
      \begin{align*}
          x'(-tA + D)x = 0
      \end{align*}
      将两个等式相加, 就可以得到 $ 2x'Dx = 0 $, 显然 $ D $ 是一个正定矩阵, 那么可以得到 $ x = 0 $, 即 $ (tA + D)x = 0 $ 只有零解, 即 $ \abs{M(t)} $ 对任意 $ t $ 恒不为 $ 0 $, 而 $ \abs{M(t)} $ 是关于 $ t $ 的连续函数, 那么有连续函数的界值定理, 就有 $ \abs{M(t)} $ 恒正或恒负. 特别地, 考虑 $ t = 0 $, 由 $ d_{i} > 0 $ 可知 $ \abs{D} > 0 $, 这说明 $ \abs{M(t)} > 0 $, 那么取 $ t = 1 $, 就得到 $ \abs{M(1)} = \abs{A + D} > 0 $.
  \end{answer}
  % TODO 主对角严格占优
  \item 如果 $ n $ 阶方阵 $ A = (a_{ij}) $ 适合条件:
  \begin{align*}
      \abs{a_{ii}} > \sum_{\mathclap{j = 1,\ j \ne i}}^{n} \abs{a_{ij}}, \quad i = 1, 2, \dots, n,
  \end{align*}
  则称 $ A $ 为\emph{严格对角占优阵}, 求证, 严格对角占优阵必是满秩阵, 若上述条件改为:
  \begin{align*}
      a_{ii} > \sum_{\mathclap{j = 1,\ j \ne i}}^{n} \abs{a_{ij}}, \quad i = 1, 2, \dots, n,
  \end{align*}
  求证 $ \abs{A} > 0 $.
  \item 设 $ f(x) $ 在 $ [a, b] $ 上有定义, 对 $ [a, b] $ 上任意一个闭区间 $ [x_{1}, x_{2}] \subset [a, b] $, 对介于 $ f(x_{1}) $ 与 $ f(x_{2}) $ 之间的任一常数 $ l $, 方程
  \begin{align*}
      f(x) = l
  \end{align*}
  在 $ [x_{1}, x_{2}] $ 上有且仅有有限个解, 证明 $ f(x) \in C[a, b] $.
  \begin{hint}
      假设 $ f(x) $ 有间断点 $ c $, 用定义可以说明在 $ f(c) $ 附近存在 $ l_{c} $, 使得 $ f(x) = l_{c} $ 无根或有无数根.
  \end{hint}
  \begin{answer}
      假设 $ f(x) \notin C[a, b] $, 则设 $ c $ 是 $ f(x) $ 的间断点, 那么 $ \limit{x}{c+}f(x) $ 与 $ \limit{x}{c-} f(x) $ 至少一个不存在或不以 $ f(c) $ 为极限, 设它为 $ \limit{x}{c+}f(c) $ . 于是存在 $ \varepsilon_{0} $, 使得对任意的 $ \delta $, 都存在 $ x $, 使得 $ c < x < c + \delta $, 但是 $ \abs{f(x) - f(c)} \ge \varepsilon_{0} $, 那么我们考虑集合
      \begin{align*}
          M_{\delta} = \set{x \in (c, c+\delta) : f(x) \ge f(c) + \varepsilon_{0}}, \quad m_{\delta} = \set{x \in (c, c+\delta) : f(x) \le f(c) - \varepsilon_{0}}
      \end{align*}
      对于 $ \delta_{1} < \delta_{2} $, 显然有 $ M_{\delta_{1}} \subset M_{\delta_{2}} $, $ m_{\delta_{1}} \subset m_{\delta_{2}} $, 并且当 $ \delta $ 不断靠近 $ 0 $ 的过程中, $ M_{\delta} $ 与 $ m_{\delta} $ 一定有一个集合恒不为空集, 否则, 如果有某 $ \delta', \delta'' $ 分别使 $ M_{\delta'} = m_{\delta''} = \varnothing $, 那么由包含关系, 就一定存在 $ 0 < \delta_{0} < \min\set{\delta', \delta''} $, 使得 $ M_{\delta_{0}} = m_{\delta_{0}} = \varnothing $, 那么对该 $ \delta_{0} $, 就不存在 $ c < x < c + \delta_{0} $, 使得 $ \abs{f(x) - f(c)} \ge \varepsilon_{0} $, 这与 $ \varepsilon_{0} $ 的取法以及 $ c $ 是间断点矛盾, 不妨设$ M_{\delta} $ 恒不为空集. 下面用两种方法说明矛盾.
      \begin{method}
          \item 若存在 $ \Delta $, 使得对任意 $ c < x < c + \Delta $ 的 $ x $, 都有 $ \abs{f(x) - f(c)} \ge \varepsilon_{0} $, 那么取 $ l_{c} = f(c) + \varepsilon_{0}/2 $, 就有在 $ [c, c + \Delta] \subset [a, b] $ 上, $ f(x) = l_{c} $ 无解, 与题设矛盾.

          若对任意的 $ \delta $, 都存在 $ y $, 使得 $ c < y < c + \delta $, 但是 $ \abs{f(y) - f(c)} < \varepsilon_{0} $, 下面进行构造: 先取定一个 $ \delta_{1} $, 由上面讨论可知存在 $ c < y_{1} < x_{1} < c + \delta_{1}  $, 但是
          \begin{align*}
              f(x_{1}) \ge f(c) + \varepsilon_{0}, \quad \abs{f(y_{1}) - f(c)} < \varepsilon_{0}
          \end{align*}
          那么可知对于
          \begin{align*}
              l_{c} = f(c) + \varepsilon_{0} \in [f(y_{1}), f(x_{1})],
          \end{align*}
          存在 $ \eta_{1} \in [y_{1}, x_{1}] $, 使得 $ f(\eta_{1}) = l_{c} $. 再取 $ \delta_{2} \in (0, y_{1} - c) $, 那么又有 $ c < y_{2} < x_{2} < c + \delta_{2} $ 满足
          \begin{align*}
              f(x_{2}) \ge f(c) + \varepsilon_{0}, \quad \abs{f(y_{2}) - f(c)} < \varepsilon_{0}
          \end{align*}
          那么就又有 $ \eta_{2} \in [y_{2}, x_{2}] $, 使得 $ f(\eta_{2}) = l_{c} $, 如此构造可以得到一列互不相同的 $ \set{\eta_{n}} $, 满足 $ c < \eta_{n} < x_{1} $, 且 $ f(\eta_{n}) = l_{c} $, 这与 $ f(x) = l_{c} $ 在 $ [c, x_{1}] $ 有有限个解矛盾. 综上: $ f(x) $ 为 $ [a, b] $ 上的连续函数.
          \item 固定 $ \delta_{0} $ 设 $ y \in M_{\delta_{0}} $, 则 $ f(y) \ge f(c) + \varepsilon_{0} $, 那么对于 $ l_{c} = f(c) + \varepsilon_{0}/2 $, 由题知对于 $ f(x) = l_{c} $ 在 $ [c, y] $ 中只有有限个解, 设最小的解为 $ t > c $, 那么记 $ \delta_{1} = t - c $, 由 $ M_{\delta_{1}} $ 非空可知仍存在 $ y' \in M_{\delta_{1}} $, 满足 $ y' < c + t - c = t $, 且 $ f(y') \ge f(c) + \varepsilon_{0} $, 再由题目条件可知存在 $ t' \in [c, y'] $, 满足 $ f(t') = l_{c} $, 这与 $ t $ 是 $ [c, y] $ 中最小的解矛盾, 故 $ f(x) $ 在 $ [a, b] $ 上连续.
      \end{method}
  \end{answer}
  \item 设 $ f(x) $ 在 $ (0, +\infty) $ 上可导, 且 $ \limit{x}{+\infty} \qty(f(x) + f'(x)) = A $, 证明 $ \limit{x}{+\infty} f(x) = A $, 其中 $ A \in \R\cup{\pm\infty} $.
  \begin{hint}
      对 $ \limit{x}{+\infty} (\me^{x} f(x))/\me^{x} $ 用 L'Hospital 法则.
  \end{hint}
  \begin{answer}
      因为 $ \limit{x}{+\infty} \qty(f(x) + f'(x)) = A $ 存在, 那么由 L'Hospital 法则可知
      \begin{align*}
          \limit{x}{+\infty} f(x) = \limit{x}{+\infty} \frac{\me^{x} f(x)}{\me^{x}} = \limit{x}{+\infty} \frac{\me^{x}(f(x) + f'(x))}{\me^{x}} = \limit{x}{+\infty} (f(x) + f'(x)) = A.
      \end{align*}
  \end{answer}
  \item 已知 $ A \in \MM[\K] $, 且 $ \tr(A) = 0 $, 证明
  \begin{exercise}
      \item 存在数域 $ \K $ 上的可逆阵 $ C $, 使得 $ C^{-1}AC $ 为主对角元全为 $ 0 $ 的矩阵.
      \item 存在 $ X, Y \in \MM[\K] $, 使得 $ XY - YX = A $.
      \item 令 $ U $ 为 $ \MM[\K] $ 中所有形如 $ XY - YX $ 的矩阵组成的集合, 证明 $ U $ 是 $ \MM[\K] $ 的一个线性子空间.
  \end{exercise}
  \item 求极限
  \begin{align*}
      \limit{n}{\infty} \qty(\frac{1}{\sqrt{n^{2} + 1}} + \frac{1}{\sqrt{n^{2} + 2}} + \dots + \frac{1}{\sqrt{n^{2} + n}})^{n}
  \end{align*}
  \item 设 $ \varphi $ 为 $ n $ 维线性空间 $ V $ 上的线性变换, $ W $ 是 $ \varphi $ 的不变子空间, 且 $ V = \Image \varphi \oplus W $, 证明
  \begin{align*}
      V = \Image \varphi \oplus \Ker \varphi.
  \end{align*}
  \item 设 $ A, B \in \MM[\C] $, 且 $ \rank(A) = \rank(B)  = 1,\ \tr(A) = \tr(B) $, 证明 $ A $ 相似于 $ B $.
  \item 设 $ \varphi $ 是 $ n $ 维线性空间 $ V $ 上的线性变换, 求证: 必存在正整数 $ m $, 使得
  \begin{align*}
      \Image \varphi^{m} = \Image \varphi^{m+1},\quad \Ker \varphi^{m} = \Ker \varphi^{m+1}, \quad V = \Image \varphi^{m} \oplus \Ker \varphi^{m+1}.
  \end{align*}
  \item 使用 Jordan 标准型证明迹非 $ 0 $ 的秩 1 矩阵可对角化.
  \item 已知 $ f(x) = \frac{1 + 2x + x^{2}}{1 - x + x^{2}} $, 求 $ f^{(4)}(0) $.
  \begin{hint}
      直接求解也不是不行, 这里可以用幂级数展开的唯一性来解决.
  \end{hint}
  \begin{answer}
      首先我们知道
      \begin{align*}
          f(x) = f(0) + f'(0)x + \frac{1}{2!}f''(0)x^{2} + \frac{1}{3!}f^{(3)}(0)x^{3} + \frac{1}{4!}f^{(4)}(0)x^{4} + o(x^{4}),\quad x \to 0
      \end{align*}
      这个分解是唯一的, 那么我们只要找到某一种分解式的 $ x^{4} $ 的系数, 就可以得到 $ f^{(4)}(0) $ 的值.

      先进行化简:
      \begin{align*}
          f(x) = 1 + \frac{3x}{1-(x - x^{2})},
      \end{align*}
      利用 $ (1 - x)^{-1} = \sum_{n = 0}^{\infty} x^{n} $ 可以得到
      \begin{align*}
          f(x) = 1 + 3x (1 + (x - x^{2}) + (x - x^{2})^{2} + (x - x^{2})^{3}) + o(x^{5}), \quad x \to 0,
      \end{align*}
      计算得 $ x^{4} $ 的系数为 $ 3 \times (-2 + 1) = -3 $, 于是 $ f^{(4)}(0) = -3 \times 4! = -72 $.
  \end{answer}
  \item 设 $ A, B \in \MM[\C] $, 其中 $ A $ 是幂零阵, 且 $ AB = BA $, 求证: $ \abs{B} = \abs{A + B} $.
  \begin{hint}
      先设 $ B $ 可逆, 然后再用摄动法得到一般性的结论.
  \end{hint}
  \begin{answer}
      先设 $ B $ 可逆, 那么就有 $ \abs{A + B} = \abs{B}\abs{I + B^{-1}A} $, 也就是要证明
      \begin{align*}
          1 = \abs{I + B^{-1}A},
      \end{align*}
      由 $ AB = BA $ 可知 $ AB^{-1} = B^{-1}A $, 于是有
      \begin{align*}
          (B^{-1}A)^{n} = B^{-1}AB^{-1}A\dotsm B^{-1}A = A^{n}B^{-n} = 0,
      \end{align*}
      也即 $ B^{-1}A $ 是幂零阵, 它只有 $ 0 $ 特征值, 那么存在可逆阵 $ P $, 使得 $ P^{-1}(B^{-1}A)P = J $, 其中 $ J $ 是 $ B^{-1}A $ 的 Jordan 标准型, 对角线全为 $ 0 $, 于是有
      \begin{align*}
          \abs{I + B^{-1}A} = \abs{P^{-1}}\abs{I + B^{-1}A}\abs{P} = \abs{I + P^{-1}(B^{-1}A)P} = \abs{I + J},
      \end{align*}
      其中 $ I + J $ 是对角线全为 $ 1 $ 的上三角阵, 于是有 $ \abs{I + B^{-1}A} = 1 $, 结论成立.

      再假设 $ B $ 不可逆, 因为 $ \abs{B + tI} $ 这个关于 $ t $ 的 $ n $ 次多项式至多只有 $ n $ 个根, 设所有非零根中模长最小值为 $ t_{0} $, 那么取 $ 0 < t < t_{0} $, 就有 $ B + tI $ 可逆, 并且 $ (B + tI)A = A(B + tI) $, 所以有
      \begin{align*}
          \abs{B + tI} = \abs{A + B + tI},
      \end{align*}
      上面的等式两侧是关于 $ t $ 的连续函数, 那么令 $ t \to 0+ $, 就有 $ \abs{B} = \abs{A + B} $.
  \end{answer}
  \item 设函数 $ f $ 在 $ x = 0 $ 连续, 并且
  \begin{align*}
      \limit{x}{0}\frac{f(2x) - f(x)}{x} = A,
  \end{align*}
  求证: $ f'(0) $ 存在, 且 $ f'(0) = A $.
  \begin{hint}
      利用极限式构造一个趋于 $ 0 $ 的自变量, 再利用 $ x = 0 $ 处的连续性及导数定义即可得到结论.
  \end{hint}
  \begin{answer}
      由
      \begin{align*}
          \limit{x}{0}\frac{f(2x) - f(x)}{x} = A,
      \end{align*}
      可知: 对任意 $ \varepsilon > 0 $, 存在 $ \delta > 0 $, 使得当 $ \abs{x} < \delta $ 时,
      \begin{align*}
          A - \varepsilon < \frac{f(2x) - f(x)}{x} < A + \varepsilon
      \end{align*}
      又因为 $ \abs{2^{-1}x} < \delta $, 那么又有
      \begin{align*}
          A - \varepsilon < \frac{f(x) - f(2^{-1}x)}{2^{-1}x} < A + \varepsilon
      \end{align*}
      即
      \begin{align*}
          2^{-1}(A - \varepsilon) < \frac{f(x) - f(2^{-1}x)}{x} < 2^{-1}(A + \varepsilon),
      \end{align*}
      再取 $ 2^{-2}x, 2^{-3}x, \dots, 2^{-n}x $, 就可以得到一列不等式:
      \begin{align}
          2^{-2}(A - \varepsilon) & < \frac{f(2^{-1}x) - f(2^{-2}x)}{x} < 2^{-2}(A + \varepsilon) \notag\\
          2^{-3}(A - \varepsilon) & < \frac{f(2^{-2}x) - f(2^{-3}x)}{x} < 2^{-3}(A + \varepsilon) \notag\\
                                            & \phantom{< f(2^{-2}x) -}\cdots \notag\\
          2^{-n}(A - \varepsilon) & < \frac{f(2^{-n + 1}x) - f(2^{-n}x)}{x} < 2^{-n}(A + \varepsilon) \notag\\
      \intertext{从 $ 2^{-1}x $ 开始将它们相加, 即可得到}
          (1 - 2^{-n})(A - \varepsilon) & < \frac{f(x) - f(2^{-n}x)}{x} < (1 - 2^{-n})(A + \varepsilon)\label{eq:f2x-fx/x:求和}
      \end{align}
      由于 $ f(x) $ 在 $ x = 0 $ 连续, 则 $ \limit{n}{\infty}f(2^{-n}x) = f(0) $, 那么对不等式 \eqref{eq:f2x-fx/x:求和} 中的 $ n $ 取极限, 可得
      \begin{align*}
          A - \varepsilon \le \frac{f(x) - f(0)}{x - 0} \le A + \varepsilon
      \end{align*}
      这就说明了 $ f'(0) $ 存在, 且 $ f'(0) = A $.
  \end{answer}
  \item 设 $ x_{n} $ 是 $ \tan x = x $ 在 $ (n\pi, n\pi + \pi/2) $ 上的解,
  \begin{exercise}
      \item 求证 $ \limit{n}{\infty}(n\pi + \pi/2 - x_{n}) = 0 $,
      \item 求 $ \limit{n}{\infty} n(n\pi + \pi/2 - x_{n}) $.
  \end{exercise}
  \begin{hint}
      \begin{hintsheet}
          \item 首先可以确定 $ n\pi + \pi/2 - x_{n} \in (0, \pi/2) $, 那么只要说明 $ \limit{n}{\infty} \tan(n\pi + \pi/2 - x_{n}) = 0 $ 就可以说明原极限为 $ 0 $.
          \item 已经知道了上问的极限为 $ 0 $, 那么利用 $ t \sim \tan t\,(t \to 0) $ 就可以得到答案.
      \end{hintsheet}
  \end{hint}
  \begin{answer}
      \begin{answersheet}
          \item 首先可以确定 $ x_{n} \in (n\pi, n\pi + \pi/2) $, 于是当 $ n \to \infty $ 时, $ x_{n} \to +\infty $. 注意到 $ \tan x $ 的周期性, 以及 $ \tan x_{n} = x_{n} $
          \begin{align*}
              \tan\qty(n \pi + \frac{\pi}{2} - x_{n}) = \tan\qty(\frac{\pi}{2} - x_{n}) = \frac{1}{\tan x_{n}} = \frac{1}{x_{n}}
          \end{align*}
          注意到 $ \limit{n}{\infty}x_{n} = +\infty $, 于是就有
          \begin{align*}
              \limit{n}{\infty}\tan\qty(n \pi + \frac{\pi}{2} - x_{n}) = \limit{n}{\infty}\frac{1}{x_{n}} = 0,
          \end{align*}
          又因为对于 $ y \in (0, \pi/2) $, $ y = \arctan(\tan y) $, 且 $ n\pi + \pi/2 - x_{n} \in (0, \pi/2) $, 于是
          \begin{align*}
              \limit{n}{\infty}\qty(n \pi + \frac{\pi}{2} - x_{n}) = \limit{n}{\infty} \arctan\tan\qty(n \pi + \frac{\pi}{2} - x_{n}) = \arctan \limit{n}{\infty} \tan\qty(n \pi + \frac{\pi}{2} - x_{n}) = \arctan 0 = 0.
          \end{align*}
          \item 由上问可知 $ \limit{n}{\infty}(n\pi + \pi/2 - x_{n}) = 0 $, 于是就有
          \begin{align*}
              \limit{n}{\infty} n\qty(n\pi + \frac{\pi}{2} - x_{n}) & = \limit{n}{\infty} n \frac{n\pi + \pi/2 - x_{n}}{\tan(n\pi + \pi/2 - x_{n})}\tan\qty(n\pi + \frac{\pi}{2} - x_{n})\\
              & = \limit{n}{\infty} n \tan\qty(n\pi + \frac{\pi}{2} - x_{n}) = \limit{n}{\infty} \frac{n}{\tan x_{n}} \\
              & = \limit{n}{\infty} \frac{n}{x_{n}} = \limit{n}{\infty} \frac{n}{n \pi + \pi/2} \frac{n \pi + \pi/2}{x_{n}},
          \end{align*}
          又因为
          \begin{align*}
              \limit{n}{\infty} \frac{n\pi + \pi/2 - x_{n}}{x_{n}} = \limit{n}{\infty} \frac{n \pi + \pi/2}{x_{n}} - 1 = 0 \implies \limit{n}{\infty} \frac{n\pi + \pi/2}{x_{n}} = 1,
          \end{align*}
          于是
          \begin{align*}
              \limit{n}{\infty} n\qty(n\pi + \frac{\pi}{2} - x_{n}) = \limit{n}{\infty} \frac{n}{n \pi + \pi/2} \frac{n \pi + \pi/2}{x_{n}} = \frac{1}{\pi}.
          \end{align*}
      \end{answersheet}
  \end{answer}
  \item 设 $ f $ 在 $ [0, +\infty) $ 上可微, 且 $ f(0) = 0 $, 并假设有实数 $ A $ 使得 $ \abs{f'(x)} \le A\abs{f(x)} $ 对 $ x \in (0, +\infty) $ 恒成立, 证明 $ f(x) \equiv 0\,(x \in [0, +\infty)) $.
  \begin{hint}
      将 $ [0, +\infty) $ 分成很多区间长度是 $ 1/(2A) $ 的小区间, 取每个小区间上的最大值来用题目条件说明矛盾, 再分别证明每个小区间上的函数值为 $ 0 $ 即可.
  \end{hint}
  \begin{answer}
      我们将 $ [0, +\infty) $ 分为无数个长度是 $ 1/(2A) $ 的小区间的并:
      \begin{align*}
          [0, +\infty) = \bigcup_{n = 0}^{\infty}\qty[\frac{n}{2A}, \frac{n + 1}{2A}] := \bigcup_{n = 0}^{\infty}\Delta_{n},
      \end{align*}
      然后对 $ n $ 用数学归纳法说明每个小区间上的函数值都为 $ 0 $. 当 $ n = 0 $ 时, 由于 $ \abs{f(x)} $ 在 $ \Delta_{0} $ 上连续, 那么有最大值, 设为 $ \abs{f(x_{0})} $, 于是由 Lagrange 中值定理, 就存在 $ \xi_{0} \in [0, x_{0}] $, 使得 $ f(x_{0}) - f(0) = f'(\xi_{0})x_{0} $, 那么由题目条件可知
      \begin{align*}
          \abs{f(x_{0})} = \abs{f'(\xi_{0})}x_{0} \le A\abs{f(\xi_{0})}\frac{1}{2A} = \frac{1}{2}\abs{f(\xi_{0})} \le \frac{1}{2}\abs{f(x_{0})}
      \end{align*}
      这就说明 $ \abs{f(x_{0})} = 0 $, 也就是 $ f(x) = 0 $ 在 $ \Delta_{0} $ 上成立. 归纳假设 $ f(x) = 0 $ 在 $ \Delta_{n - 1} $ 上成立, 那么对于 $ \Delta_{n} $, 同样有最大值 $ \abs{f(x_{n})} $, 由 Lagrange 中值定理可知存在 $ \xi_{n} \in [n/(2A), x_{n}] $, 使得
      \begin{align*}
          f(x_{n}) - f\qty(\frac{n}{2A}) = f'(\xi_{n})\qty(x_{n} - \frac{n}{2A}),
      \end{align*}
      注意到 $ f(n/(2A)) = 0 $, $ x_{n} - n/(2A) < 1/(2A) $, 于是和上面一样的讨论可知
      \begin{align*}
          \abs{f(x_{n})} = \abs{f'(\xi_{n})}\qty(x_{n} - \frac{n}{2A}) \le \frac{1}{2}\abs{f(x_{n})}
      \end{align*}
      即 $ f(x) = 0 $ 在 $ \Delta_{n} $ 上也成立. 于是由数学归纳法可知 $ f(x) = 0 $ 在 $ [0, +\infty) $ 上恒成立.
  \end{answer}
  \item 设偶函数 $ f(x) $ 在 $ x = 0 $ 处二阶连续可导, 且 $ f(0) = 1 $, 证明级数 $ \sum_{n = 1}^{\infty} (f(1/n) - 1) $ 绝对收敛.
  \begin{hint}
      注意到 $ f(x) $ 为 $ x = 0 $ 处可导的偶函数, 那么 $ f'(0) = 0 $, 再利用比较判别法即可得到结论.
  \end{hint}
  \begin{answer}
      首先我们证明 $ f'(0) = 0 $, 由 $ f(x) $ 为偶函数可得
      \begin{align*}
          f'(0) = \begin{dcases}
              \limit{x}{0}\frac{f(x) - f(0)}{x} = \limit{x}{0}\frac{f(-x) - f(0)}{x} \\
              \limit{x}{0} \frac{f(-x) - f(0)}{-x} = -\limit{x}{0}\frac{f(-x) - f(0)}{x}
          \end{dcases}
      \end{align*}
      对比 $ f'(0) $ 的两种表示可知 $ f'(0) = 0 $, 再由 $ f(x) $ 在 $ x = 0 $ 处二阶连续可导, 那么在 $ x $ 的某个邻域 $ [x, x+1/n] $ 上二阶可导. 于是由带 Lagrange 余项的 Taylor 定理知存在 $ \xi \in (0, 1/n) $ 使得
      \begin{align*}
          f\qty(\frac{1}{n}) = f(0) + \frac{1}{n}f'(0) + \frac{1}{2n^{2}}f''(\xi) = 1 + \frac{1}{2n^{2}}f''(\xi),
      \end{align*}
      那么 $ \abs{f(1/n) - 1} = \order{n^{-2}}\,(n \to \infty) $, 那么由比较判别法的极限形式可以知道 $ \sum_{n = 1}^{\infty} (f(1/n) - 1) $ 绝对收敛.
  \end{answer}
  \item 设 $ f $ 在 $ [a, b] $ 上可导, 且 $ f' $ 在 $ [a, b] $ 上可积, $ f(a) = 0 $, 证明:
  \begin{align*}
      2\int_{a}^{b}(f(x))^{2} \dx* \le (b-a)^{2} \int_{a}^{b} (f'(x))^{2} \dx*.
  \end{align*}
  \item 设 $ f(x) $ 在 $ [0, +\infty) $ 上可微, 且存在实数 $ A > 0 $, 使得 $ \abs{f'(x)} \le A\abs{f(x)} $, 证明 $ f(x) \equiv 0 $ 对 $ x \in [0, +\infty) $ 均成立.
  \item 设 $ f(x) $ 在 $ [0, 1] $ 上有连续的二阶导数, $ f(0) = f(1) = 0 $, 且对任意的 $ x \in (0, 1) $, 都有 $ f(x) \ne 0 $, 证明
  \begin{align*}
      \int_{0}^{1} \abs{f''(x)\over f(x)} \dx* \ge 4
  \end{align*}
  \item 设 $ f(x) \in C^{2}[a, b] $, 证明: 存在 $ \xi \in (a, b) $ 使得
  \begin{align*}
      \int_{a}^{b} f(x) \dx* = (b - a) f\qty(\frac{a + b}{2}) + \frac{1}{24}(b - a)^{3} f''(\xi).
  \end{align*}
  \item 设 $ x_{1} $, $ x_{2} $, $ x_{3} $ 是多项式 $ f(x) = x^{3} + ax + 1 $ 的三个根, 求一个首一多项式以 $ x_{1}^{2} $, $ x_{2}^{2} $, $ x_{3}^{2} $ 为根.
  \item 设 $ f(x) $ 在有限区间 $ (a, b) $ 内可微, 且 $ f'(x) $ 在 $ (a, b) $ 内有界, 证明 $ f(x) $ 在 $ (a, b) $ 内有界.
  \item 设 $ f(x) $ 在 $ [a, b] $ 上连续, 且对任意 $ x_{1}, x_{2} \in [a, b] $, $ \lambda \in (0, 1) $, 恒有 $ f(\lambda x_{1} + (1-\lambda) x_{2}) \le \lambda f(x_{1}) + (1 - \lambda) f(x_{2}) $. 证明
  \begin{align*}
      f\qty(\frac{a+b}{2}) \le \frac{1}{b - a}\int_{a}^{b} f(x) \dx* \le \frac{f(a) + f(b)}{2}.
  \end{align*}
  \item 设 $ \limit{x}{+\infty} f(x) = A \in \R $, 且满足下列条件之一, 则有 $ \limit{x}{+\infty} f'(x) = 0 $.
  \begin{exercise}
      \item $ f''(x) $ 在 $ (0, +\infty) $ 有界;
      \item $ \limit{x}{+\infty} f'(x) $ 存在.
  \end{exercise}
  \begin{hint}
      \begin{hintsheet}
          \item 对任意足够大的 $ x $ 将 $ f(x+\sqrt{\varepsilon}) $ 在 $ x $ 处 Taylor 展开, 就可以得到 $ f'(x) $ 的表达式, 再利用 $ f''(x) $ 有界, 以及 $ f(x) \to A $, 即可说明.
          \item 如果 $ f'(x) $ 极限不为 $ 0 $, 可以推出 $ f(x) $ 无界.
      \end{hintsheet}
  \end{hint}
  \begin{answer}
      \begin{answersheet}
          \item 首先存在 $ M > 0 $ 使得 $ \abs{f''(x)} \le M $. 由 Cauchy 收敛准则可知对任意 $ 0 < \varepsilon < 1 $, 存在 $ G > 0 $, 使得当 $ x, y > G $ 时, $ \abs{f(x) - f(y)} < \varepsilon $, 于是我们在 $ x $ 处展开 $ f(x + \sqrt{\varepsilon}) $, 其中 $ x > G $.
          \begin{align*}
              f\qty(x + \sqrt{\varepsilon}) = f(x) + \sqrt{\varepsilon}f'(x) + \frac{\varepsilon}{2}f''(\xi),\quad \xi \in \qty(x, x + \sqrt{\varepsilon}),
          \end{align*}
          移项可得
          \begin{align*}
              \abs{f'(x)} & = \abs{\frac{f\qty(x + \sqrt{\varepsilon}) - f(x)}{\sqrt{\varepsilon}} - \frac{f''(\xi)\sqrt{\varepsilon}}{2}}\\
              & \le \sqrt{\varepsilon} + \frac{M\sqrt{\varepsilon}}{2} = \qty(1 + \frac{M}{2})\sqrt{\varepsilon}.
          \end{align*}
          \item 用反证法, 如果 $ \limit{x}{+\infty} f'(x) = B > 0 $, $ B < 0 $ 的时候同理. 那么存在 $ M > 0 $, 使得当 $ x \ge M $ 时 $ f'(x) \ge B/2 $, 于是对于 $ x > M $, 有
          \begin{align*}
              f(x) = f(M) + f'(\xi)(x - M) > f(M) + \frac{B(x - M)}{2} \to +\infty,
          \end{align*}
          这与 $ \limit{x}{+\infty}f(x) $ 存在矛盾.
      \end{answersheet}
  \end{answer}
  \item 广义积分 $ \int_{a}^{+\infty} f(x) \dx* $ 收敛, 加上下面任一条件即可推出 $ \limit{x}{+\infty} f(x) = 0 $:
  \begin{exercise}
      \item $ \limit{x}{+\infty} f(x) $ 存在,
      \item $ \int_{a}^{+\infty} f'(x) \dx* $ 收敛,
      \item $ f(x) $ 单调, 这时有更强的结果: $ \limit{x}{+\infty} xf(x) = 0 $,
      \item $ f(x) $ 在 $ [a, +\infty) $ 上一致连续,
      \item $ f'(x) $ 在 $ [a, +\infty) $ 上有界.
  \end{exercise}
  \item 设 $ A $ 是三阶正交矩阵, 且 $ \abs{A} = 1 $, 证明存在正交阵 $ B $, 使得 $ A = B^{2} $.
  \item 设函数 $ f(x) $ 在 $ \R $ 上有定义, 且在任何有限闭区间上可积, 证明: 对任何闭区间 $ [a, b] $, 有
  \begin{align*}
      \limit{h}{0} \int_{a}^{b} \abs{f(x + h) - f(x)} \dx* = 0.
  \end{align*}
  \item 设数列 $ \set{x_{n}} $ 满足 $ \set{2x_{n+1} + x_{n}} $ 收敛, 证明数列 $ \set{x_{n}} $ 收敛.
  \item (\emph{Young 不等式})设 $ y=f(x) $ 是区间 $ [0, +\infty) $ 上严格递增的连续函数, 且满足 $ f(0) = 0 $, 证明对任意的 $ a, b > 0 $, 有
  \begin{align*}
      ab \le \int_{0}^{a}f(x) \dx* + \int_{0}^{b} f^{-1}(y) \dd{y}.
  \end{align*}
  \item 设 $ f, g \in C[a, b] $, $ g $ 在 $ [a, b] $ 上不变号, 证明存在 $ \xi\in (a, b) $, 使得
  \begin{align*}
      \int_{a}^{b} f(x)g(x) \dx* = f(\xi)\int_{a}^{b} g(x) \dx*.
  \end{align*}
  \item 设 $ A $ 为 $ 3 $ 阶非零实矩阵, $ A^{T} = A^{*} $, 且 $ \abs{I + A} = \abs{I - A} = 0 $, 计算行列式 $ \abs{A^{2} - A - 3I} $.
  \item 设 $ f(x) $ 在 $ [a, b] $ 上单调, $ g(x) $ 是 $ \R $ 上以 $ T>0 $ 为周期的连续函数,且 $ \int_{0}^{T} g(x) \dx* = 0 $, 求
  \begin{align*}
      \limit{\lambda}{\infty} \int_{a}^{b} f(x)g(\lambda x) \dx*
  \end{align*}
  \sitem 设 $ f(x) $ 是实多项式, 且对任意实数 $ r $, 都有 $ f(r) \ge 0 $. 证明存在实多项式 $ g(x), h(x) $ 使得 $ f(x) = g^{2}(x) + h^{2}(x) $, 更进一步可以要求 $ \deg g(x) > \deg h(x) $.
  \begin{hint}
      $ f(r) \ge 0 $ 说明 $ f(x) $ 的实根均成对, 也就是只需要处理形如 $ x^{2} + bx + c $ 的不可约多项式即可.
  \end{hint}
  \begin{answer}
      \begin{answersheet}
          \item 首先可以说明 $ f(x) $ 的实根的重数都是偶数, 且首项系数 $ k > 0 $ 即
          \begin{align}
              f(x) = k\prod_{i = 1}^{s}(x - a_{i})^{2k_{i}}\prod_{j = 1}^{t}(x^{2} + b_{j}x + c_{j})^{l_{j}} := kp^{2}(x)q(x), \label{eq:f=kp2q}
          \end{align}
          其中 $ k, a_{i}, b_{j}, c_{j} \in \R,\ k_{i}, l_{j} \in \N \,(1 \le i \le s, 1 \le j \le t) $,  $ b_{j}^{2} - 4c_{j} < 0 $, 又可知对于每个 $ j $,  $ x^{2} + b_{j}x + c_{j} $ 都有一对共轭的复根, 记为 $ \omega_{j}, \bar{\omega}_{j} $, 并注意到 $ x \in \R $ 于是又有
          \begin{align*}
              q(x) = \prod_{j = 1}^{t} ( x - \omega_{j})^{l_{j}}\prod_{j = 1}^{t} ( x - \bar{\omega}_{j})^{l_{j}} = \prod_{j = 1}^{t} ( x - \omega_{j})^{l_{j}}\overline{\prod_{j = 1}^{t} ( x - \omega_{j})^{l_{j}}}
          \end{align*}
          若记 $ \prod_{j = 1}^{t} ( x - \omega_{j})^{l_{j}} = g_{i}(x) + \mi h_{1}(x) $, 其中 $ g_{1}(x), h_{1}(x) \in \R[x] $, 那么就有
          \begin{align*}
              q(x) = (g_{1}(x) + \mi h_{1}(x))\overline{(g_{1}(x) + \mi h_{1}(x))} = (g_{1}(x) + \mi h_{1}(x))(g_{1}(x) - \mi h_{1}(x)) = g_{1}^{2}(x) + h_{1}^{2}(x)
          \end{align*}
          于是令 $ g(x) = \sqrt{k}p(x)g_{1}(x) $, $ h(x) = \sqrt{k}p(x)h_{1}(x) $, 就有 $ f(x) = g^{2}(x) + h^{2}(x) $.
          \item 注意到在刚才的过程中得到的 $ g(x) $ 与 $ h(x) $ 的次数是相等的, 如果要得到次数不等的 $ g, h $, 可以用数学归纳法. 首先依旧可以得到 \eqref{eq:f=kp2q} 的分解式, 当 $ b^{2} - 4c < 0 $ 时利用
          \begin{align*}
              x^{2} + b_{1}x + c_{1} = \qty(x + \frac{b_{1}}{2})^{2} + \frac{4c_{1} - b_{1}^{2}}{4} = \varphi_{1}^{2}(x) + \psi_{1}^{2}(x)
          \end{align*}
          以及
          \begin{align*}
              (a^{2} + b^{2})(c^{2} + d^{2}) = \frac{1}{2}(ac + bd)^{2} + \frac{1}{2}(ad - bc)^{2}
          \end{align*}
          可以得到 $ q(x) $ 新的分解式 $ g_{2}(x), h_{2}(x) $, 此时 $ \deg g_{2}(x) > \deg h_{2}(x) $, 且令 $ g(x) = \sqrt{k}p(x)g_{2}(x) $, $ h(x) = \sqrt{k}p(x)h_{2}(x) $, 就有 $ f(x) = g^{2}(x) + h^{2}(x) $, 这时就有 $ \deg g(x) > \deg h(x) $.
      \end{answersheet}
  \end{answer}
  \item 设 $ \varphi_{1}, \varphi_{2}, \dots, \varphi_{m} $ 是 $ n $ 维线性空间 $ V $ 上的线性变换, 且适合条件:
  \begin{align*}
      \varphi_{i}^{2} = \varphi_{i}, \quad \varphi_{i}\varphi_{j} = 0\,(i\ne j), \quad \bigcap_{i=1}^{m}\Ker\varphi_{i} = 0,
  \end{align*}
  求证: $ V = \bigoplus_{i=1}^{m}\Image\varphi_{i} $.
  \item 设 $ f $ 在 $ (0, 1] $ 上可导, 且 $ \limit{x}{0+} \sqrt{x} f'(x) = A \in \R $, 证明 $ f $ 在 $ (0, 1] $ 上一致连续.
  \item 设 $ f(x) $ 在 $ [0, 1] $ 可积, $ f(1) = 0 $, $ f'(1) = a $, 证明
  \begin{align*}
      \limit{n}{\infty}\int_{0}^{1}n^{2}x^{n}f(x)\dx* = -a.
  \end{align*}
  \item 设 $ f(x), g(x) $ 是次数不小于 $ 1 $ 的互素多项式, 求证, 必唯一地存在两个多项式 $ u(x), v(x) $ 使得
  \begin{align*}
      f(x)u(x) + g(x)v(x) = 1,
  \end{align*}
  且 $ \deg v(x) < \deg f(x) $, $ \deg u(x) < \deg g(x) $.
  \item 设 $ f(x) $ 是次数大于 $ 0 $ 的首一整系数多项式, 若 $ f(0), f(1) $ 都是奇数, 求证 $ f(x) $ 没有有理根.
  \begin{hint}
      首先可以知道如果 $ f(x) $ 有有理根, 那这个有理根必是整数根, 再考虑该根的奇偶性.
  \end{hint}
  \begin{answer}
      设
      \begin{align*}
          f(x) = x^{n} + a_{n - 1}x^{n - 1} + \dots + a_{1}x + a_{0},
      \end{align*}
      那么可以知道, 如果 $ p/q $ 为 $ f(x) $ 的有理根, 其中 $ p, q $ 为互素的整数, 那么应该有 $ p \mid a_{0} $, $ q \mid 1 $, 也就是说 $ p/q $ 只能为整数 $ p $ , 那么只要说明 $ f(x) $ 没有整数根就可以了.

      因为 $ f(0) $ 为奇数, 所以 $ a_{0} $ 为奇数, 用系数的奇偶来重写 $ f(x) $:
      \begin{align*}
          f(x) = \sum_{i = 1}^{s}a_{r_{i}}x^{r_{i}} + \sum_{j = 1}^{\mathclap{n + 1 - s}}a_{t_{j}}x^{t_{j}} := g(x) + h(x),
      \end{align*}
      其中 $ a_{r_{i}} \equiv 1 \pmod{2} $, $ a_{t_{j}} \equiv 0 \pmod{2} $ 因为 $ f(1) $ 为奇数, 那么就有 $ s $ 为奇数, 于是若 $ p $ 为奇数, 则 $ g(p) \equiv 1 \pmod{2} $, 而 $ h(p) \equiv 0 \pmod{2} $, 这说明 $ f(p) \equiv 1 \pmod{2} $, 这说明奇数不能作为 $ f(x) = 0 $ 的解. 若 $ p $ 为偶数, 那么
      \begin{align*}
          x^{n} + a_{n - 1}x^{n - 1} + \dots + a_{1}x \equiv 0 \pmod{2},
      \end{align*}
      而 $ a_{0} \equiv 1 \pmod{2} $, 同样说明 $ f(p) \equiv 1 \pmod{2} $, 那么偶数也不能作为 $ f(x) = 0 $ 的解. 这就说明了 $ f(x) $ 不存在整数解.
  \end{answer}
  \item 设 $ f(x) $ 是次数大于 $ 1 $ 的奇数次有理系数多项式, 且它在有理数域上不可约, 求证: 若 $ x_{1}, x_{2} $ 是 $ f(x) $ 在复数域上的两个不同的根, 则 $ x_{1} + x_{2} $ 必不是有理数.
  \item 设 $ A $ 是实矩阵, 又 $ I_{n} - A $ 的特征值的模长都小于 $ 1 $, 求证: $ 0<\abs{A}<2^{n} $.
  \item 设
  \begin{align*}
      \Q(\sqrt[n]{2}) = \set{a_{0} + a_{1}\sqrt[n]{2} + a_{2}\sqrt[n]{4} + \dots + a_{n-1}\sqrt[n]{2^{n-1}} : a_{i} \in \Q, 0\le i\le n-1}
  \end{align*}
  证明 $ \Q(\sqrt[n]{2}) $ 是一个数域, 并求 $ \Q(\sqrt[n]{2}) $ 做为 $ \Q $ 上线性空间的一组基.
  \item 设 $ f(x) = x^{n} + a_{1}x^{n-1} + \dots + a_{n-1}x + a_{n} $ 是数域 $ \K $ 上的不可约多项式, $ \varphi $ 是 $ \K $ 上的 $ n $ 维线性空间 $ V $ 上的线性变换, $ \alpha_{1} \ne 0, \alpha_{2}, \dots, \alpha_{n} $, 是 $ V $ 中的向量, 满足
  \begin{align*}
      \varphi(\alpha_{i}) = \alpha_{i + 1}\,(i = 1, 2, \dots, n-1), \quad \varphi(\alpha_{n}) = -a_{n}\alpha_{1} - a_{n-1}\alpha_{2} - \dots - a_{1}\alpha_{n}.
  \end{align*}
  证明 $ \set{\alpha_{1}, \alpha_{2}, \dots, \alpha_{n}} $ 是 $ V $ 的一组基.
  \item\label{item:CtoR} 设 $ A, B \in \MM $, 存在可逆复矩阵 $ P $, 使得 $ P^{-1}AP = B $, 证明存在可逆实矩阵 $ Q $ 使得 $ Q^{-1}AQ = B $.
  \begin{hint}
      设 $ P = M + \mi N $, 那么就有 $ AM = MB, AN = NB $, 又知存在 $ t \in \R $ 使得 $ M + tN $ 可逆, 即可得出结论.
  \end{hint}
  \begin{answer}
      设 $ P = M + \mi N $, 其中 $ M, N \in \MM $, 那么在 $ A(M + \mi N) = (M + \mi N)B $ 中分别对应实部和虚部, 即可得到 $ AM = MB,\ AN = NB $. 而 $ \abs{M + tN} = 0 $ 这个关于 $ t $ 的 $ n $ 次多项式至多只有 $ n $ 个解, 那么一定存在 $ t_{0} \in \R $ 使得 $ \abs{M + t_{0}N} \ne 0 $, 于是令 $ Q = M + t_{0}N $, 就有
      \begin{align*}
          AQ = AM + t_{0}AN = MB + t_{0}NB = (M + t_{0}N)B = QB \implies Q^{-1}AQ = B.
      \end{align*}
  \end{answer}
  \item 设 $ A, B $ 为 $ n $ 阶方阵, 满足 $ \rank(ABA) = \rank(A) $, 求证: $ AB $ 与 $ BA $ 相似.
  \item 设 $ A, B $ 为 $ n $ 阶方阵, 则 $ AB $ 与 $ BA $ 相似的充要条件是 $ \rank((AB)^{i}) = \rank((BA)^{i})\,(1 \le i \le n - 1) $.
  \item 设 $ f $ 在 $ \R $ 上连续, 又 $ \varphi(x) = f(x)\int_{0}^{x} f(t) \dd{t} $ 单调递减, 证明 $ f(x) \equiv 0, x\in\R $.
  \item 计算积分
  \begin{align*}
      I = \int_{0}^{\pi/2} \frac{\sin x}{\sin x + \cos x} \dx*.
  \end{align*}
  \begin{hint}
      考虑以下两个积分的关系:
      \begin{align*}
          I = \int_{0}^{\pi/2} \frac{\sin x}{\sin x + \cos x} \dx*, \quad J = \int_{0}^{\pi/2} \frac{\cos x}{\sin x + \cos x} \dx*
      \end{align*}
  \end{hint}
  \begin{answer}
      设
      \begin{align*}
          I = \int_{0}^{\pi/2} \frac{\sin x}{\sin x + \cos x} \dx*, \quad J = \int_{0}^{\pi/2} \frac{\cos x}{\sin x + \cos x} \dx*
      \end{align*}
      那么首先可以得到
      \begin{align*}
          I + J = \int_{0}^{\pi/2} 1\dx* = \frac{\pi}{2},
      \end{align*}
      再对积分 $ I $ 用换元 $ t = \pi/2 - x $, 那么就有
      \begin{align*}
          I = \int_{0}^{\pi/2} \frac{\sin(\pi/2 - t)}{\sin(\pi/2 - t) + \cos(\pi/2 - t)} \dd{\qty(\frac{\pi}{2} - t)} = \int_{0}^{\pi/2} \frac{\cos t}{\cos t + \sin t} \dd{t} = J,
      \end{align*}
      于是 $ I = J = \pi/4 $.
  \end{answer}
  \item 讨论广义积分
  \begin{align*}
      \int_{1}^{+\infty} \frac{\sin x}{x^{p} + \sin x} \dx*
  \end{align*}
  在何时绝对收敛或条件收敛.
  \item 设 $ f(x) $ 在 $ [a, +\infty) $ 上一阶连续可导, 且 $ x \to +\infty $ 时, $ f(x) $ 单调递减趋于 $ 0 $, 证明无穷积分 $ \int_{a}^{+\infty} f(x) \dx* $ 收敛当且仅当 $ \int_{a}^{+\infty} xf'(x) \dx* $ 收敛.
  \item 设 $ \set{a_{n}} $ 是正数列, $ \liminf\limits_{n\to\infty} a_{n} = 1 $, $ \limsup\limits_{n\to\infty} a_{n} = A < +\infty $, 且 $ \limit{n}{\infty} \sqrt[n]{a_{1}a_{2}\dots a_{n}} = 1 $, 求证:
  \begin{align*}
      \limit{n}{\infty}\frac{a_{1} + a_{2} + \dots + a_{n}}{n} = 1.
  \end{align*}
  \item 设 $ \limit{n}{\infty} a_{n} = A $, 求 $ \limit{n}{\infty} \sum_{k=1}^{n} \frac{a_{n+k}}{n+k} $.
  \begin{hint}
      首先注意到 $ \limit{n}{\infty}\sum_{k = 1}^{n} A(n + k)^{-1} = A\ln 2 $, 再利用拟合法, 即考虑 $ \sum_{k = 1}^{n} (a_{n + k} - A)(n + k)^{-1} $ 即可.
  \end{hint}
  \begin{answer}
      一方面, 我们可以用定积分的定义求出
      \begin{align*}
          \limit{n}{\infty}\sum_{k = 1}^{n}\frac{A}{n + k} = A \limit{n}{\infty}\sum_{k = 1}^{n} \frac{1}{n + k} = A \limit{n}{\infty}\frac{1}{n} \sum_{k = 1}^{n}\frac{1}{1 + k/n} = A\int_{0}^{1} \frac{\dx}{1 + x} = A\ln 2.
      \end{align*}

      然后我们来证明
      \begin{align*}
          \limit{n}{\infty}\sum_{k = 1}^{n} \frac{a_{n + k} - A}{n + k} = 0.
      \end{align*}
      由于 $ \limit{n}{\infty}a_{n} = A $, 于是对于任意的 $ \varepsilon > 0 $, 存在 $ N \in \N $, 使得对于任意 $ k \in \N $, 都有 $ \abs{a_{n + k} - A} < \varepsilon $, 于是对于 $ n > N $
      \begin{align*}
          \abs{\sum_{k = 1}^{n} \frac{a_{n + k} - A}{n + k}} \le \sum_{k = 1}^{n} \frac{\abs{a_{n + k} - A}}{n + k} \le \varepsilon \sum_{k = 1}^{n} \frac{1}{n + k} \le \varepsilon\sum_{k = 1}^{n} \frac{1}{n} \le \varepsilon
      \end{align*}
      也就是说明
      \begin{align*}
          \limit{n}{\infty} \sum_{k = 1}^{n}\frac{a_{n + k}}{n + k} = \limit{n}{\infty} \sum_{k = 1}^{n}\frac{A}{n + k} = A\ln 2.
      \end{align*}
  \end{answer}
  \item 设 $ f(x) \in C[1, +\infty) $, 对任意 $ x\in [1, +\infty) $, 有 $ f(x) > 0 $, 且 $ \limit{x}{+\infty} \ln(f(x))/\ln(x) = -\lambda $, 证明: $ \lambda > 1 $ 时 $ \int_{1}^{+\infty} f(x) \dx* $ 收敛.\footnote{数列形式下的该判别法称为\emph{对数判别法}}
  \item 设函数 $ f(x) $ 在 $ [0, 1] $ 上单调, 并且积分 $ \int_{0}^{1} f(x) \dx* $ 收敛, 证明:
  \begin{align*}
      \int_{0}^{1} f(x) \dx* = \limit{n}{\infty} \frac{1}{n} \sum_{k=1}^{n} f\qty(\frac{k}{n}).
  \end{align*}
  并且举反例说明``去掉单调条件, 结论则不成立.''
  \item 设 $ V $ 是 $ n $ 维线性空间, 对于整数 $ k \ge n $, 证明存在一组向量 $ \alpha_{1}, \alpha_{2}, \dots, \alpha_{k} \in V $, 使得其中任意 $ n $ 个线性无关.
  \item 设 $ \set{a_{n}} $ 是递减正数列, 证明: $ \sum_{n=1}^{\infty}a_{n} $ 与 $ \sum_{n=1}^{\infty}2^{n}a_{2^{n}} $ 同时敛散.
  \item 设对任意 $ n\in \N $, $ a_{n} > 0 $, 且级数 $ \sum_{n=1}^{\infty}1/a_{n} $ 收敛, 证明下述级数收敛 (利用绝对收敛函数重排不改变敛散性与级数值):
  \begin{align*}
      \sum_{n=1}^{\infty}\frac{n}{a_{1} + a_{2} + \dots + a_{n}}.
  \end{align*}
  \item 判断级数 $ \sum_{n=1}^{\infty} (-1)^{[\sqrt{n}]}/n^{p} $ 的敛散性.
  \item 设 $ f(x) $ 在 $ [-1, 1] $ 上二次连续可微, 且有 $ \limit{x}{0} f(x)/x = 0 $, 证明级数 $ \sum_{n=1}^{\infty}f(1/n) $ 绝对收敛.
  \item 已知 $ \sum_{n=1}^{\infty}(a_{n} - a_{n-1}) $ 绝对收敛, $ \sum_{n=1}^{\infty}b_{n} $ 收敛, 证明 $ \sum_{n=1}^{\infty}a_{n}b_{n} $ 收敛.
  \item 设 $ A, B \in \MM[\C] $, 若 $ AB = BA $, 则 $ A, B $ 至少有一个公共的特征向量.
  \item 设 $ \varphi $ 是 $ n $ 维复线性空间 $ V $ 上的线性变换, 求证 $ \varphi $ 可对角化的充要条件是对 $ \varphi $ 的任一特征值 $ \lambda_{0} $, 总有 $ \Ker(\varphi - \lambda_{0}I) \cap \Image(\varphi - \lambda_{0}I) = 0 $.
  \sitem 设在数域 $ \K $ 上, 一元多项式 $ f(x) = f_{1}f_{2} $, 且 $ (f_{1}, f_{2}) = 1 $, $ V $ 是数域 $ \K $ 上的 $ n $ 维线性空间, $ \varphi $ 是 $ V $ 上的线性变换, 证明 $ \Ker f(\varphi) = \Ker f_{1}(\varphi) \oplus \Ker f_{2}(\varphi) $.
  \item 设 $ f(x) $ 在 $ [a, +\infty) $ 上可微, 且对任意 $ x \in [a, +\infty) $, 都有
  \begin{align*}
      f(x+1) - f(x) = f'(x)
  \end{align*}
  若 $ \limit{x}{+\infty}f'(x) = c $, 证明 $ f'(x) = c $ 在 $ [a, +\infty) $ 上恒成立.
  \item 设 $ a_{n} > 0 $, $ \sum_{n=1}^{\infty}a_{n} < +\infty $, 对于 $ \alpha, \beta > 0 $, 且 $ \alpha + \beta > 1 $, 证明 $ \sum_{n=1}^{\infty} a_{n}^{\alpha}/n^{\beta} < +\infty $.
  \item 设 $ A \in \MM $, 对任意 $ 0\ne x \in \R^{n} $, 都有 $ x'Ax > 0 $, 证明 $ \abs{A} > 0 $.
  \begin{hint}
      这里给出两种提示
      \begin{method}
          \item 将 $ A $ 分解为对称阵与反对称阵的和, 再利用 \ref{item:反称加对角} 题得出结论.
          \item 注意到实矩阵 $ A $ 的虚特征值成对, 再利用 $ x'Ax > 0 $ 得到 $ A $ 的实特征值均为正, 那么 $ \abs{A} = \prod_{i = 1}^{n} \lambda_{i} > 0 $.
      \end{method}
  \end{hint}
  \begin{answer}
      我们用两种方法进行证明
      \begin{method}
          \item 将 $ A $ 分解为 $ B + C $, 其中 $ B $ 是对称阵, $ C $ 是反对称阵, 那么
          \begin{align*}
              x'Ax = x'Bx + x'Cx > 0,
          \end{align*}
          另外注意到 $ x'Cx\in \R $, 那么应当有
          \begin{align*}
              x'Cx = (x'Cx)' = -x'Cx \implies x'Cx = 0,
          \end{align*}
          于是 $ x'Bx > 0\,(0 \ne x \in \R^{n}) $, 即 $ B $ 是正定阵, 那么由 \ref{item:反称加对角} 题可知 $ \abs{A} = \abs{B + C} > 0 $.
          \item 设 $ A $ 的特征值 $ \lambda_{1}, \bar{\lambda}_{1}, \lambda_{2}, \bar{\lambda}_{2}, \dots, \lambda_{r}, \bar{\lambda}_{r}, \lambda_{2r + 1}, \dots, \lambda_{n} $, 其中前 $ 2r $ 个特征值为复特征值, 后 $ n - 2r $ 个为实数. 下面来说明 $ \lambda_{i} > 0\,(2r + 1 \le i \le n) $, 取 $ \lambda_{i} $ 的特征向量 $ \alpha_{i} \in \R^{n} $, 那么有
          \begin{align*}
              \alpha_{i}'A\alpha_{i} = \lambda_{i}\alpha_{i}'\alpha > 0,
          \end{align*}
          这说明 $ A $ 的全体实特征值全大于 $ 0 $, 于是
          \begin{align*}
              \abs{A} = \prod_{j = 1}^{r}\lambda_{j}\bar{\lambda}_{j}\prod_{i = 2r + 1}^{n} \lambda_{i} > 0.
          \end{align*}
      \end{method}
  \end{answer}
  \item 设有 $ n $ 阶分块对角阵
  \begin{align*}
      A = \begin{pmatrix}
          A_{1} & & \\
          & \ddots & \\
          & & A_{k}
      \end{pmatrix}\quad
      B = \begin{pmatrix}
          B_{1} & & \\
          & \ddots & \\
          & & B_{k}
      \end{pmatrix}
  \end{align*}
  其中 $ A_{i} $ 与 $ B_{i} $ 为同阶方阵, 假定矩阵 $ A_{i} $ 适合非零多项式 $ g_{i}(x) $, 且 $ g_{i}(x)\,(i = 1, \dots, k) $ 两两互素. 求证: 若对于每个 $ i $, 存在多项式 $ f_{i}(x) $, 使 $ B_{i} = f_{i}(A_{i}) $, 则必存在次数不超过 $ n-1 $ 的多项式 $ f(x) $, 使得 $ B = f(A) $.
  \item 设 $ n $ 阶方阵 $ A $ 的秩为 $ n-1 $, $ B $ 是同阶非零阵, 且有 $ AB = BA = 0 $, 证明: 存在不超过 $ n-1 $ 阶的多项式 $ f(x) $, 使得 $ B = f(A) $.
  \sitem 设 $ V $ 为数域 $ \K $ 上的 $ n $ 维线性空间, $ \varphi $ 是 $ V $ 上的线性变换, 其特征多项式与极小多项式分别设为 $ f(\lambda) $ 与 $ m(\lambda) $, 设
  \begin{align*}
      f(\lambda) = P_{1}(\lambda)^{r_{1}}P_{2}(\lambda)^{r_{2}}\dots P_{t}(\lambda)^{r_{t}}, \quad m(\lambda) = P_{1}(\lambda)^{s_{1}}P_{2}(\lambda)^{s_{2}}\dots P_{t}(\lambda)^{s_{t}}
  \end{align*}
  分别为 $ f(\lambda) $ 与 $ m(\lambda) $ 的不可约分解, 其中 $ P_{i}(\lambda) $ 为 $ \K $ 上互异的首一不可约多项式, $ r_{i}, s_{i} > 0\, (i = 1, 2, \dots, t) $. 设 $ V_{i} = \Ker P_{i}(\varphi)^{r_{i}} $, $ U_{i} = \Ker P_{i}(\varphi)^{s_{i}}\,(i = 1, 2, \dots, t) $. 求证:
  \begin{exercise}
      \item $ V = V_{1} \oplus V_{2} \oplus \dots \oplus V_{t} $, $ U = U_{1} \oplus U_{2} \oplus \dots \oplus U_{t} $, 且 $ U_{i} = V_{i}\,(i = 1, 2, \dots, t) $;
      \item $ \varphi|_{V_{i}} $ 的特征多项式为 $ P_{i}(\lambda)^{r_{i}} $, 极小多项式为 $ P_{i}(\lambda)^{s_{i}} $. 特别地, $ \dim V_{i} = r_{i}\deg P_{i}(\lambda) $.
  \end{exercise}
  \item 证明任一 $ n $ 阶复矩阵 $ A $ 都相似于一个复对称阵.
  \item 设 $ A $ 为 $ n $ 阶实对称矩阵, 求证: $ A $ 为半正定阵或半负定阵的充要条件是对任意满足 $ \alpha' A\alpha = 0 $ 的 $ n $ 维实向量 $ \alpha $, 都有 $ A\alpha = 0 $.
  \item 设 $ f(x) $ 在 $ [a, b] $ 上连续, $ (a, b) $ 上可导, 且 $ f(a) = f(b) $, 若 $ \abs{f'(x)} \le 1 $, 证明对任意的 $ x_{1}, x_{2} \in [a, b] $, 都有
  \begin{align*}
      \abs{f(x_{1}) - f(x_{2})} \le \frac{(b - a)}{2}.
  \end{align*}
  \item 设 $ f(x) $ 在 $ [0, 1] $ 上连续, 且 $ f(1) = 0 $, 证明 $ \set{f(x)x^{n}} $ 在 $ [0, 1] $ 上一致收敛.
  \item 设 $ p(x), q(x), r(x) $ 是数域 $ \K $ 上的正次数多项式, 且 $ p(x) $ 与 $ q(x) $ 互素, $ \deg r(x) < \deg p(x) + \deg q(x) $, 证明存在数域 $ \K $ 上的多项式 $ u(x), v(x) $, 满足 $ \deg u(x) < \deg p(x) $, $ \deg v(x) < \deg q(x) $, 使得 $ r(x) = p(x)v(x) + q(x)u(x) $.
  \item (\emph{Dini 定理}) 设函数列 $ \set{f_{n}(x)} $ 在有限闭区间 $ [a, b] $ 上连续. 如果对每一个 $ x\in [a, b] $, 数列 $ \set{f_{n}(x)} $ 关于 $ n $ 递减趋于 $ 0 $. 那么 $ f_{n}(x) $ 在 $ [a, b] $ 上一致收敛于 $ 0 $.
  \item 对任意 $ n\in \N $, $ f_{n}(x) $ 在 $ [a, b] $ 上关于 $ x $ 单调递增, 且 $ \set{f_{n}(x)} $ 收敛于连续函数 $ f(x) $. 证明: $ \set{f_{n}(x)} $ 在 $ [a, b] $ 上一致收敛于 $ f(x) $.
  \item 设 $ f(x) $ 在 $ [a, b] $ 上可导, 且 $ f'(x) $ 在 $ [a, b] $ 上可积, 记
  \begin{align*}
      A_{n} = \frac{b - a}{n}\sum_{i = 1}^{n} f\qty(a + \frac{i}{n}(b - a)) - \int_{a}^{b} f(x) \dx*,
  \end{align*}
  证明 $ \limit{n}{\infty} nA_{n} = (b - a)(f(b) - f(a))/2 $.
  \item 设 $ V $ 是数域 $ \K $ 上的 $ n $ 维线性空间, $ \sigma, \tau $ 是 $ V $ 上的线性变换, 且 $ \sigma^{2} = \tau^{2} = 0 $, 且 $ \sigma\tau + \tau\sigma = I_{V} $, 其中 $ I_{V} $ 是 $ V $ 上的恒等变换, 证明
  \begin{exercise}
      \item $ V = \Ker \sigma \oplus \Ker \tau $;
      \item $ V $ 必是偶数维线性空间.
  \end{exercise}
  \item 设函数 $ f(x) \in C[a, b] $, $ f(x) $ 不恒为 $ 0 $ 并且满足 $ 0 \le f(x) \le M $. 证明:
  \begin{align*}
      \qty(\int_{a}^{b} f(x) \cos x \dx*)^{2} + \qty(\int_{a}^{b} f(x) \sin x \dx*)^{2} + \frac{M^{2}(b-a)^{4}}{12} \ge \qty(\int_{a}^{b} f(x) \dx*)^{2}.
  \end{align*}
  \item 计算极限 $ \limit{\lambda}{\infty} \int_{0}^{1} \ln x\cos^{2}(\lambda x) \dx* $.
  \sitem 设 $ A, B \in \MM[\K][m\times n] $, 求证: 方程组 $ Ax = 0 $ 与 $ Bx = 0 $ 同解的充分必要条件是存在可逆阵 $ P $, 使得 $ B = PA $.
  \item 计算行列式
  \begin{align*}
      D = \begin{vNiceMatrix}[cell-space-limits = 1pt]
          1 & 0 & 0 & \cdots & 0 & 1 \\
          1 & {1}\choose{1} & 0 & \cdots & 0 & x\\
          1 & {2}\choose{1} & {2}\choose{2} & \cdots & 0 & x^{2} \\
          \vdots & \vdots & \vdots & \ddots & \vdots & \vdots\\
          1 & {n-1}\choose{1} & {n-1}\choose{2} & \cdots & {n-1}\choose{n-1} & x^{n-1}\\
          1 & {n}\choose{1} & {n}\choose{2} & \cdots & {n}\choose{n-1} & x^{n}
      \end{vNiceMatrix}
  \end{align*}
  \item 设 $ f(x) $ 是定义在 $ [0, 1] $ 上的单调非增函数, 对于任意 $ a \in (0, 1) $, 证明: $ \int_{0}^{a}f(x) \dx* \ge a\int_{0}^{1}f(x) \dx* $.
  \item 设 $ f(x) $ 在 $ [0, 1] $ 上 Riemann 可积, 且有 $ 0 < m \le f(x) \le M $. 证明:
  \begin{align*}
      1 \le \int_{0}^{1}f(x) \dx* \int_{0}^{1} \frac{1}{f(x)} \dx* \le \frac{(M+m)^{2}}{4mM}.
  \end{align*}
  \item 设开集 $ D\subset \R^{2} $, $ f: D \to \R $ , 如果 $ \pdv{f}{x}, \pdv{f}{y}, \pdv{f}{x}{y} $ 在 $ (\vb*{x}_{0}, \vb*{y}_{0}) $ 的某个邻域上存在, 且 $ \pdv{f}{x}{y} $ 在 $ (\vb*{x}_{0}, \vb*{y}_{0}) $ 处连续, 那么 $ \pdv{f}{y}{x} $ 在 $ (\vb*{x}_{0}, \vb*{y}_{0}) $ 处存在, 且
  \begin{align*}
      \pdv{f}{x}{y}{} (\vb*{x}_{0}, \vb*{y}_{0}) = \pdv{f}{y}{x}{} (\vb*{x}_{0}, \vb*{y}_{0}).
  \end{align*}
  \item 设 $ D \subset \R^{2} $ 是一个凸区域, $ f: D \to \R $ 有连续的一阶偏导数, 则 $ f $ 在 $ D $ 内为凸函数的充必条件为对任意 $ \vb*{x}, \vb*{y}\in D $, 有 $ f(\vb*{y}) \ge f(\vb*{x}) + (\vb*{y} - \vb*{x})\cdot\grad f(\vb*{x}) $.
  \item 设 $ \hom{A} $ 为数域 $ \K $ 上的 $ n\,(n \ge 3) $ 维线性空间 $ V $ 上的线性变换, $ \hom{A} $ 的特征多项式为
  \begin{align*}
      f(\lambda) = \lambda^{n} + a_{n-1}\lambda^{n-1} + a_{n-2}\lambda^{n-2} + \dots + a_{1}\lambda + a_{0},
  \end{align*}
  试证明:
  \begin{align*}
      a_{n - 2} = \frac{1}{2}\qty(\tr^{2}(\hom{A}) - \tr(\hom{A}^{2})).
  \end{align*}
  \item 设 $ a\ne 0 $, 计算积分:
  \begin{align*}
      \int_{0}^{+\infty}\frac{\dx*}{(1+x^{2})(1+x^{a})}.
  \end{align*}
  \item 设 $ A $ 是 $ n $ 阶实正定阵, $ x \in \R^{n} $ 是非零列向量, 求证:
  \begin{exercise}
      \item $ A + xx' $ 可逆.
      \item $ 0 < x'(A+xx')^{-1}x < 1 $.
  \end{exercise}
  其中 $ \lambda_{1}, \lambda_{2}, \dots, \lambda_{n} $ 为 $ A^{-1}B $ 的全体特征值.
  \item 设 $ A $ 是 $ n $ 阶半正定的实对称阵, $ S $ 为 $ n $ 阶实反对称阵, 满足 $ AS + SA = 0 $, 证明 $ \abs{A + S} > 0 $ 的充分必要条件为 $ \rank(A) + \rank(S) = n $.
  \item 设 $ A, B $ 为 $ n $ 阶正定实对称阵, 若 $ A - B $ 半正定, 证明 $ B^{-1} - A^{-1} $ 为半正定阵.
  \item 设 $ A $ 是 $ n $ 阶正定实对称阵, $ B $ 是同阶半正定实对称阵, 求证 $ \abs{A + B} \ge \abs{A} + \abs{B} $.
  \item 设 $ V $ 是 $ n $ 维酉空间, $ \varphi $ 是 $ V $ 上的线性变换, 求证: $ \varphi $ 是正规算子的充分必要条件是 $ \norm{\varphi(\alpha)} = \norm{\varphi^{*}(\alpha)} $ 对任意 $ \alpha \in V $ 都成立.
  \sitem 设 $ V $ 是 $ n $ 维酉空间, $ \varphi $ 是 $ V $ 上的线性变换, 求证: $ \varphi $ 是正规算子的充分必要条件是若 $ v $ 是 $ \varphi $ 属于特征值 $ \lambda $ 的特征向量, 则 $ v $ 也是 $ \varphi^{*} $ 属于特征值 $ \bar\lambda $ 的特征向量.
\end{exercise}
  \begin{tcolorbox}[title={同时上三角化/对角化/标准型化}]
      \begin{exercise}[resume=exer]
          \item 设 $ n $ 阶矩阵 $ \set{A_{i} : i = 1, 2, \dots, m} $ 两两可交换, 即 $ A_{i}A_{j} = A_{j}A_{i} $ 对一切 $ i, j $ 都成立, 假定每一个 $ A_{i} $ 均可对角化, 证明: 它们可同时对角化.
          \item 若 $ A, B \in \MM[\K] $, 且 $ AB = BA $, 假定 $ A, B $ 的特征值都在 $ \K $ 中, 证明: 存在 $ \K $ 上的可逆阵 $ P $, 使得 $ P^{-1}AP $ 与 $ P^{-1}BP $ 都是上三角矩阵.
          \item 设 $ A $ 为 $ n $ 阶正定实对称阵, $ B $ 为同阶对称阵, 则存在可逆阵 $ C $ 使得
          \begin{align*}
              C'AC = I_{n}, \quad C'BC = \diag\set{\lambda_{1}, \lambda_{2}, \dots, \lambda_{n}}
          \end{align*}
          其中 $ \lambda_{1}, \lambda_{2}, \dots, \lambda_{n} $ 是 $ A^{-1}B $ 的特征值.
          \item 设 $ A, B $ 都是 $ n $ 阶半正定实对称矩阵, 证明: 存在可逆阵 $ C $, 使得
          \begin{align*}
              C'AC = \diag\{\underbrace{1, \dots, 1}_{r\text{ 个 }}, 0, \dots, 0\}, \quad C'BC = \diag\set{\lambda_{1}, \dots, \lambda_{r}, \lambda_{r+1}, \dots, \lambda_{n}}
          \end{align*}
          \item 设 $ A $ 为 $ n $ 阶正定实对称阵, $ S $ 为同阶实反对称阵, 则存在可逆矩阵 $ C $, 使得
          \begin{align*}
              C'AC = I_{n}, \quad C'SC = \diag\set{\mqty(0 & b_{1} \\ -b_{1} & 0), \dots, \mqty(0 & b_{r} \\ -b_{r} & 0), 0, \dots, 0}
          \end{align*}
          \item 设 $ A_{i}\,(i = 1, 2, \dots, m) $ 是 $ m $ 个实对称(复正规)矩阵且两两可交换, 证明: 存在正交(酉)矩阵 $ P $, 使得 $ P'A_{i}P $ ($ P^{\symrm{H}}A_{i}P $)都是对角阵.
          \item 设 $ A, B $ 是两个 $ n $ 阶实正规矩阵, 且 $ AB = BA $, 证明: 存在正交矩阵 $ P $, 使得 $ P'AP $ 和 $ P'BP $ 同时为如下形状的分块对角矩阵: $ \diag\set{A_{1}, \dots, A_{r}, c_{2r+1}, \dots, c_{n}} $, 其中 $ c_{i} $ 是实数, $ A_{i} $ 为形如 $ \smqty(a_{i} & b_{i} \\ -b_{i} & a_{i}) $ 的二阶实矩阵.
          \item 设 $ A, B $ 是 $ n $ 阶实对称矩阵, 满足 $ AB + BA = 0 $, 证明: 若 $ A $ 半正定, 则存在正交矩阵 $ P $, 使得:
          \begin{align*}
              P'AP = \diag\set{\lambda_{1}, \dots, \lambda_{r}, 0, \dots, 0}, \quad P'BP = \diag\set{0, \dots, 0, \mu_{r + 1}, \dots, \mu_{n}}.
          \end{align*}
      \end{exercise}
  \end{tcolorbox}
  \begin{tcolorbox}[title={矩阵/线性算子分解}]
      \begin{exercise}[resume=exer]
          \item 设 $ A \in \MM[\C] $, 则 $ A $ 可以分解为 $ A = B + C $, 其中 $ B $ 为 Hermite 阵, $ C $ 为斜 Hermite 阵.
          \item 设 $ A\in \MM[\C] $, 则 $ A $ 可以分解成两个对角阵的乘积, 即 $ A = BC $, 且可以任意指定 $ B $ 或 $ C $ 为可逆阵.
          \item 设 $ A $ 是 $ n $ 阶(半)正定实对称阵, 则
          \begin{exercise}
              \item 存在主对角线上元素全等于 $ 1 $ 的上三角矩阵 $ B $, 使得 $ A = B'DB $, 其中 $ D $ 是(半)正定对角矩阵;
              \item (\emph{Cholesky 分解})存在主对角线上元素全为正(非负)的上三角阵 $ C $, 使得 $ A = C'C $.
          \end{exercise}
          \item (\emph{QR 分解}) 设 $ A $ 是 $ n $ 阶实(复)矩阵, 则 $ A $ 可以分解为 $ A = QR $, 其中 $ Q $ 是正交(酉)矩阵, $ R $ 是一个上三角阵, 且主对角线上的元素非负, 若 $ A $ 可逆, 则这样的分解唯一.
          \item (\emph{极分解}) 设 $ V $ 是 $ n $ 维酉(欧式)空间, $ \varphi $ 是 $ V $ 上的任意一个线性算子, 则存在 $ V $ 上的酉(正交)算子 $ \omega $ 以及 $ V $ 上的半正定自伴随算子 $ \psi $, 使得 $ \varphi = \omega\psi $, 其中 $ \psi $ 是唯一的, 并且若 $ \varphi $ 是非异线性算子, 则 $ \omega $ 也唯一 \footnote{ $ \varphi $ 也可以做这样的分解: $ \varphi = \psi_{1}\omega_{1} $ 其中 $ \omega_{1} $ 为酉(正交)算子, $ \psi_{1} $ 为半正定自伴随算子, 这样的分解也叫极分解, 下面矩阵版本的同理.}.
          \item \label{item:矩阵极分解}(\emph{极分解}) 设 $ A \in \MM $, 则存在 $ n $ 阶正交阵 $ Q $ 以及 $ n $ 阶半正定对称阵 $ S $, 使得 $ A = QS $. 又设 $ B \in \MM[\C] $, 则存在 $ n $ 阶酉矩阵 $ U $, 以及 $ n $ 阶半正定 Hermite 阵 $ H $, 使得 $ B = UH $, 上述分解式当 $ A, B $ 为非异阵的时候被唯一确定.
          \item (\emph{谱分解}) 设 $ V $ 是 $ n $ 维欧式(酉)空间, $ \varphi $ 为 $ V $ 上的自伴随(正规)算子. $ \lambda_{1}, \lambda_{2}, \dots, \lambda_{k} $ 为 $ \varphi $ 的全体不同特征值, $ W_{i} $ 为属于 $ \lambda_{i} $ 的特征子空间, 则 $ V $ 是 $ W_{i}\,(i = 1, 2, \dots, k) $ 的正交直和. 这时若设 $ E_{i} $ 是 $ V $ 到 $ W_{i} $ 的正交投影, 则 $ \varphi $ 有如下分解式:
          \begin{align*}
              \varphi = \lambda_{1}E_{1} + \lambda_{2}E_{2} + \dots + \lambda_{k}E_{k}.
          \end{align*}
          \item (\emph{奇异值分解}) 设 $ V, U $ 分别为 $ n, m $ 维欧式(酉)空间, $ \varphi : V \to U $ 是线性映射, 则存在 $ V $ 和 $ U $ 的标准正交基, 使 $ \varphi $ 在这两组基下的表示矩阵为 $ \smqty(S & 0 \\ 0 & 0) $, 其中 $ S = \diag\set{\sigma_{1}, \sigma_{2}, \dots, \sigma_{r}} $, $ \sigma_{1} \ge \sigma_{2} \ge \dots \ge \sigma_{r} > 0 $ 是 $ \varphi $ 的非零奇异值.
          \item (\emph{奇异值分解}) 设 $ A $ 是 $ m\times n $ 的实(复)矩阵, 则存在 $ m $ 阶正交(酉)矩阵 $ P $, $ n $ 阶正交(酉)矩阵 Q, 使得 $ A = P\smqty(S & 0 \\ 0 & 0)Q $, 其中 $ S = \diag\set{\sigma_{1}, \sigma_{2}, \dots, \sigma_{r}} $, $ \sigma_{1} \ge \sigma_{2} \ge \dots \ge \sigma_{r} > 0 $ 是 $ A $ 的非零奇异值.
          \hitem (\emph{Jordan -- Chevalley 分解}) 设 $ A \in \MM[\C] $, 则 $ A $ 可以分解为 $ A = B + C $, 其中 $ B, C $ 符合以下条件:
          \begin{exercise}
              \item $ B $ 是一个相似可对角化矩阵;
              \item $ C $ 是一个幂零阵;
              \item $ BC = CB $;
              \item $ B, C $ 均可以表示为 $ A $ 的多项式.
          \end{exercise}
          不仅如此, 上述满足 (1) -- (3) 的分解是唯一的.
      \end{exercise}
  \end{tcolorbox}
  \begin{exercise}[resume=exer]
      \sitem 设 $ A = (a_{ij}) $ 是 $ n $ 阶复矩阵, $ \lambda_{1}, \lambda_{2}, \dots, \lambda_{n} $ 是其特征值, 求证: $ A $ 是正规矩阵的充分必要条件是
      \begin{align*}
          \sum_{i=1}^{n}\abs{\lambda_{i}}^{2} = \tr(A^{\symrm{H}}A) = \sum_{i, j=1}^{n}\abs{a_{ij}}^{2}.
      \end{align*}
      \item 利用 \ref{item:矩阵极分解} 题证明: 存在正交阵 $ Q_{1}, Q_{2} $, 使得 $ Q_{1} A Q_{2} = \diag(\lambda_{1}, \lambda_{2}, \dots, \lambda_{n}) $, 并且 $ \lambda_{1}^{2}, \lambda_{2}^{2}, \dots, \lambda_{n}^{2} $ 是 $ A'A $ 的特征值.
      \item 利用 $ \cos px $ 在 $ [-\pi, \pi] $ 上的 Fourier 展开证明积分
      \begin{align*}
          \int_{0}^{+\infty} \frac{x^{p-1}}{1+x} \dx* = \frac{\pi}{\sin p\pi}\quad (0<p<1).
      \end{align*}
      \item 判断含参变量反常积分 $ \int_{0}^{+\infty}\sqrt{u}\me^{-ux^{2}} \dx* $ 对于 $ u \in [0, +\infty) $ 的一致收敛性.
    % TODO duplicated of 5
      \item 设 $ f(x)\in C^{1}[a, b] $, 记
      \begin{align*}
          A = \frac{1}{b - a}\int_{a}^{b} f(x) \dx*
      \end{align*}
      证明:
      \begin{align*}
          \int_{a}^{b}(f(x) - A)^{2} \dx* \le (b-a)^{2} \int_{a}^{b} (f'(x))^{2} \dx*.
      \end{align*}
      \item 计算积分 $ \int_{0}^{\pi/2} \ln(a^{2}\sin^{2}x + b^{2}\cos^{2} x)\dx*\,(a^{2} + b^{2} \ne 0) $.
      \item 已知 $ f(x) \in C[-1, 1] $, 证明:
      \begin{align*}
          \limit{y}{0+} \int_{-1}^{1}\frac{yf(x)}{x^{2} + y^{2}} \dx* = \pi f(0).
      \end{align*}
      \item 设 $ A $ 是 $ n $ 阶实对称阵:
      \begin{align*}
          A = \begin{pmatrix}
              a_{1} & b_{1} & & & \\
              b_{1} & a_{2} & b_{2} & & \\
              & b_{2} & a_{3} & \ddots & \\
              & & \ddots & \ddots & b_{n-1} \\
              & & & b_{n-1} & a_{n}
          \end{pmatrix},\quad b_{j} \ne 0
      \end{align*}
      证明:
      \begin{exercise}
          \item $ \rank(A) \ge n-1 $;
          \item $ A $ 的特征值各不相同.
      \end{exercise}
      \item 设 $ \lambda_{1}, \lambda_{2}, \dots, \lambda_{k} $ 是 $ A\in \MM[\C] $ 的全体特征值, 若有 $ \rank(\lambda_{i}I - A) = \rank(\lambda_{i}I - A)^{2}\,(i = 1, 2, \dots, k) $, 证明 $ A $ 可以相似对角化.
      \item 证明
      \begin{align*}
          \int_{1}^{+\infty}\exp\qty(-\frac{1}{\alpha^{2}}\qty(x-\frac{1}{\alpha})^{2}) \dx*
      \end{align*}
      在 $ (0, 1) $ 上一致收敛.
      \item 设 $ f(x) $ 为单调递减的正值函数, 证明 $ \int_{a}^{+\infty} f(x) \dx* $ 与 $ \int_{a}^{+\infty} f(x)\sin^{2}{x} \dx* $ 同时敛散.
      \item 设 $ \hom{A}, \hom{B} $ 均是 $ n $ 维线性空间 $ V $ 上的线性变换, 且 $ \Ker \hom{A} \subset \Ker \hom{B} $, 证明, 存在线性变换 $ \hom{T} $, 使得 $ \hom{B} = \hom{TA} $.
      \item 计算极限
      \begin{align*}
          \limit{r}{+\infty} r\int_{0}^{+\infty} \me^{-x^{2}}\sin(rx) \dx*.
      \end{align*}
      \item 设数域 $ \K $ 上的全体矩阵 $ \MM[\K] $ 上有线性变换 $ \sigma(X) = AX - XA $, 其中 $ A \in \MM[\K] $:
      \begin{exercise}
          \item 若 $ A $ 为幂零阵, 证明 $ \sigma $ 为幂零变换;
          \item 若 $ A $ 有特征值 $ \lambda_{1}, \lambda_{2}, \dots, \lambda_{n} $, 证明 $ \lambda_{i} - \lambda_{j}\,(1\le i, j\le n) $ 为 $ \sigma $ 的特征值.
      \end{exercise}
      \item 设 $ f(x) $ 是实系数多项式, 若有 $ f(x) \mid f(x^2 + x + 1) $, 证明 $ 2 \mid \deg f(x) $.
      \item 设函数 $ f(x) $ 在 $ [0, 1] $ 上连续, 证明:
      \begin{align*}
          \int_{0}^{1} \dx*\int_{x}^{1} \dd{y}\int_{x}^{y}f(x)f(y)f(z)\dd{z} = \frac{1}{6}\qty(\int_{0}^{1}f(t) \dd{t})^{3}.
      \end{align*}
      \item 设 $ A, B $ 都是 $ n $ 阶正交阵, 证明 $ \abs{\det(A+B)} \le 2^{n} $.
      \item 设 $ A, B \in \MM[\C] $, 若 $ AB = BA = 0 $, $ \rank(A^{2}) = \rank(A) $, 证明
      \begin{align*}
          \rank(A+B) = \rank(A) + \rank(B).
      \end{align*}
      \item 设 $ V $ 是 $ n $ 维内积空间,  $ V_{1}, V_{2}, \dots, V_{r} $ 是 $ V $ 的 $ r $ 个真子空间, 证明: 存在 $ V $ 的一组标准正交基 $ \alpha_{1}, \alpha_{2}, \dots, \alpha_{n} $, 使得对任意的 $ 1 \le i \le n, 1\le j \le r $ 都有 $ \alpha_{i} \notin V_{j} $.
      \item 设 $ f(x) $ 在 $ \R $ 上连续且有界, 证明对任意 $ T > 0 $, 都存在一个数列 $ \set{x_{n}} $, 满足
      \begin{align*}
          \limit{n}{\infty} x_{n} = +\infty, \quad \limit{n}{\infty}(f(x_{n}+T) - f(x_{n})) = 0.
      \end{align*}
      \item (华师 2020) 计算级数 $ \sum_{n=0}^{\infty}\frac{(-1)^{n}}{3^{n}(2n+1)} $.
      \item (华师 2020) 计算
      \begin{align*}
          \oiint_{\Sigma} (z^{2}+x) \dd{y}\dd{z} + \sqrt{z} \dx*\dd{y}
      \end{align*}
      其中 $ \Sigma $ 为抛物面 $ z = (x^{2}+y^{2})/2 $ 在平面 $ x=0 $ 与 $ x=2 $ 之间的部分, 方向取下侧.
      \item 计算第二型曲线积分
      \begin{align*}
          \oint_{L} \frac{4x-y}{4x^{2}+y^{2}}\dx* + \frac{x-y}{4x^{2}+y^{2}}\dd{y}
      \end{align*}
      其中 $ L $ 为 $ x^{2} + y^{2} = 2 $ 的圆周, 方向为逆时针.
      \item (浙大 2018) 设 $ A $ 为 $ n $ 阶非零实方阵, 且 $ A^{2} = A $, 设 $ \rank(A) = r $, 求证 $ A $ 正交相似于分块矩阵: $ \smqty(I_{r} & 0 \\ B & 0) $, 其中 $ I_{r} $ 为 $ r $ 阶单位阵, $ B $ 为某一实矩阵.
      \item 求抛物面 $ x^{2} + y^{2} + az = 4a^{2} $ 将球体 $ x^{2} + y^{2} + z^{2} \le 4az $ 分成两部分的体积之比.
      \item 计算第一型曲面积分
      \begin{align*}
          I = \iint_{S} \frac{x^{3} + y^{3} + z^{3}}{1-z} \dd{S}
      \end{align*}
      其中 $ S $ 为 $ x^{2} + y^{2} = (1-z)^{2}\,(0 \le z \le 1) $.
      \item 计算 $ I = \iiint_{\Omega} x^{2}\sqrt{x^{2} + y^{2}}\,\dx\dd y\dd z $, 其中 $ \Omega $ 是曲面 $ z = \sqrt{x^{2} + y^{2}} $ 与 $ z = x^{2} + y^{2} $ 围成的有界区域.
      \item 计算第二型曲面积分
      \begin{align*}
          \iint_{\Sigma} x\,\dd y\dd z + (2 + y^{3})\,\dd z\dx + z^{3}\,\dx\dd y
      \end{align*}
      其中 $ \Sigma $ 为 $ x = \sqrt{1 - 2y^{2} - 3z^{2}} $, 方向取 $ x $ 轴正向.
      \item 计算积分 $ I = \iint_{S} (x + z) \dd{\sigma} $, 其中 $ S $ 是曲面 $ x^{2} + z^{2} = 2az\,(a > 0) $ 被曲面 $ z = \sqrt{x^{2} + y^{2}} $ 所截取的有限部分.
      \item 设 $ A, B \in \MM $ 为对称阵, 若 $ A $ 正定, 证明 $ AB $ 的特征值全为实数.
      \item 设 $ \varphi $ 是实数域上 $ n\,(n\ge 1) $ 维线性空间 $ V $ 的一个线性变换, 证明 $ \varphi $ 至少有一个维数是 $ 1 $ 或 $ 2 $ 的不变子空间.
      \begin{hint}
          从特征值去考虑, 如果 $ \varphi $ 有实特征值, 那么该特征值对应的特征向量张成的空间就是一个一维的不变子空间. 如果不存在实根, 那么就化成矩阵去考虑, 将共轭的特征值对应的特征向量拆开, 然后可以得到二维的不变子空间.
      \end{hint}
      \begin{answer}
          若 $ \varphi $ 存在实特征值 $ \lambda $, 它对应的特征向量为 $ \alpha $, 那么 $ \alpha $ 张成的 $ 1 $ 维线性空间 $ L(\alpha) $ 是 $ \varphi $ 的不变子空间.

          若 $ \varphi $ 不存在实特征值, 那么取一组基 $ e_{1}, e_{2}, \dots, e_{n} $, 设 $ A $ 是 $ \varphi $ 在这组基下的矩阵. 那么 $ A $ 与 $ \varphi $ 有相同的特征值, 那么取 $ A $ 的一个复特征值 $ \mu = a + b\mi $, 它对应的复特征向量 $ \alpha = \beta + \gamma\mi $, 其中 $ a, b \in \R,\ \beta, \gamma \in \R^{n} $, 于是我们可以得到
          \begin{align*}
              A(\beta + \gamma\mi) = (a + b\mi)(\beta + \gamma\mi) = (a\beta - b\gamma) + (a\gamma + b\beta)\mi,
          \end{align*}
          那么对应实部和虚部可得:
          \begin{align*}
              A\beta & = a\beta - b\gamma\\
              A\gamma & = a\gamma + b\beta
          \end{align*}

      \end{answer}
      \item 设 $ A $ 是 $ n $ 阶半正定实对称矩阵, $ B $ 是 $ n $ 阶实矩阵, 若对于某个正整数 $ k \ge 2 $, 有 $ A^{k}B = BA^{k} $ 成立, 证明 $ AB = BA $.
      \item 设函数 $ f \in C(\R) $, 若有 $ f(f(x)) = x $, 证明存在 $ \xi\in \R $ 使得 $ f(\xi) = \xi $.
      \item 设 $ f $ 是有整系数多项式, 若 $ f $ 在有理数域上不可约, 证明 $ f $ 在复数域上没有重根.
      \item 计算
      \begin{align*}
          I = \oint_{L} (y^{2} + z^{2}) \dx* + (z^{2} + x^{2}) \dd{y} + (x^{2} + y^{2}) \dd{z},
      \end{align*}
      其中 $ L $ 是曲面 $ x^{2} + y^{2} + z^{2} = 4x $ 与 $ x^{2} + y^{2} = 2x $ 的交线的 $ z \ge 0 $ 的部分, 曲线方向为从 $ z $ 轴上方向下看是顺时针方向.
      \item 计算曲线积分 $ I = \int_{L} z^{2} \dd{s} $, 其中 $ L $ 为 $ x^{2} + y^{2} + z^{2} = 1 $ 与 $ x + y = 1 $ 的交线.
      \item 设矩阵 $ A \in \MM[\Q] $, 其中 $ n\ge 2 $, 设 $ f(\lambda) = \lambda^{n} + 2\lambda^{n-1} + 2 $, 若有 $ f(A) = 0 $, 证明 $ f(\lambda) $ 是 $ A $ 的特征多项式.
      % TODO Duplicate of 95 奇数维 V 实, AB = BA
      \item
      \item (中科院 2019) 设有 $ n + 1 $ 个列向量 $ \alpha_{1}, \alpha_{2}, \dots, \alpha_{n}, \beta \in \R^{n} $, $ A $ 是一个 $ n $ 阶实对称正定阵, $ \alpha_{i}' $ 为 $ \alpha_{i} $ 的转置, 如果满足下列条件:
      \begin{exercise}
          \item $ \alpha_{j} \ne 0\, (j = 1, 2, \dots, n) $;
          \item 对于任意 $ i \ne j \,(i, j = 1, 2, \dots, n) $, 都有 $ \alpha_{i}' A \alpha_{j} = 0 $;
          \item $ \beta $ 与每个 $ \alpha_{j}\,(j = 1, 2, \dots, n) $ 都正交;
      \end{exercise}
      证明 $ \beta = 0 $.
      \item (浙大 2020, 与 \ref{item:CtoR} 题类似) 若 $ A, B \in \MM $, 存在 $ n $ 阶可逆复方阵 $ X $ 使得 $ XA + 2BX = 0 $, 证明存在可逆实方阵 $ Y $, 使得 $ YA + 2BY = 0 $.
      \item (浙大 2020) 求行列式 $ \abs{A} $, 其中 $ A $ 的第 $ (i, j) $ 元为 $ \sgn(i - j) $.
      \item  设 $ \hom{A} $ 是实数域 $ \R $ 上 $ n\,(n\ge 1) $ 维线性空间 $ V $ 上的线性变换, 证明 $ \hom{A} $ 至少有一个维数是 $ 1 $ 或 $ 2 $ 的不变子空间.
      \item (华师 2021) 设 $ c_{1}, c_{2}, c_{3} $ 是多项式 $ f(x) = 2x^{3} - 4x^{2} + 6x - 1 $ 的三个复根, 求 $ (c_{1}c_{2} + c_{3}^{2})(c_{2}c_{3} + c_{1}^{2})(c_{1}c_{3} + c_{2}^{2}) $.
      \item (华师 2021) 设实矩阵
      \begin{align*}
          A = \begin{pmatrix}
              a & b \\ c & d
          \end{pmatrix},\quad a, b, c, d > 0
      \end{align*}
      证明: 一定存在 $ A $ 的特征向量 $ (x, y)'\in \R^{2} $, 满足 $ x, y > 0 $.
      \item (华师 2021) 设 $ A \in \MM $, $ B $ 是 $ n $ 阶实对称正定阵.
      \begin{exercise}
          \item 证明: 存在唯一 $ n $ 阶实矩阵 $ C $ 满足 $ BC + CB = A $;
          \item 证明: 对 $ (1) $ 中的实矩阵 $ C $, $ BC = CB $ 当且仅当 $ AB = BA $.
      \end{exercise}
      \item 设数列 $ \set{na_{n}} $ 收敛, 级数 $ \sum_{n=1}^{\infty}n(a_{n} - a_{n-1}) $ 收敛, 证明 $ \sum_{n=1}^{\infty}a_{n} $ 也收敛.
      \item 求 $ \limit{n}{+\infty} (I_{n}/\ln n) $, 其中
      \begin{align*}
          I_{n} = \int_{0}^{\pi/2} \frac{\sin^{2}(nt)}{\sin t}\dd{t}.
      \end{align*}
      \sitem 设 $ I(x) = \int_{0}^{+\infty} x^{3/2}\exp(-x^{2}y^{2})\dd{y} $, 证明 $ I(x) $ 关于 $ x\geq 0 $ 一致收敛.
      \item 计算曲线积分: $ I = \int_{L} (x^{2} - y^{2}) \dx* + (x^{2} - z^{2}) \dd{y} + (y^{2} - z^{2})\dd{z} $ , 其中 $ L $ 是曲面 $ x = \sqrt{2z - y^{2} - z^{2}} $ 与 $ x + z = 1 $ 的交线上, 从点 $ (0, -1, 1) $ 到点 $ (0, 1, 1) $ 的一段有向弧.
      \sitem 记 $ \abs{M} $ 为矩阵 $ M $ 的行列式, 设 $ A, B\in \MM $,
      \begin{exercise}
          \item 证明
          \begin{align*}
              \begin{vmatrix}
                  A & B \\ -B & A
              \end{vmatrix} = \abs{A + \sqrt{-1}B}\cdot\abs{A - \sqrt{-1}B}.
          \end{align*}
          \item 证明
          \begin{align*}
              \begin{vmatrix}
                  A & B \\ B & A
              \end{vmatrix} = \abs{A + B}\cdot\abs{A - B}.
          \end{align*}
      \end{exercise}
      \item 设函数
      \begin{align*}
          f(x) = \begin{cases}
              \int_{0}^{x}\sin(1/x) \dx* & ,x \ne 0\\
              0 & ,x = 0
          \end{cases}
      \end{align*}
      则 $ f(x) $ 在 $ x = 0 $ 处是否可导?
      \item 求积分
      \begin{align*}
          I = \int_{0}^{1} \frac{\ln(1 + x)}{1 + x^{2}} \dx*.
      \end{align*}
      \item 设 $ D $ 是 $ xOy $ 平面上由曲线 $ y = \sqrt{x} $ 和直线 $ y = x $ 围成的图形, 求 $ D $ 绕直线 $ y = x $ 旋转产生的旋转体的体积.
      \item 设 $ A $ 为 $ n $ 阶实对称矩阵, 证明: $ \rank(A) = n $ 的充分必要条件为存在 $ n $ 阶实矩阵 $ B $, 使得 $ AB + B'A $ 为正定阵.
      \sitem\label{item:f有限区间一致收敛} 设 $ f(x) $ 是 $ \R $ 上的连续函数, 设
      \begin{align*}
          f_{n}(x) = \sum_{k=0}^{n-1}\frac{1}{n}f\qty(x+\frac{k}{n}),
      \end{align*}
      证明: 函数列 $ \set{f_{n}(x)} $ 在任意有限区间上一致收敛.
      \item 与 \ref{item:f有限区间一致收敛} 很类似, 如果令 $ f(x) = \sin(x) $, 证明 $ \set{f_{n}(x)} $ 在 $ \R $ 上一致收敛.
      \item 判断下列级数的敛散性
      \begin{align*}
          \sum_{n = 1}^{\infty} (-1)^{n - 1} \frac{\sqrt{n}}{\sqrt{n} + (-1)^{n-1}}\sin\frac{1}{\sqrt{n}}.
      \end{align*}
      \sitem 设 $ f(x) \in C^{2}(0, 1) $, 满足 $ f(0) = f(1) = 0 $, 且对任意 $ x\in (0, 1) $, $ f(x) \ne 0 $, 证明积分不等式
      \begin{align*}
          \int_{0}^{1} \abs{\frac{f''(x)}{f(x)}} \dx* > 4.
      \end{align*}
      \item 设 $ V $ 是 $ n $ 维欧式空间, 证明
      \begin{exercise}
          \item 对于任意两个不同的单位向量 $ \alpha, \beta \in V $, 总存在 $ V $ 上的镜面变换 $ \hom{A} $, 使得 $ \hom{A}(\alpha) = \beta $;
          \item 证明 $ V $ 上的任意正交变换 $ \hom{B} $ 都可以表示成有限个镜面变换的乘积, 即存在 $ m $ 个镜面变换 $ \hom{A}_{1}, \hom{A}_{2}, \dots, \hom{A}_{m} $, 使得
          \begin{align*}
              \hom{B} = \hom{A}_{1}\hom{A}_{2}\dots\hom{A}_{m}.
          \end{align*}
      \end{exercise}
      \item 计算积分 $ \int_{0}^{\pi/2} \ln (a^{2}\sin^{2}x + b^{2}\cos^{2}x) \dx* $, 其中 $ a, b $ 不全为 $ 0 $.
      \begin{hint}
          令 $ I $ 为 $ a $ 的函数
          \begin{align*}
              I(t) = \int_{0}^{\pi/2} \ln (t^{2}\sin^{2}x + b^{2}\cos^{2}x) \dx*
          \end{align*}
          利用方程组解出 $ I'(t) $, 再计算 $ \int_{b}^{a} I'(t) \dd{t} $, 最后再考虑 $ a $ 或 $ b $ 为 $ 0 $ 的情况.
      \end{hint}
      \begin{answer}
          不妨设 $ a, b > 0 $ 记
          \begin{align*}
              I(t) = \int_{0}^{\pi/2} \ln (t^{2}\sin^{2}x + b^{2}\cos^{2}x) \dx*, \quad t > 0
          \end{align*}
          则待求积分为 $ I(a) $, 由于 $ \ln (t^{2}\sin^{2}x + b^{2}\cos^{2}x) $ 在 $ (x, t) \in [0, \pi/2]\times(0, +\infty) $ 上连续可微, 故有
          \begin{align*}
              I'(t) = \int_{0}^{\pi/2} \pdv{t}\ln (t^{2}\sin^{2}x + b^{2}\cos^{2}x) \dx* = \int_{0}^{\pi/2} \frac{2t\sin^{2}x}{t^{2}\sin^{2}x + b^{2}\cos^{2}x} \dx*
          \end{align*}
          下面来计算 $ I'(t) $. 记
          \begin{align*}
              A = \int_{0}^{\pi/2} \frac{\sin^{2}x}{t^{2}\sin^{2}x + b^{2}\cos^{2}x} \dx*, \quad B = \int_{0}^{\pi/2} \frac{\cos^{2}x}{t^{2}\sin^{2}x + b^{2}\cos^{2}x} \dx*,
          \end{align*}
          那么可以看出
          \begin{align}\label{eq:t2A+b2B}
              t^{2}A + b^{2}B = \frac{\pi}{2},
          \end{align}
          另外
          \begin{align*}
              A + B = \int_{0}^{\pi/2} \frac{1}{t^{2}\sin^{2}x + b^{2}\cos^{2}x} \dx* = \int_{0}^{\pi/2} \frac{\dd\tan x}{t^{2}\tan^{2}x + b^{2}},
          \end{align*}
          令 $ u = \tan x $, 则
          \begin{align}\label{eq:A+B=pi/2tb}
              A + B = \int_{0}^{+\infty} \frac{\dd u}{t^{2}u^{2} + b^{2}} = \frac{1}{tb}\arctan\frac{ut}{b}\bigg|_{u = 0}^{+\infty } = \frac{\pi}{2tb}.
          \end{align}
          联立等式 \eqref{eq:t2A+b2B} 与 \eqref{eq:A+B=pi/2tb} 即可解出
          \begin{align*}
              A = \frac{\pi}{2}\frac{1}{t(t + b)}
          \end{align*}
          则 $ I'(t) = 2tA = \pi/(t + b) $. 于是有
          \begin{align*}
              I(a) - I(b) = \int_{b}^{a} I'(t) \dd{t} = \pi\ln\frac{a + b}{2b},
          \end{align*}
          又因为 $ I(b) = \int_{0}^{\pi/2}\ln b^{2} \dx* = \pi\ln b $, 所以
          \begin{align*}
              I(a) = \pi\ln\frac{a + b}{2},
          \end{align*}

          再考虑 $ a > 0, b = 0 $ 的情况, 利用
          \begin{align*}
              \int_{0}^{\pi/2} \ln\sin x\dx* = \int_{0}^{\pi/2} \ln\cos x\dx* = -\frac{\pi}{2} \ln 2
          \end{align*}
          可得
          \begin{align*}
              \int_{0}^{\pi/2} \ln(a^{2}\sin^{2}x) \dx* = \int_{0}^{\pi/2} 2\ln a + 2\ln \sin x\dx* = \pi\ln\frac{a}{2},
          \end{align*}
          这说明了对于任意 $ a^{2} + b^{2} \ne 0 $, 总有
          \begin{align*}
              I(a) = \int_{0}^{\pi/2} \ln (a^{2}\sin^{2}x + b^{2}\cos^{2}x) \dx* = \pi\ln\frac{\abs{a} + \abs{b}}{2}.
          \end{align*}
      \end{answer}
      \item 设 $ A $ 为 $ n $ 阶正定阵, $ X \in \R^{n} $ 为非零列向量, 证明
      \begin{exercise}
          \item 矩阵 $ A + XX' $ 可逆,
          \item $ 0 < X'(A + XX')^{-1}X < 1 $.
      \end{exercise}
      \item (中科院 2021) 设 $ \hom{A} $ 是线性空间 $ V $ 上的可逆线性变换, $ v_{1}, v_{2}, \dots, v_{m} $ 张成 $ V $, 且
      \begin{align*}
          \hom{A}(v_{i}) \in \set{v_{1}, v_{2}, \dots, v_{m}}, \quad i = 1, 2, \dots, m,
      \end{align*}
      求证 $ \hom{A} $ 可对角化, 且特征值都为单位根.
      \item (中科院 2021) 证明 $ \abs{\int_{a}^{a + 1} \sin t^{2} \dd{t}} \le a^{-1}\,(a > 0) $, 也可以证明更强的结论: $ \abs{\int_{a}^{a + 1} \sin t^{2} \dd{t}} < a^{-1}\,(a > 0) $.
      \item (中科院 2021) 求积分
      \begin{align*}
          I = \iint_{D} \frac{x^{2} + y^{2} - 2}{(x^{2} + y^{2})^{5/2}}\,\dx\dd y,
      \end{align*}
      其中 $ D: x^{2} + y^{2} \ge 2,\ x \le 1 $.
      \item 计算积分
      \begin{align*}
          \iint_{S} (x + z) \dd{S},
      \end{align*}
      其中 $ S $ 是曲面 $ x^{2} + y^{2} = 2az\,(a > 0) $ 被曲面 $ z = \sqrt{x^{2} + y^{2}} $ 所截取的有限部分.
      \item 设 $ W = \set{A : \tr(A) = 0, A \in \MM} $, 证明 $ W $ 是 $ \MM $ 的子空间, 并且求它的一组基.
      \begin{hint}
          首先每个 $ E_{ij}\,(i \ne j) $ 均是 $ W $ 的元素, 再考虑对角线元素的特点.
      \end{hint}
      \begin{answer}
          $ W $ 是子空间是显然的. 下面求维数: 可知 $ W_{1} = \set{\sum_{i \ne j} x_{ij}E_{ij} : x_{ij} \in \R} $ 是 $ W $ 的 $ (n - 1)n $ 维的子空间, 而 $ W_{2} = \set{\sum_{i = 1}^{n} x_{i}E_{ii} : x_{1} + x_{2} + \dots + x_{n} = 0} $ 是 $ W  $ 的 $ n - 1 $ 维的子空间, 且 $ W_{2} \cap W_{1} = 0 $. 则
          \begin{align*}
              \dim W \ge \dim(W_{1} + W_{2}) = \dim W_{1} + \dim W_{2} = n^{2} - 1
          \end{align*}
          又因为 $ W $ 是 $ \MM $ 的真子空间, 则 $ \dim W = n^{2} - 1 $, $ W = W_{1} \oplus W_{2} $.
          $ W $ 是 $ \MM $ 的 $ n^{2} - 1 $ 维的子空间, 且 $ \set{E_{ij} : i \ne j} \cup \set{E_{11} - E_{kk} : k > 1} $ 是 $ W $ 的一组基.
      \end{answer}
      \item 已知 $ A, C $ 是 $ n $ 阶正定对称矩阵, 且矩阵方程 $ AX + XA = C $ 有唯一解 $ B $, 证明 $ B $ 也是正定实对称阵.
      \begin{hint}
          将 $ AB + BA = C $ 两侧同时转置, 就可以得到 $ B $ 是对称阵, 再设 $ B $ 的特征值 $ \lambda $ 与特征向量 $ \alpha $, 考虑 $ \alpha'(AB + BA)\alpha > 0 $ 即可.
      \end{hint}
      \begin{answer}
          将 $ AB + BA = C $ 两侧同时转置, 可以得到 $ B'A + AB' = C $, 由于 $ B $ 是该矩阵方程的唯一解, 那么就有 $ B = B' $, 也即 $ B $ 是对称阵. 那么要说明 $ B $ 是正定阵, 只要说明 $ B $ 的特征值全为正数. 设 $ B $ 的特征值 $ \lambda $ 与对应的特征向量 $ \alpha $, 于是由 $ C $ 的正定性可知
          \begin{align*}
              \alpha'(AB + BA)\alpha = \alpha'C\alpha > 0,
          \end{align*}
          也就是
          \begin{align*}
              \alpha'A\lambda\alpha + \alpha'\lambda A\alpha = 2\lambda\alpha' A\alpha > 0,
          \end{align*}
          那么由 $ A $ 的正定性可知 $ \alpha' A\alpha > 0 $, 那么 $ \lambda > 0 $, 这就说明了 $ B $ 是正定实对称阵.
      \end{answer}
      \item 设 $ A \in \MM $, 定义 $ \MM $ 上的线性变换:
      \begin{align*}
          L_{A} : X \in \MM \mapsto AX;\\
          R_{A} : X \in \MM \mapsto XA,
      \end{align*}
      证明存在 $ \MM $ 上的可逆线性变换 $ T $, 使得 $ L_{A} = TR_{A}T^{-1} $.
      \begin{hint}
          首先由 Jordan 标准型理论可知 $ A' $ 与 $ A $ 相似, 即 $ AT_{1} = T_{1}A' $, 定义 $ T(X) = T_{1}X' $, 然后验证即可.
      \end{hint}
      \begin{answer}
          我们断言 $ A \in \MM $ 与 $ A' $ 相似, 则存在可逆阵 $ T_{1} \in \MM $, 使得 $ AT_{1} = T_{1}A' $, 那么我们定义 $ T(X) = T_{1}X' $, 先说明 $ T(X) $ 是可逆变换: 令 $ T_{2} : X \mapsto (T^{-1}_{1}X)' $, 可以验证
          \begin{align*}
              T_{2}T(X) = T_{2}(T_{1}X') = (T_{1}^{-1}T_{1}X')' = X,
          \end{align*}
          对任意 $ X \in \MM $ 都成立, 于是 $ T_{2}T = I $, 其中 $ I $ 为恒等变换, 这说明 $ T $ 为可逆变换. 下面再验证 $ L_{A}T = TR_{A} $: 对任意的 $ X \in \MM $, 都有
          \begin{align*}
              L_{A}T(X) = L_{A}(T_{1}X') = AT_{1}X' = T_{1}A'X' = T_{1}(XA)' = T(XA) = TR_{A}(X),
          \end{align*}
          这就说明了 $ L_{A}T = TR_{A} $, 即 $ L_{A} = TR_{A}T^{-1} $.

          下面再来说明为什么 $ A \in \MM $ 与 $ A' $ 相似, 首先存在可逆阵 $ P $, 使得 $ A $ 相似于它的 Jordan 标准型 $ J_{A} $ , 即
          \begin{align*}
              P^{-1}AP = \mqty(\dmat{J_{r_{1}}(\lambda_{1}), J_{r_{2}}(\lambda_{2}), \ddots, J_{r_{s}}(\lambda_{s})})
          \end{align*}
          那么就有 $ A' $ 相似于 $ J_{A}' $:
          \begin{align*}
              P'A'(P')^{-1} = \mqty(\dmat{J_{r_{1}}(\lambda_{1})', J_{r_{2}}(\lambda_{2})', \ddots, J_{r_{s}}(\lambda_{s})'})
          \end{align*}
          那么只需要说明 $ J_{r_{i}}(\lambda_{i})' $ 相似于 $ J_{r_{i}}(\lambda_{i}) $ 即可. 首先它的特征多项式为 $ (\lambda - \lambda_{i})^{r_{i}} $, 考虑
          \begin{align*}
              J_{r_{i}}(\lambda_{i})' = \begin{pmatrix}
                  \lambda_{i} & & & \\
                  1 & \lambda_{i} & & \\
                  &  \ddots & \ddots & \\
                  & & 1 & \lambda_{i}
              \end{pmatrix}_{r_{i}}
          \end{align*}
          的 $ n-1 $ 阶行列式因子: 由于 $ \det J_{r_{i}}(\lambda_{i})'\smqty(2 & 3 & \dots & r_{i} \\ 1 & 2 & \dots & r_{i}-1) = 1 $, 于是它的 $ n - 1 $ 阶行列式因子 $ D_{n-1} = 1 $, 这说明它的 $ n $ 阶行列式因子 $ D_{n} $ 和它的第 $ n $ 个不变因子 $ d_{n} $ 相同, 均为它的特征多项式 $ (\lambda - \lambda_{i})^{r_{i}} $, 那么它的初等因子组只有一个元素 $ (\lambda - \lambda_{i})^{r_{i}} $, 即它的 Jordan 标准型只有一块 $ J_{r_{i}}(\lambda_{i}) $, 于是 $ A $ 与 $ A' $ 的 Jordan 标准型相同, $ A $ 与 $ A' $ 相似.
      \end{answer}
      \item 设 $ \alpha > 0 $, $ x_{1} > 0 $, $ x_{n + 1} = \alpha(1 + x_{n})^{-1}\,(n = 1, 2, \dots) $, 证明数列 $ \set{x_{n}} $ 收敛, 并求其极限.
      \begin{hint}
          可以 ``先斩后奏''. 先假设极限存在, 并求得极限应该为 $ y = (-1 + \sqrt{1 + 4\alpha})/2 $. 再讨论 $ x_{1} $ 和 $ y $ 的大小关系来分别确定 $ \set{x_{n}} $ 的性质.
      \end{hint}
  \begin{answer}
      若 $ \set{x_{n}} $ 极限存在, 设极限为 $ x $, 那么在
      \begin{align*}
          x_{n + 1} = \frac{\alpha}{1 + x_{n}}
      \end{align*}
      两侧同时对 $ n $ 取极限, 可得 $ x^{2} + x - \alpha = 0 $. 解得
      \begin{align*}
          x = \frac{-1 + \sqrt{1 + 4\alpha}}{2} \ \ \text{或}\ \  x = \frac{-1 - \sqrt{1 + 4\alpha}}{2},
      \end{align*}
      由 $ x_{1}, \alpha > 0 $ 可知 $ x_{n} > 0 $, 那么可知, 如果 $ \set{x_{n}} $ 极限存在, 则一定为 $ x = (-1 + \sqrt{1 + 4\alpha})/2 $ , 下面讨论 $ \set{x_{n}} $ 的极限存在性.

      我们研究 $ x_{n} $ 的分布情况. 若 $ x_{n} < x $, 则
      \begin{align*}
          x_{n + 1} = \frac{\alpha}{1 + x_{n}} > \frac{\alpha}{1 + x} = x.
      \end{align*}
      而若 $ x_{n} > x $, 则
      \begin{align*}
          x_{n + 1} = \frac{\alpha}{1 + x_{n}} < \frac{\alpha}{1 + x} = x.
      \end{align*}
      这说明 $ x_{n} $ 的值在 $ x $ 左右来回跳动, 不妨设 $ x_{1} > x $, $ x_{1} < x $ 的情况同理. 那么就有 $ \set{x_{2n + 1}} > x $, $ \set{x_{2n}} < x $, 并且, 若 $ \set{x_{2n + 1}} $, $ \set{x_{2n}} $ 收敛于 $ x $, 则 $ \set{x_{n}} $ 收敛于 $ x $.
      那么可以猜想 $ \set{x_{2n}} $ 单调递增, $ \set{x_{2n + 1}} $ 单调递减, 下面考察 $ x_{n + 2} - x_{n} $ 的符号:
      \begin{align*}
          x_{n + 2} - x_{n} = \frac{\alpha}{1 + x_{n + 1}} - x_{n} = \frac{\alpha}{1 + \alpha(1 + x_{n})^{-1}} - x_{n} = \frac{\alpha - x_{n}^{2} - x_{n}}{1 + \alpha + x_{n}}
          \begin{cases}
              > 0 & , x_{n} < x\\
              < 0 & , x_{n} > x
          \end{cases},
      \end{align*}
      这说明 $ \set{x_{2n}} $ 单调递增有上界, $ \set{x_{2n + 1}} $ 单调递减有下界, 那么就可以说明
      \begin{align*}
          \limit{n}{\infty}x_{2n} = \limit{n}{\infty}x_{2n + 1} = x
      \end{align*}
      于是 $ \limit{n}{\infty}x_{n} = x $.
  \end{answer}
\end{exercise}
\stopexercise