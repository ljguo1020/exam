\documentclass{ctexart}
\IfFontExistsTF{Source Han Serif SC}{
    \setCJKmainfont{Source Han Serif SC}
}{}
\usepackage{unicode-math}

% \usepackage[mono=false]{libertinus-otf}
\setmathfont{NewCMMath-Book.otf}
\setmathfont{latinmodern-math.otf}[range=\sqrt]
% \setmathfont{TeX Gyre Pagella Math}[range=bb]


\usepackage[margin=1.5cm]{geometry}
\usepackage{mathtools}
\usepackage{physics}

% https://tex.stackexchange.com/questions/307412/redefine-pi-to-pi-with-unicode-math
\AtBeginDocument{%
  \let\umathpi\pi
  \renewcommand\pi{\symup\umathpi}%
}

\let\set\qty
% \let\vb\symbfit
\usepackage[shortlabels, inline]{enumitem} % 继承并扩展了enumerate宏包的功能
\setlist[enumerate, 1]{left=\parindent..0pt, noitemsep, itemindent=1.7\parindent, listparindent=\parindent, label=\arabic*.}
\setlist[enumerate, 2]{left=1.7\parindent..0pt, noitemsep, itemindent=2.7\parindent, listparindent=2.5\parindent, label=(\arabic*)}

\newcommand{\limit}[2]{\lim\limits_{#1 \to #2}}
\newcommand{\me}{\symrm{e}}
\newcommand{\R}{\ensuremath{\symbb{R}}}
\newcommand{\K}{\ensuremath{\symbb{K}}}
\newcommand{\N}{\ensuremath{\symbb{N}}}
\newcommand{\C}{\ensuremath{\symbb{C}}}

\NewDocumentCommand{\MM}{ O{\R} O{n} }{ \ensuremath{\symup{M}_{#2}(#1)} }
\DeclareMathOperator{\diag}{diag} 
\DeclareMathOperator{\Image}{Im}
\DeclareMathOperator{\Ker}{Ker}


\begin{document}
\begin{enumerate}
    \item 设 $ \Gamma $ 是由球面 $ x^{2} + y^{2} + z^{2} = a^{2} $ 和平面 $ x + y + z = 0 $ 交成的圆周, 从第一卦限内看 $ \Gamma $, 它的方向是逆时针. 计算第二型曲线积分
    \begin{align*}
        \int_{\Gamma} z \dd{x} + x \dd{y} + y \dd{z}.
    \end{align*}
    \item 设 $ f(x) $ 在 $ [1, +\infty) $ 上可导, 且 $ \limit{x}{+\infty}f'(x) = +\infty $, 证明 $ f(x) $ 在 $ [1, +\infty) $ 上不一致连续.
    \item 设 $ \set{a_{n}} $ 为非负递减的数列, 如果级数 $ \sum_{n=0}^{\infty}a_{n} $ 收敛, 那么 $ \limit{n}{\infty} na_{n} = 0 $. 
    \item 设 $ a, b > 0 $, $ f \in C[0, +\infty) $, 证明:
    \begin{enumerate}
        \item  如果 $ \limit{x}{+\infty}f(x) = f(+\infty) $ 存在, 那么
        \begin{align*}
            \int_{0}^{+\infty} \frac{f(ax) - f(bx)}{x} \dd{x} = (f(0) - f(+\infty))\ln\frac{b}{a}.
        \end{align*}
        \item 如果无穷积分 $ \int_{1}^{+\infty} f(x)/x \dd{x} $ 收敛, 那么
        \begin{align*}
            \int_{0}^{+\infty} \frac{f(ax) - f(bx)}{x} \dd{x} = f(0)\ln\frac{b}{a}.
        \end{align*}
        \item 如果 $ f(+\infty) $ 存在, 且积分 $ \int_{0}^{1} f(x)/x \dd{x} $ 收敛, 那么
        \begin{align*}
            \int_{0}^{+\infty} \frac{f(ax) - f(bx)}{x} \dd{x} = - f(+\infty)\ln\frac{b}{a}.
        \end{align*}
    \end{enumerate}
    \item 计算积分
    \begin{align*}
        I(r) = \int_{0}^{\pi} \ln(1 - 2 r \cos x + r^{2}) \dd{x}, \quad \abs{r} <1.
    \end{align*}
    \item 设 $ p > 0 $, 讨论积分
    \begin{align*}
        \int_{0}^{\infty} \frac{\sin(1/x)}{x^{p}} \dd{x}
    \end{align*}
    的敛散性. 
    \item 求积分 $ \int_{0}^{+\infty} \sin(x)/x \dd{x} $.
    \item 设 $ f(x) \in C[0, 1] $, 证明
    \begin{align*}
        \limit{n}{\infty} \frac{1}{n} \sum_{k=1}^{n} (-1)^{k+1} f\qty(\frac{k}{n}) = 0.
    \end{align*}
    \item 证明: 积分 
    \begin{align*}
        \int_{0}^{+\infty} \me^{-(\alpha + u^{2})t} \sin(t) \dd{t},\quad \alpha > 0
    \end{align*}
    关于 $ u $ 在 $ [0, +\infty) $ 上一致收敛.    
    \item 证明: 积分 
    \begin{align*}
        \int_{0}^{+\infty} \me^{-(\alpha + u^{2})t} \sin(t) \dd{u},\quad \alpha > 0
    \end{align*}
    关于 $ t $ 在 $ [0, +\infty) $ 上一致收敛.
    \item 计算 Fresnel 积分 $ \int_{0}^{+\infty} \sin(x^{2}) \dd{x} $.
    \item 计算积分 $ \int_{0}^{+\infty} \exp(-a x^{2})\cos(bx) \dd{x} $, 其中 $ a > 0,\ b\in \R $.
    \item 计算积分 $ \int_{0}^{\pi/2} \tan[\alpha](x) \dd{x} $, 其中 $ \abs{\alpha} < 1 $. 
    \item 设 $ \K $ 是数域, $ A \in \MM[\K][m] $, $ B \in \MM[\K][n] $, 且 $ A, B $ 没有相同的特征值, 证明矩阵方程 $ AX = XB $ 只有零解.
    \item 设 $ A $ 是 4 阶方阵, 满足 $ \tr(A^{i}) = i\,(i = 1, 2, 3, 4) $, 求 $ \abs{A} $. 
    \item $ n $ 阶方阵可对角化的充分必要条件.
    \item 设 $ f(x) $ 在 $ [0, 1] $ 上可积, 在 $ x = 1 $ 处左连续, 证明:
    \begin{align*}
        \limit{n}{\infty}\frac{\int_{0}^{1} x^{n} f(x) \dd{x}}{\int_{0}^{1} x^{n} \dd{x}} = f(1).
    \end{align*}
    \item 设 $ A \in \MM $, 若 $ A^{2} = AA' $, 证明 $ A $ 为实对称阵.
    \item 设 $ A, B \in \MM $, 若 $ A^{2} = A $, $ B^{2} = B $ 以及 $ (A + B)^{2} = A + B $, 证明 $ AB = BA = 0 $.  
    \item 设 $ A, B $ 都是 $ n $ 阶矩阵, 若 $ A^{k} = 0 $, 且 $ AB + BA = B $, 证明 $ B = 0 $. 
    \item $ A, B $ 是 $ n $ 阶方阵, $ A + B = AB $, 求证
    \begin{enumerate}
        \item $ AB = BA $,
        \item $ \rank(A) = \rank(B) $,
        \item $ A $ 可对角化当且仅当 $ B $ 可对角化.
    \end{enumerate}
    \item 设 $ f(x), g(x) $ 为多项式, 且 $ (f(x), g(x)) = 1 $, $ A $ 是 $ n $ 阶方阵, 求证: $ f(A)g(A) = 0 $ 的充分必要条件为 $ \rank(f(A)) + \rank(g(A)) = n $.
    \item 设 $ A, B $ 为实对称阵, 求证:
    \begin{enumerate}
        \item 若 $ A $ 正定, 则存在实可逆阵 $ P $ 使得 $ P'AP $ 和 $ P'BP $ 同时为对角阵;
        \item 若 $ A, B $ 半正定, 则 $ \tr(AB) \geqslant 0 $, 并且等号成立当且仅当 $ AB = 0 $.
    \end{enumerate}
    \item $ A, B, C \in \MM $, 并且 $ A = B + C $, 其中 $ B $ 为对称阵, $ C $ 为反对称阵, 证明: 若 $ A^{2} = 0 $, 则 $ A = 0 $.
    \item 求极限 $ \limit{n}{\infty} \sin[2](\pi \sqrt{n^{2} + n}) $.  
    \item $ f(x) $ 在 $ [a, b] $ 上二阶可导, 证明存在 $ \xi \in (a, b) $, 使得
    \begin{align*}
        f(b) - 2f\qty(\frac{a + b}{2}) + f(a) = \frac{1}{4} (b - a)^{2} f''(\xi),
    \end{align*}
    \item 设 $ f(x) $ 在 $ [a, b] $ 上二阶可导, 且 $ f(a) = f(b)  = 0 $, 证明对每个 $ x \in (a, b) $, 都存在对应的 $ \xi \in (a, b) $, 使得
    \begin{align*}
        f(x) = \frac{f''(\xi)}{2} (x - a) (x - b).
    \end{align*}
    \item 设 $ f(x) $ 在 $ [a, b] $ 上三阶可导, 证明存在 $ \xi \in (a, b) $, 使得
    \begin{align*}
        f(b) = f(a) + \frac{1}{2} (b - a) [f'(a) + f'(b)] - \frac{1}{12} (b - a)^{3} f'''(\xi).
    \end{align*}
    \item $ f(x) $ 在 $ [0, +\infty) $ 非负连续, 单调递减, 求证 $ \set{a_{n}} $ 极限存在, 其中
    \begin{align*}
        a_{n} = \sum_{k = 1}^{n} f(k) - \int_{0}^{n} f(x) \dd{x}.
    \end{align*}
    \item 求 $ \limit{n}{\infty} n \qty(\pi/4 - x_{n}) $, 其中:
    \begin{align*}
        x_{n} = \frac{n}{n^{2} + 1} + \frac{n}{n^{2} + 2^{2}} + \dots + \frac{n}{n^{2} + n^{2}}.
    \end{align*}
    \item 如果级数 $ \sum_{n = 1}^{\infty} a_{n} $ 收敛, $ \limit{n}{\infty} p_{n} = \infty $, 证明极限
    \begin{align*}
        \limit{n}{\infty}\frac{a_{1} p_{1} + a_{2} p_{2} + \dots + a_{n} p_{n}}{p_{n}} = 0.
    \end{align*}
    \item 如果级数 $ \sum_{n = 1}^{\infty} a_{n} $ 收敛, 证明极限
    \begin{align*}
        \limit{n}{\infty}\qty(n! a_{1} a_{2} \dots a_{n})^{1/n} = 0.
    \end{align*}
    \item 面积原理
    \begin{enumerate}
        \item 设 $ x \geqslant m \in \N^{*} $, $ f $ 是一个非负的递增函数, 则当 $ \xi \geqslant m $ 时有
        \begin{align*}
            \abs{\sum_{k = m}^{[\xi]} f(k) - \int_{m}^{\xi} f(x) \dd{x}} \leqslant f(\xi).
        \end{align*}
        \item 设 $ x \geqslant m \in \N^{*} $, $ f $ 是一个非负的递减函数, 则极限
        \begin{align*}
            \limit{n}{\infty} \qty(\sum_{k = m}^{[\xi]} f(k) - \int_{m}^{\xi} f(x) \dd{x}) = \alpha
        \end{align*}
        存在, 且 $ 0 \leqslant \alpha \leqslant f(m) $. 更进一步, 如果 $ \limit{x}{+\infty} f(x) = 0 $, 那么
        \begin{align*}
            \abs{\sum_{k = m}^{[\xi]} f(k) - \int_{m}^{\xi} f(x) \dd{x} - \alpha} \leqslant f(\xi - 1),
        \end{align*} 
        这里 $ \xi \geqslant m + 1 $. 
    \end{enumerate}
    \item 设 $ f(x) $ 在 $ [a, b] $ 上二次可微, 且 $ f(a)f(b) < 0 $, 对任意 $ x \in [a, b] $ 都有 $ f'(x) > 0 $, $ f''(x) > 0 $. 证明序列 $ \set{x_{n}} $ 极限存在, 其中 $ x_{1} \in [a, b] $, $ x_{n + 1} = x_{n} - f(x_{n})/f'(x_{n})\,(n = 1, 2, \dots) $, 进而可以证明此极限为方程 $ f(x) = 0 $ 的根. 
    \item 设正项级数 $ \sum_{n = 1}^{\infty} a_{n} $ 收敛, 数列 $ \set{y_{n}} : y_{1} = 1,\ 2y_{n + 1} = y_{n} + \sqrt{y_{n}^{2} + a_{n}} $\, $ (n = 1, 2, \dots) $. 证明 $ \set{y_{n}} $ 是单调递增的收敛数列. 
    \item 设数列 $ \set{x_{n}} $ 满足: 当 $ n < m $ 时, $ \abs{x_{n} - x_{m}} > 1/n $. 证明数列 $ \set{x_{n}} $ 无界.
    \item 设 $ f(x) $ 在闭区间 $ [0, 1] $ 上具有二阶导数, 且 $ f(0) = f'(0) = f(1) = 0 $, 证明: 存在 $ \xi \in (0, 1) $, 使得 $ f''(\xi) = f(\xi) $.
    \item $ n $ 阶方阵的每行之和与每列之和均为 0, 证明其所有代数余子式全相等.
    \item 设函数 $ f(x) $ 定义在 $ (a, +\infty) $, 且 $ f(x) $ 在每个有限区间 $ (a, b) $ 内都有界, 并满足
    \begin{align*}
        \limit{x}{+\infty} \qty(f(x + 1) - f(x)) = A.
    \end{align*}
    证明 $ \limit{x}{+\infty} (f(x) / x) = A $. 
    \item 证明: \begin{enumerate*}[(1)]
        \item 关于 $ x $ 的方程 $ \sum_{k=1}^{n} \me^{kx} = n + 1 $ 在 $ (0, 1) $ 上存在唯一的实根 $ a_{n} $;
        \item 数列 $ \set{a_{n}} $ 收敛, 并求其极限.
    \end{enumerate*}
    \item 设 $ a > 0 $, 求积分
    \begin{align*}
        \int_{0}^{\pi/2} \frac{1}{\sqrt{x}} \dd{x} \int_{\sqrt{x}}^{\sqrt{\pi/2}} \frac{1}{1 + \tan^{a}y^{2}} \dd{y}
    \end{align*}
    \item $ \alpha, \beta $ 是 $ n $ 维列向量, $ A $ 是 $ n $ 阶方阵, 求证: $ \abs{A + \alpha\beta'} = \abs{A} + \beta'A^{*}\alpha $ . 
    \item 设 $ A \in \MM[\R][3\times 2] $, $ B \in \MM[\R][2 \times 3] $, 且
    \begin{align*}
        AB = \begin{pmatrix}
            8 & 2 & -2 \\
            2 & 5 & 4 \\
            -2 & 4 & 5
        \end{pmatrix},
    \end{align*}
    求证 $ BA = 9 I $. 
    \item $ A \in \MM $, $ A^{2} = A $, 若对任意列向量 $ x $, 都有 $ x'A'Ax \leqslant x'x $, 证明 $ A' = A $.    
    \item 证明对任意 $ m \times n $ 矩阵 $ A $, 都有 $ \rank(AA') = \rank(A) $.  
    \item (极分解) 对可逆阵 $ A \in \MM $, 证明
    \begin{enumerate}
        \item 存在正交阵 $ P $, 正定阵 $ B $, 满足 $ A = PB $.
        \item 存在正交阵 $ Q_{1}, Q_{2} $, 使得 $ Q_{1} A Q_{2} = \diag(\lambda_{1}, \lambda_{2}, \dots, \lambda_{n}) $, 并且 $ \lambda_{1}^{2}, \lambda_{2}^{2}, \dots, \lambda_{n}^{2} $ 是 $ A'A $ 的特征值. 
    \end{enumerate}
    \item $ f(x) \in C[a, b] $, 证明函数 $ m(x) = \min\limits_{a \leqslant \xi \leqslant x}f(\xi) $ 在 $ [a, b] $ 连续.
    \item $ f(x) $ 在 $ (0, +\infty) $ 上二阶可导, 且 $ \limit{x}{+\infty} f(x) $ 存在, $ f''(x) $ 有界, 证明 $ \limit{x}{+\infty} f'(x) = 0 $.
    \item $ f(x) $ 在 $ \R $ 上三阶连续可导, 且对任意的 $ h > 0 $, 有
    \begin{align*}
        \frac{f(x + h) - f(x)}{h} = f'\qty(x + \frac{h}{2})
    \end{align*}
    求证: $ f(x) $ 为次数至多为 2 的多项式.
    \item 设 $ A' = A $, 证明 $ A $ 可逆当且仅当存在矩阵 $ B $ 使得 $ AB + B'A $ 正定.
    \item 设 $ f(x) $ 在 $ [a, b] $ 上可微, $ f(a) = 0 $, 并且存在实数 $ A > 0 $, 使得对任意 $ x \in [a, b] $, 都有 $ \abs{f'(x)} \leqslant A \abs{f(x)} $, 证明在 $ [a, b] $ 上, $ f(x) \equiv 0 $. 
    \item 设 $ f(x) $ 在 $ [1, +\infty) $ 上一阶连续可导, 且
    \begin{align*}
        f'(x) = \frac{1}{1 + f^{2}(x)}\qty(\frac{1}{\sqrt{x}} - \sqrt{\ln\qty(1 + \frac{1}{x})})
    \end{align*}
    证明: $ \limit{x}{+\infty}f(x) $ 存在.
    \item 设 $ f(x) $ 在 $ [0, +\infty) $ 上一致连续, 若对于任意 $ x \in \R $, 都有 $ \limit{n}{+\infty} f(x + n) = 0 $, 证明 $ \limit{x}{+\infty} f(x) = 0 $.  
    \item 设 $ f(x) $ 在 $ [0, +\infty) $ 上一致连续, 且对任意 $ \delta > 0 $, 都有 $ \limit{n}{\infty} f(n\delta) = 0 $, 证明 $ \limit{x}{+\infty} f(x) = 0 $.  
    \item 设 $ f(x) $ 在 $ \R $ 上一致连续, 则存在正实数 $ a, b $, 使得 $ \abs{f(x)} \leqslant a\abs{x} + b $. 
    \item 设 $ f(x) $ 在 $ [1, +\infty) $ 上一致连续, 证明 $ \abs{f(x)/x} $ 在 $ [1, +\infty) $ 有界. 
    \item 证明欧式空间中两标准正交基的过渡矩阵为正交阵.
    \item 设 $ \alpha $ 是欧式空间 $ V $ 中的一个非零向量, $ \alpha_{1}, \alpha_{2}, \dots, \alpha_{p} $ 是 $ V $ 中的 $ p $ 个向量, 满足
    \begin{align*}
        (a_{i}, a_{j}) \leqslant 0,\ (\alpha_{i}, \alpha) > 0, \quad i, j = 1, 2, \dots, p, i \ne j
    \end{align*}
    证明
    \begin{enumerate}
        \item $ \alpha_{1}, \alpha_{2}, \dots, \alpha_{p} $ 线性无关;
        \item $ n $ 维欧式空间中最多有 $ n + 1 $ 个向量, 使其两两互成钝角;
        \item $ n $ 维欧式空间中一定存在 $ n + 1 $ 个向量, 使其两两互为钝角.
    \end{enumerate}
    \item 设 $ A, B \in \MM[\K] $, 且 $ AB = BA $, 利用线性方程组的知识证明
    \begin{align*}
        \rank(A + B) \leqslant \rank(A) + \rank(B) - \rank(AB)
    \end{align*}
    \item 设 $ B \in \MM[\C][n \times 2] $, 
    \begin{align*}
        C = \begin{pmatrix}
            1 & 1 & \dots & 1\\
            1 & 2 & \dots & n
        \end{pmatrix}
    \end{align*}
    若 $ A = BC $, 且 $ CB $ 的特征多项式为 $ x^{2} - 2x + 1 $, 求 $ A $ 的特征值, 并求 $ AX = 0 $ 的基础解系. 
    \item  计算 $ n $ 阶 $ b $ -- 循环行列式:
    \begin{align*}
        A = \begin{vmatrix}
            a_{1}   & a_{2} & a_{3} & \ldots    & a_{n} \\
            ba_{n}  & a_{1} & a_{2} & \ldots    & a_{n-1} \\
            ba_{n-1}    & ba_{n}    & a_{1} & \ldots & a_{n-1} \\
            \vdots  & \vdots & \vdots & \ddots & \vdots \\
            ba_{2} & ba_{3} & ba_{4} & \ldots & a_{1}
        \end{vmatrix}
    \end{align*}
    \item 设 $ A $ 是 $ n $ 阶实反对称阵, $ D = \diag\qty{d_{1}, d_{2}, \dots, d_{n}} $ 是同阶的对角阵, 且 $ d_{i} > 0\,(i = 1, 2, \dots, n) $. 求证 $ \abs{A + D} > 0 $, 特别地, $ I_{n} + A $ 与 $ I_{n} - A $ 都是非异阵. 
    \item 如果 $ n $ 阶方阵 $ A = (a_{ij}) $ 适合条件:
    \begin{align*}
        \abs{a_{ii}} > \sum_{\mathclap{j = 1,\ j \ne i}}^{n} \abs{a_{ij}}, \quad i = 1, 2, \dots, n,
    \end{align*}
    则称 $ A $ 为\textbf{严格对角占优阵}, 求证, 严格对角占优阵必是满秩阵, 若上述条件改为:
    \begin{align*}
        a_{ii} > \sum_{\mathclap{j = 1,\ j \ne i}}^{n} \abs{a_{ij}}, \quad i = 1, 2, \dots, n,
    \end{align*}   
    求证 $ \abs{A} > 0 $. 
    \item 设 $ f(x) $ 在 $ [a, b] $ 上有定义, 对 $ [a, b] $ 上任意一个闭区间 $ [x_{1}, x_{2}] \subset [a, b] $, 对介于 $ f(x_{1}) $ 与 $ f(x_{2}) $ 之间的任一常数 $ l $, 方程 
    \begin{align*}
        f(x) = l
    \end{align*}
    在 $ [x_{1}, x_{2}] $ 上有且仅有有限个解, 证明 $ f(x) \in C[a, b] $.
    \item 设 $ f(x) $ 在 $ (0, +\infty) $ 上可导, 且 $ \limit{x}{+\infty} \qty(f(x) + f'(x)) = A $, 证明 $ \limit{x}{+\infty} f(x) = A $, 其中 $ A \in \R\cup{\pm\infty} $. 
    \item 已知 $ A \in \MM[\K] $, 且 $ \tr(A) = 0 $, 证明
    \begin{enumerate}
        \item 存在数域 $ \K $ 上的可逆阵 $ C $, 使得 $ C^{-1}AC $ 为主对角元全为 $ 0 $ 的矩阵.
        \item 存在 $ X, Y \in \MM[\K] $, 使得 $ XY - YX = A $.
        \item 令 $ U $ 为 $ \MM[\K] $ 中所有形如 $ XY - YX $ 的矩阵组成的集合, 证明 $ U $ 是 $ \MM[\K] $ 的一个线性子空间. 
    \end{enumerate}
    \item 求极限
    \begin{align*}
        \limit{n}{\infty} \qty(\frac{1}{\sqrt{n^{2} + 1}} + \frac{1}{\sqrt{n^{2} + 2}} + \dots + \frac{1}{\sqrt{n^{2} + n}})^{n}
    \end{align*}
    \item 设 $ \varphi $ 为 $ n $ 维线性空间 $ V $ 上的线性变换, $ W $ 是 $ \varphi $ 的不变子空间, 且 $ V = \Image \varphi \oplus W $, 证明
    \begin{align*}
        V = \Image \varphi \oplus \Ker \varphi.
    \end{align*}
    \item 设 $ A, B \in \MM[\C] $, 且 $ \rank(A) = \rank(B)  = 1,\ \tr(A) = \tr(B) $, 证明 $ A $ 相似于 $ B $.
    \item 设 $ \varphi $ 是 $ n $ 维线性空间 $ V $ 上的线性变换, 求证: 必存在正整数 $ m $, 使得
    \begin{align*}
        \Image \varphi^{m} = \Image \varphi^{m+1},\quad \Ker \varphi^{m} = \Ker \varphi^{m+1}, \quad V = \Image \varphi^{m} \oplus \Ker \varphi^{m+1}.
    \end{align*}
    \item 使用 Jordan 标准型证明迹非 $ 0 $ 的秩 1 矩阵可对角化.
    \item 设 $ A $ 是 $ n $ 阶实对称阵, 证明: $ A $ 可逆的充分必要条件为存在矩阵 $ B $, 使得 $ AB + B'A $ 正定.
    \item 设 $ A, B \in \MM[\C] $, 其中 $ A $ 是幂零阵, 且 $ AB = BA $, 求证: $ \abs{B} = \abs{A + B} $. 
    \item 设函数 $ f $ 在 $ x = 0 $ 连续, 并且
    \begin{align*}
        \limit{x}{0}\frac{f(2x) - f(x)}{x} = A,
    \end{align*}
    求证: $ f'(0) $ 存在, 且 $ f'(0) = A $. 
    \item 设 $ x_{n} $ 是 $ \tan x = x $ 在 $ (n\pi, n\pi + \pi/2) $ 上的解,
    \begin{enumerate}
        \item 求证 $ \limit{n}{\infty}(n\pi + \pi/2 - x_{n}) = 0 $,
        \item 求 $ \limit{n}{\infty} n(n\pi + \pi/2 - x_{n}) $.  
    \end{enumerate}
    \item 设 $ f $ 在 $ [0, +\infty) $ 上可微, 且 $ f(0) = 0 $, 并假设有实数 $ A $ 使得 $ \abs{f'(x)} \leqslant A\abs{f(x)} $ 对 $ x \in (0, +\infty) $ 恒成立, 证明 $ f(x) \equiv 0\,(x \in (0, +\infty)) $.  
    \item 设偶函数 $ f(x) $ 在 $ x = 0 $ 处二阶可导, 且 $ f(0) = 1 $, 证明级数 $ \sum_{n = 1}^{\infty} (f(1/n) - 1) $ 绝对收敛.
    \item 设 $ f $ 在 $ [a, b] $ 上可导, 且 $ f' $ 在 $ [a, b] $ 上可积, $ f(a) = 0 $, 证明:
    \begin{align*}
        2\int_{a}^{b}(f(x))^{2} \dd{x} \leqslant (b-a)^{2} \int_{a}^{b} (f'(x))^{2} \dd{x}.
    \end{align*}
    \item 设 $ f(x) $ 在 $ [0, +\infty) $ 上可微, 且存在实数 $ A > 0 $, 使得 $ \abs{f'(x)} \leqslant A\abs{f(x)} $, 证明 $ f(x) \equiv 0 $ 对 $ x \in [0, +\infty) $ 均成立.
\end{enumerate}
\end{document}
