
\documentclass{ctexart}
\IfFontExistsTF{Source Han Serif SC}{
    \setCJKmainfont{Source Han Serif SC}
}{}

% \usepackage{showframe}

% \ctexset{
%     section/format  =   {\normalsize\raggedright\bfseries}
% }

\xeCJKsetup{CheckSingle=true}
\usepackage{tcolorbox}
\tcbuselibrary{breakable}
\tcbuselibrary{skins}
\tcbset{
    width   =   \linewidth,
    fonttitle=  \bfseries,
    breakable,
    enhanced jigsaw
    % lines before break=0
}
\usepackage{mathtools}
\usepackage{unicode-math}
\usepackage[mono=false]{libertinus-otf}
% \setmainfont{TeX Gyre Pagella}
% \setmathfont{TeX Gyre Pagella Math}
% \setmathfont{NewCMMath-Book.otf}
% \setmathfont{latinmodern-math.otf}[range=\sqrt]
% \setmathfont{TeX Gyre Pagella Math}[range=bb]


\usepackage[margin=1.5cm]{geometry}

\usepackage[notrig]{physics}
\usepackage{nicematrix}
\usepackage[colorlinks=true]{hyperref}

% https://tex.stackexchange.com/questions/307412/redefine-pi-to-pi-with-unicode-math
\AtBeginDocument{%
  \let\umathpi\pi
  \renewcommand\pi{\symup\umathpi}%
}

\let\set\qty
\let\emph\textbf
\AtBeginDocument{
    \let\ge\geqslant
    \let\le\leqslant
}

\def\grad{\symbf{\nabla}}
\let\boldsymbol\symbfit

% \let\vb\symbfit
\usepackage[shortlabels, inline]{enumitem} % 继承并扩展了enumerate宏包的功能
\setlist[enumerate, 1]{leftmargin=*, label=\arabic*., widest=111, ref=\arabic*}
\setlist[enumerate, 2]{leftmargin=*, label=(\arabic*)}

% https://tex.stackexchange.com/a/52718/180617
\def\asteriskitem{*}
\makeatletter
\def\sitem{%
  \expandafter\let\expandafter\originallabel\csname labelenum\romannumeral\@enumdepth\endcsname
  \expandafter\def\csname labelenum\romannumeral\@enumdepth\expandafter\endcsname\expandafter{%
    \expandafter\bfseries\expandafter\color{red}\expandafter{\expandafter\asteriskitem\expandafter\originallabel}}%
  \item
  \expandafter\let\csname labelenum\romannumeral\@enumdepth\endcsname\originallabel
}
\makeatother

\newcommand{\limit}[2]{\lim_{#1 \to #2}}
\newcommand{\me}{\symrm{e}}
\newcommand{\R}{\ensuremath{\symbb{R}}}
\newcommand{\K}{\ensuremath{\symbb{K}}}
\newcommand{\N}{\ensuremath{\symbb{N}}}
\newcommand{\C}{\ensuremath{\symbb{C}}}
\newcommand{\Q}{\ensuremath{\symbb{Q}}}

\NewDocumentCommand{\MM}{ O{\R} O{n} }{ \ensuremath{\symup{M}_{#2}(#1)} }
\DeclareMathOperator{\diag}{diag} 
% \DeclareMathOperator{\deg}{deg} 
\DeclareMathOperator{\Image}{Im}
\DeclareMathOperator{\Ker}{Ker}

\renewcommand{\thempfootnote}{\arabic{mpfootnote}}

\begin{document}
\begin{enumerate}[series=exer]
    \item 设 $ \Gamma $ 是由球面 $ x^{2} + y^{2} + z^{2} = a^{2} $ 和平面 $ x + y + z = 0 $ 交成的圆周, 从第一卦限内看 $ \Gamma $, 它的方向是逆时针. 计算第二型曲线积分
    \begin{align*}
        \int_{\Gamma} z \dd{x} + x \dd{y} + y \dd{z}.
    \end{align*}
    \item 设 $ f(x) $ 在 $ [1, +\infty) $ 上可导, 且 $ \limit{x}{+\infty}f'(x) = +\infty $, 证明 $ f(x) $ 在 $ [1, +\infty) $ 上不一致连续.
    \item 设 $ \set{a_{n}} $ 为非负递减的数列, 如果级数 $ \sum_{n=0}^{\infty}a_{n} $ 收敛, 那么 $ \limit{n}{\infty} na_{n} = 0 $. 
    \item 设 $ a, b > 0 $, $ f \in C[0, +\infty) $, 证明:
    \begin{enumerate}
        \item  如果 $ \limit{x}{+\infty}f(x) = f(+\infty) $ 存在, 那么
        \begin{align*}
            \int_{0}^{+\infty} \frac{f(ax) - f(bx)}{x} \dd{x} = \qty(f(0) - f(+\infty))\ln\frac{b}{a}.
        \end{align*}
        \item 如果无穷积分 $ \int_{1}^{+\infty} f(x)/x \dd{x} $ 收敛, 那么
        \begin{align*}
            \int_{0}^{+\infty} \frac{f(ax) - f(bx)}{x} \dd{x} = f(0)\ln\frac{b}{a}.
        \end{align*}
        \item 如果 $ f(+\infty) $ 存在, 且积分 $ \int_{0}^{1} f(x)/x \dd{x} $ 收敛, 那么
        \begin{align*}
            \int_{0}^{+\infty} \frac{f(ax) - f(bx)}{x} \dd{x} = - f(+\infty)\ln\frac{b}{a}.
        \end{align*}
    \end{enumerate}
    \item 计算积分
    \begin{align*}
        I(r) = \int_{0}^{\pi} \ln(1 - 2 r \cos x + r^{2}) \dd{x}, \quad \abs{r} <1.
    \end{align*}
    \item 设 $ p > 0 $, 讨论积分
    \begin{align*}
        \int_{0}^{+\infty} \frac{\sin(1/x)}{x^{p}} \dd{x}
    \end{align*}
    的敛散性. 
    \item 求积分 $ \int_{0}^{+\infty} \sin(x)/x \dd{x} $.
    \item 设 $ f(x) \in C[0, 1] $, 证明
    \begin{align*}
        \limit{n}{\infty} \frac{1}{n} \sum_{k=1}^{n} (-1)^{k+1} f\qty(\frac{k}{n}) = 0.
    \end{align*}
    \item 证明: 积分 
    \begin{align*}
        \int_{0}^{+\infty} \me^{-(\alpha + u^{2})t} \sin(t) \dd{t},\quad \alpha > 0
    \end{align*}
    关于 $ u $ 在 $ [0, +\infty) $ 上一致收敛.    
    \item 证明: 积分 
    \begin{align*}
        \int_{0}^{+\infty} \me^{-(\alpha + u^{2})t} \sin(t) \dd{u},\quad \alpha > 0
    \end{align*}
    关于 $ t $ 在 $ [0, +\infty) $ 上一致收敛.
    \item 计算 Fresnel 积分 $ \int_{0}^{+\infty} \sin(x^{2}) \dd{x} $.
    \item 计算积分 $ \int_{0}^{+\infty} \exp(-a x^{2})\cos(bx) \dd{x} $, 其中 $ a > 0,\ b\in \R $.
    \item 计算积分 $ \int_{0}^{\pi/2} \tan^{\alpha}(x) \dd{x} $, 其中 $ \abs{\alpha} < 1 $. 
    \sitem\label{item:AX=XB} 设 $ \K $ 是数域, $ A \in \MM[\K][m] $, $ B \in \MM[\K][n] $, 且 $ A, B $ 没有相同的特征值, 证明矩阵方程 $ AX = XB $ 只有零解.
    \item 设 $ A $ 是 4 阶方阵, 满足 $ \tr(A^{i}) = i\,(i = 1, 2, 3, 4) $, 求 $ \abs{A} $. 
    \item $ n $ 阶方阵可对角化的充分必要条件.
    \item 设 $ f(x) $ 在 $ [0, 1] $ 上可积, 在 $ x = 1 $ 处左连续, 证明:
    \begin{align*}
        \limit{n}{\infty}\frac{\int_{0}^{1} x^{n} f(x) \dd{x}}{\int_{0}^{1} x^{n} \dd{x}} = f(1).
    \end{align*}
    \item 设 $ A \in \MM $, 若 $ A^{2} = AA' $, 证明 $ A $ 为实对称阵.
    \item 设 $ A, B \in \MM $, 若 $ A^{2} = A $, $ B^{2} = B $ 以及 $ (A + B)^{2} = A + B $, 证明 $ AB = BA = 0 $.  
    \item 设 $ A, B $ 都是 $ n $ 阶矩阵, 若 $ A^{k} = 0 $, 且 $ AB + BA = B $, 证明 $ B = 0 $. 
    \item $ A, B $ 是 $ n $ 阶方阵, $ A + B = AB $, 求证
    \begin{enumerate}
        \item $ AB = BA $,
        \item $ \rank(A) = \rank(B) $,
        \item $ A $ 可对角化当且仅当 $ B $ 可对角化.
    \end{enumerate}
    \sitem 设 $ f(x), g(x) $ 为多项式, 且 $ (f(x), g(x)) = 1 $, $ A $ 是 $ n $ 阶方阵, 求证: $ f(A)g(A) = 0 $ 的充分必要条件为 $ \rank(f(A)) + \rank(g(A)) = n $.
    \item 设 $ A, B $ 为实对称阵, 求证:
    \begin{enumerate}
        \item 若 $ A $ 正定, 则存在实可逆阵 $ P $ 使得 $ P'AP $ 和 $ P'BP $ 同时为对角阵;
        \item 若 $ A, B $ 半正定, 则 $ \tr(AB) \ge 0 $, 并且等号成立当且仅当 $ AB = 0 $.
    \end{enumerate}
    \item $ A, B, C \in \MM $, 并且 $ A = B + C $, 其中 $ B $ 为对称阵, $ C $ 为反对称阵, 证明: 若 $ A^{2} = 0 $, 则 $ A = 0 $.
    \item 求极限 $ \limit{n}{\infty} \sin^{2}\qty(\pi \sqrt{n^{2} + n}) $.  
    \item $ f(x) $ 在 $ [a, b] $ 上二阶可导, 证明存在 $ \xi \in (a, b) $, 使得
    \begin{align*}
        f(b) - 2f\qty(\frac{a + b}{2}) + f(a) = \frac{1}{4} (b - a)^{2} f''(\xi),
    \end{align*}
    \item 设 $ f(x) $ 在 $ [a, b] $ 上二阶可导, 且 $ f(a) = f(b)  = 0 $, 证明对每个 $ x \in (a, b) $, 都存在对应的 $ \xi \in (a, b) $, 使得
    \begin{align*}
        f(x) = \frac{f''(\xi)}{2} (x - a) (x - b).
    \end{align*}
    \item 设 $ f(x) $ 在 $ [a, b] $ 上三阶可导, 证明存在 $ \xi \in (a, b) $, 使得
    \begin{align*}
        f(b) = f(a) + \frac{1}{2} (b - a) [f'(a) + f'(b)] - \frac{1}{12} (b - a)^{3} f'''(\xi).
    \end{align*}
    \item $ f(x) $ 在 $ [0, +\infty) $ 非负连续, 单调递减, 求证 $ \set{a_{n}} $ 极限存在, 其中
    \begin{align*}
        a_{n} = \sum_{k = 1}^{n} f(k) - \int_{0}^{n} f(x) \dd{x}.
    \end{align*}
    \item 求 $ \limit{n}{\infty} n \qty(\pi/4 - x_{n}) $, 其中:
    \begin{align*}
        x_{n} = \frac{n}{n^{2} + 1} + \frac{n}{n^{2} + 2^{2}} + \dots + \frac{n}{n^{2} + n^{2}}.
    \end{align*}
    \item 如果级数 $ \sum_{n = 1}^{\infty} a_{n} $ 收敛, $ \limit{n}{\infty} p_{n} = \infty $, 证明极限
    \begin{align*}
        \limit{n}{\infty}\frac{a_{1} p_{1} + a_{2} p_{2} + \dots + a_{n} p_{n}}{p_{n}} = 0.
    \end{align*}
    \item 如果级数 $ \sum_{n = 1}^{\infty} a_{n} $ 收敛, 证明极限
    \begin{align*}
        \limit{n}{\infty}\qty(n! a_{1} a_{2} \dots a_{n})^{1/n} = 0.
    \end{align*}
    \item \textbf{面积原理}
    \begin{enumerate}
        \item 设 $ f $ 是一个非负的递增函数, 则当 $ \xi \ge m $ 时有
        \begin{align*}
            \abs{\sum_{k = m}^{[\xi]} f(k) - \int_{m}^{\xi} f(x) \dd{x}} \le f(\xi).
        \end{align*}
        \item 设 $ f $ 是一个非负的递减函数, 则极限
        \begin{align*}
            \limit{\xi}{\infty} \qty(\sum_{k = m}^{[\xi]} f(k) - \int_{m}^{\xi} f(x) \dd{x}) = \alpha
        \end{align*}
        存在, 且 $ 0 \le \alpha \le f(m) $. 更进一步, 如果 $ \limit{x}{+\infty} f(x) = 0 $, 那么
        \begin{align*}
            \abs{\sum_{k = m}^{[\xi]} f(k) - \int_{m}^{\xi} f(x) \dd{x} - \alpha} \le f(\xi - 1),
        \end{align*} 
        这里 $ \xi \ge m + 1 $. 
    \end{enumerate}
    \item 设 $ f(x) $ 在 $ [a, b] $ 上二次可微, 且 $ f(a)f(b) < 0 $, 对任意 $ x \in [a, b] $ 都有 $ f'(x) > 0 $, $ f''(x) > 0 $. 证明序列 $ \set{x_{n}} $ 极限存在, 其中 $ x_{1} \in [a, b] $, $ x_{n + 1} = x_{n} - f(x_{n})/f'(x_{n})\,(n = 1, 2, \dots) $, 进而可以证明此极限为方程 $ f(x) = 0 $ 的根. 
    \item 设正项级数 $ \sum_{n = 1}^{\infty} a_{n} $ 收敛, 数列 $ \set{y_{n}} : y_{1} = 1,\ 2y_{n + 1} = y_{n} + \sqrt{y_{n}^{2} + a_{n}} $\, $ (n = 1, 2, \dots) $. 证明 $ \set{y_{n}} $ 是单调递增的收敛数列. 
    \item 设数列 $ \set{x_{n}} $ 满足: 当 $ n < m $ 时, $ \abs{x_{n} - x_{m}} > 1/n $. 证明数列 $ \set{x_{n}} $ 无界.
    \item 设 $ f(x) $ 在闭区间 $ [0, 1] $ 上具有二阶导数, 且 $ f(0) = f'(0) = f(1) = 0 $, 证明: 存在 $ \xi \in (0, 1) $, 使得 $ f''(\xi) = f(\xi) $.
    \item $ n $ 阶方阵的每行之和与每列之和均为 0, 证明其所有代数余子式全相等.
    \item 设函数 $ f(x) $ 定义在 $ (a, +\infty) $, 且 $ f(x) $ 在每个有限区间 $ (a, b) $ 内都有界, 并满足
    \begin{align*}
        \limit{x}{+\infty} \qty(f(x + 1) - f(x)) = A.
    \end{align*}
    证明 $ \limit{x}{+\infty} (f(x) / x) = A $. 
    \item 证明: \begin{enumerate*}[(1)]
        \item 关于 $ x $ 的方程 $ \sum_{k=1}^{n} \me^{kx} = n + 1 $ 在 $ (0, 1) $ 上存在唯一的实根 $ a_{n} $;
        \item 数列 $ \set{a_{n}} $ 收敛, 并求其极限.
    \end{enumerate*}
    \item 设 $ a > 0 $, 求积分
    \begin{align*}
        \int_{0}^{\pi/2} \frac{1}{\sqrt{x}} \dd{x} \int_{\sqrt{x}}^{\sqrt{\pi/2}} \frac{1}{1 + \tan^{a}y^{2}} \dd{y}
    \end{align*}
    \item $ \alpha, \beta $ 是 $ n $ 维列向量, $ A $ 是 $ n $ 阶方阵, 求证: $ \abs{A + \alpha\beta'} = \abs{A} + \beta'A^{*}\alpha $ . 
    \item 设 $ A \in \MM[\R][3\times 2] $, $ B \in \MM[\R][2 \times 3] $, 且
    \begin{align*}
        AB = \begin{pmatrix}
            8 & 2 & -2 \\
            2 & 5 & 4 \\
            -2 & 4 & 5
        \end{pmatrix},
    \end{align*}
    求证 $ BA = 9 I $. 
    \item $ A \in \MM $, $ A^{2} = A $, 若对任意列向量 $ x $, 都有 $ x'A'Ax \le x'x $, 证明 $ A' = A $.    
    \item 证明对任意 $ m \times n $ 矩阵 $ A $, 都有 $ \rank(AA') = \rank(A) $.  
    \item $ f(x) \in C[a, b] $, 证明函数 $ m(x) = \min\limits_{a \le \xi \le x}f(\xi) $ 在 $ [a, b] $ 连续.
    \item $ f(x) $ 在 $ (0, +\infty) $ 上二阶可导, 且 $ \limit{x}{+\infty} f(x) $ 存在, $ f''(x) $ 有界, 证明 $ \limit{x}{+\infty} f'(x) = 0 $.
    \item $ f(x) $ 在 $ \R $ 上三阶连续可导, 且对任意的 $ h > 0 $, 有
    \begin{align*}
        \frac{f(x + h) - f(x)}{h} = f'\qty(x + \frac{h}{2})
    \end{align*}
    求证: $ f(x) $ 为次数至多为 2 的多项式.
    \item 设 $ A' = A $, 证明 $ A $ 可逆当且仅当存在矩阵 $ B $ 使得 $ AB + B'A $ 正定.
    \item 设 $ f(x) $ 在 $ [a, b] $ 上可微, $ f(a) = 0 $, 并且存在实数 $ A > 0 $, 使得对任意 $ x \in [a, b] $, 都有 $ \abs{f'(x)} \le A \abs{f(x)} $, 证明在 $ [a, b] $ 上, $ f(x) \equiv 0 $. 
    \item 设 $ f(x) $ 在 $ [1, +\infty) $ 上一阶连续可导, 且
    \begin{align*}
        f'(x) = \frac{1}{1 + f^{2}(x)}\qty(\frac{1}{\sqrt{x}} - \sqrt{\ln\qty(1 + \frac{1}{x})})
    \end{align*}
    证明: $ \limit{x}{+\infty}f(x) $ 存在.
    \item 设 $ f(x) $ 在 $ [0, +\infty) $ 上一致连续, 若对于任意 $ x \in \R $, 都有 $ \limit{n}{+\infty} f(x + n) = 0 $, 证明 $ \limit{x}{+\infty} f(x) = 0 $.  
    \item 设 $ f(x) $ 在 $ [0, +\infty) $ 上一致连续, 且对任意 $ \delta > 0 $, 都有 $ \limit{n}{\infty} f(n\delta) = 0 $, 证明 $ \limit{x}{+\infty} f(x) = 0 $.  
    \item 设 $ f(x) $ 在 $ \R $ 上一致连续, 则存在正实数 $ a, b $, 使得 $ \abs{f(x)} \le a\abs{x} + b $. 
    \item 设 $ f(x) $ 在 $ [1, +\infty) $ 上一致连续, 证明 $ \abs{f(x)/x} $ 在 $ [1, +\infty) $ 有界. 
    \item 证明欧式空间中两标准正交基的过渡矩阵为正交阵.
    \item 设 $ \alpha $ 是欧式空间 $ V $ 中的一个非零向量, $ \alpha_{1}, \alpha_{2}, \dots, \alpha_{p} $ 是 $ V $ 中的 $ p $ 个向量, 满足
    \begin{align*}
        (a_{i}, a_{j}) \le 0,\ (\alpha_{i}, \alpha) > 0, \quad i, j = 1, 2, \dots, p, i \ne j
    \end{align*}
    证明
    \begin{enumerate}
        \item $ \alpha_{1}, \alpha_{2}, \dots, \alpha_{p} $ 线性无关;
        \item $ n $ 维欧式空间中最多有 $ n + 1 $ 个向量, 使其两两互成钝角;
        \item $ n $ 维欧式空间中一定存在 $ n + 1 $ 个向量, 使其两两互为钝角.
    \end{enumerate}
    \item 设 $ A, B \in \MM[\K] $, 且 $ AB = BA $, 利用线性方程组的知识证明
    \begin{align*}
        \rank(A + B) \le \rank(A) + \rank(B) - \rank(AB)
    \end{align*}
    \item 设 $ B \in \MM[\C][n \times 2] $, 
    \begin{align*}
        C = \begin{pmatrix}
            1 & 1 & \dots & 1\\
            1 & 2 & \dots & n
        \end{pmatrix}
    \end{align*}
    若 $ A = BC $, 且 $ CB $ 的特征多项式为 $ x^{2} - 2x + 1 $, 求 $ A $ 的特征值, 并求 $ AX = 0 $ 的基础解系. 
    \item  计算 $ n $ 阶 $ b $ -- 循环行列式:
    \begin{align*}
        A = \begin{vmatrix}
            a_{1}   & a_{2} & a_{3} & \ldots    & a_{n} \\
            ba_{n}  & a_{1} & a_{2} & \ldots    & a_{n-1} \\
            ba_{n-1}    & ba_{n}    & a_{1} & \ldots & a_{n-1} \\
            \vdots  & \vdots & \vdots & \ddots & \vdots \\
            ba_{2} & ba_{3} & ba_{4} & \ldots & a_{1}
        \end{vmatrix}
    \end{align*}
    \sitem\label{item:反称加对角} 设 $ A $ 是 $ n $ 阶实反对称阵, $ D = \diag\qty{d_{1}, d_{2}, \dots, d_{n}} $ 是同阶的对角阵, 且 $ d_{i} > 0\,(i = 1, 2, \dots, n) $. 求证 $ \abs{A + D} > 0 $, 特别地, $ I_{n} + A $ 与 $ I_{n} - A $ 都是非异阵. 
    \item 如果 $ n $ 阶方阵 $ A = (a_{ij}) $ 适合条件:
    \begin{align*}
        \abs{a_{ii}} > \sum_{\mathclap{j = 1,\ j \ne i}}^{n} \abs{a_{ij}}, \quad i = 1, 2, \dots, n,
    \end{align*}
    则称 $ A $ 为\textbf{严格对角占优阵}, 求证, 严格对角占优阵必是满秩阵, 若上述条件改为:
    \begin{align*}
        a_{ii} > \sum_{\mathclap{j = 1,\ j \ne i}}^{n} \abs{a_{ij}}, \quad i = 1, 2, \dots, n,
    \end{align*}   
    求证 $ \abs{A} > 0 $. 
    \item 设 $ f(x) $ 在 $ [a, b] $ 上有定义, 对 $ [a, b] $ 上任意一个闭区间 $ [x_{1}, x_{2}] \subset [a, b] $, 对介于 $ f(x_{1}) $ 与 $ f(x_{2}) $ 之间的任一常数 $ l $, 方程 
    \begin{align*}
        f(x) = l
    \end{align*}
    在 $ [x_{1}, x_{2}] $ 上有且仅有有限个解, 证明 $ f(x) \in C[a, b] $.
    \item 设 $ f(x) $ 在 $ (0, +\infty) $ 上可导, 且 $ \limit{x}{+\infty} \qty(f(x) + f'(x)) = A $, 证明 $ \limit{x}{+\infty} f(x) = A $, 其中 $ A \in \R\cup{\pm\infty} $. 
    \item 已知 $ A \in \MM[\K] $, 且 $ \tr(A) = 0 $, 证明
    \begin{enumerate}
        \item 存在数域 $ \K $ 上的可逆阵 $ C $, 使得 $ C^{-1}AC $ 为主对角元全为 $ 0 $ 的矩阵.
        \item 存在 $ X, Y \in \MM[\K] $, 使得 $ XY - YX = A $.
        \item 令 $ U $ 为 $ \MM[\K] $ 中所有形如 $ XY - YX $ 的矩阵组成的集合, 证明 $ U $ 是 $ \MM[\K] $ 的一个线性子空间. 
    \end{enumerate}
    \item 求极限
    \begin{align*}
        \limit{n}{\infty} \qty(\frac{1}{\sqrt{n^{2} + 1}} + \frac{1}{\sqrt{n^{2} + 2}} + \dots + \frac{1}{\sqrt{n^{2} + n}})^{n}
    \end{align*}
    \item 设 $ \varphi $ 为 $ n $ 维线性空间 $ V $ 上的线性变换, $ W $ 是 $ \varphi $ 的不变子空间, 且 $ V = \Image \varphi \oplus W $, 证明
    \begin{align*}
        V = \Image \varphi \oplus \Ker \varphi.
    \end{align*}
    \item 设 $ A, B \in \MM[\C] $, 且 $ \rank(A) = \rank(B)  = 1,\ \tr(A) = \tr(B) $, 证明 $ A $ 相似于 $ B $.
    \item 设 $ \varphi $ 是 $ n $ 维线性空间 $ V $ 上的线性变换, 求证: 必存在正整数 $ m $, 使得
    \begin{align*}
        \Image \varphi^{m} = \Image \varphi^{m+1},\quad \Ker \varphi^{m} = \Ker \varphi^{m+1}, \quad V = \Image \varphi^{m} \oplus \Ker \varphi^{m+1}.
    \end{align*}
    \item 使用 Jordan 标准型证明迹非 $ 0 $ 的秩 1 矩阵可对角化.
    \item 设 $ A $ 是 $ n $ 阶实对称阵, 证明: $ A $ 可逆的充分必要条件为存在矩阵 $ B $, 使得 $ AB + B'A $ 正定.
    \item 设 $ A, B \in \MM[\C] $, 其中 $ A $ 是幂零阵, 且 $ AB = BA $, 求证: $ \abs{B} = \abs{A + B} $. 
    \item 设函数 $ f $ 在 $ x = 0 $ 连续, 并且
    \begin{align*}
        \limit{x}{0}\frac{f(2x) - f(x)}{x} = A,
    \end{align*}
    求证: $ f'(0) $ 存在, 且 $ f'(0) = A $. 
    \item 设 $ x_{n} $ 是 $ \tan x = x $ 在 $ (n\pi, n\pi + \pi/2) $ 上的解,
    \begin{enumerate}
        \item 求证 $ \limit{n}{\infty}(n\pi + \pi/2 - x_{n}) = 0 $,
        \item 求 $ \limit{n}{\infty} n(n\pi + \pi/2 - x_{n}) $.  
    \end{enumerate}
    \item 设 $ f $ 在 $ [0, +\infty) $ 上可微, 且 $ f(0) = 0 $, 并假设有实数 $ A $ 使得 $ \abs{f'(x)} \le A\abs{f(x)} $ 对 $ x \in (0, +\infty) $ 恒成立, 证明 $ f(x) \equiv 0\,(x \in (0, +\infty)) $.  
    \item 设偶函数 $ f(x) $ 在 $ x = 0 $ 处二阶可导, 且 $ f(0) = 1 $, 证明级数 $ \sum_{n = 1}^{\infty} (f(1/n) - 1) $ 绝对收敛.
    \item 设 $ f $ 在 $ [a, b] $ 上可导, 且 $ f' $ 在 $ [a, b] $ 上可积, $ f(a) = 0 $, 证明:
    \begin{align*}
        2\int_{a}^{b}(f(x))^{2} \dd{x} \le (b-a)^{2} \int_{a}^{b} (f'(x))^{2} \dd{x}.
    \end{align*}
    \item 设 $ f(x) $ 在 $ [0, +\infty) $ 上可微, 且存在实数 $ A > 0 $, 使得 $ \abs{f'(x)} \le A\abs{f(x)} $, 证明 $ f(x) \equiv 0 $ 对 $ x \in [0, +\infty) $ 均成立.
    \item 设 $ f(x) $ 在 $ [0, 1] $ 上有连续的二阶导数, $ f(0) = f(1) = 0 $, 且对任意的 $ x \in (0, 1) $, 都有 $ f(x) \ne 0 $, 证明
    \begin{align*}
        \int_{0}^{1} \abs{f''(x)\over f(x)} \dd{x} \ge 4
    \end{align*}
    \item 设 $ f(x) \in C^{2}[a, b] $, 证明: 存在 $ \xi \in (a, b) $ 使得
    \begin{align*}
        \int_{a}^{b} f(x) \dd{x} = (b - a) f\qty(\frac{a + b}{2}) + \frac{1}{24}(b - a)^{3} f''(\xi).
    \end{align*}
    \item 设 $ x_{1} $, $ x_{2} $, $ x_{3} $ 是多项式 $ f(x) = x^{3} + ax + 1 $ 的三个根, 求一个首一多项式以 $ x_{1}^{2} $, $ x_{2}^{2} $, $ x_{3}^{2} $ 为根.
    \item 设 $ f(x) $ 在有限区间 $ (a, b) $ 内可微, 且 $ f'(x) $ 在 $ (a, b) $ 内有界, 证明 $ f(x) $ 在 $ (a, b) $ 内有界.
    \item 设 $ f(x) $ 在 $ [a, b] $ 上连续, 且对任意 $ x_{1}, x_{2} \in [a, b] $, $ \lambda \in (0, 1) $, 恒有 $ f(\lambda x_{1} + (1-\lambda) x_{2}) \le \lambda f(x_{1}) + (1 - \lambda) f(x_{2}) $. 证明
    \begin{align*}
        f\qty(\frac{a+b}{2}) \le \frac{1}{b - a}\int_{a}^{b} f(x) \dd{x} \le \frac{f(a) + f(b)}{2}.
    \end{align*}
    \item 设 $ \limit{x}{+\infty} f(x) = A $, 且满足下列条件之一, 则有 $ \limit{x}{+\infty} f'(x) = 0 $.
    \begin{enumerate}
        \item $ f''(x) $ 在 $ (0, +\infty) $ 有界;
        \item $ \limit{x}{+\infty} f'(x) $ 存在.
    \end{enumerate}
    \item 广义积分 $ \int_{a}^{+\infty} f(x) \dd{x} $ 收敛, 加上下面任一条件即可推出 $ \limit{x}{+\infty} f(x) = 0 $:
    \begin{enumerate}
        \item $ \limit{x}{+\infty} f(x) $ 存在,
        \item $ \int_{a}^{+\infty} f'(x) \dd{x} $ 收敛,
        \item $ f(x) $ 单调, 这时有更强的结果: $ \limit{x}{+\infty} xf(x) = 0 $,
        \item $ f(x) $ 在 $ [a, +\infty) $ 上一致连续,
        \item $ f'(x) $ 在 $ [a, +\infty) $ 上有界. 
    \end{enumerate}
    \item 设 $ A $ 是三阶正交矩阵, 且 $ \abs{A} = 1 $, 证明存在正交阵 $ B $, 使得 $ A = B^{2} $.  
    \item 设函数 $ f(x) $ 在 $ \R $ 上有定义, 且在任何有限闭区间上可积, 证明: 对任何闭区间 $ [a, b] $, 有
    \begin{align*}
        \limit{h}{0} \int_{a}^{b} \abs{f(x + h) - f(x)} \dd{x} = 0.
    \end{align*}
    \item 设数列 $ \set{x_{n}} $ 满足 $ \set{2x_{n+1} + x_{n}} $ 收敛, 证明数列 $ \set{x_{n}} $ 收敛. 
    \item (\emph{Young 不等式})设 $ y=f(x) $ 是区间 $ [0, +\infty) $ 上严格递增的连续函数, 且满足 $ f(0) = 0 $, 证明对任意的 $ a, b > 0 $, 有
    \begin{align*}
        ab \le \int_{0}^{a}f(x) \dd{x} + \int_{0}^{b} f^{-1}(y) \dd{y}.
    \end{align*}
    \item 设 $ f, g \in C[a, b] $, $ g $ 在 $ [a, b] $ 上不变号, 证明存在 $ \xi\in (a, b) $, 使得
    \begin{align*}
        \int_{a}^{b} f(x)g(x) \dd{x} = f(\xi)\int_{a}^{b} g(x) \dd{x}.
    \end{align*}   
    \item 设 $ A $ 为 $ 3 $ 阶非零实矩阵, $ A^{T} = A^{*} $, 且 $ \abs{I + A} = \abs{I - A} = 0 $, 计算行列式 $ \abs{A^{2} - A - 3I} $.
    \item 设 $ f(x) $ 在 $ [a, b] $ 上单调, $ g(x) $ 是 $ \R $ 上以 $ T>0 $ 为周期的连续函数,且 $ \int_{0}^{T} g(x) \dd{x} = 0 $, 求
    \begin{align*}
        \limit{\lambda}{\infty} \int_{a}^{b} f(x)g(\lambda x) \dd{x}
    \end{align*}
    \item 设 $ f(x) $ 是实多项式, 且对任意实数 $ r $, 都有 $ f(r) \ge 0 $. 证明存在实多项式 $ g(x), h(x) $ 使得 $ f(x) = g^{2}(x) + h^{2}(x) $.   
    \item 设 $ \varphi_{1}, \varphi_{2}, \dots, \varphi_{m} $ 是 $ n $ 维线性空间 $ V $ 上的线性变换, 且适合条件:
    \begin{align*}
        \varphi_{i}^{2} = \varphi_{i}, \quad \varphi_{i}\varphi_{j} = 0\,(i\ne j), \quad \bigcap_{i=1}^{m}\Ker\varphi_{i} = 0,
    \end{align*}
    求证: $ V = \bigoplus_{i=1}^{m}\Image\varphi_{i} $. 
    \item 设 $ f $ 在 $ (0, 1] $ 上可导, 且 $ \limit{x}{0+} \sqrt{x} f'(x) = A \in \R $, 证明 $ f $ 在 $ (0, 1] $ 上一致连续. 
    \item 设 $ f(x) $ 在 $ [0, 1] $ 可积, $ f(1) = 0 $, $ f'(1) = a $, 证明
    \begin{align*}
        \limit{n}{\infty}\int_{0}^{1}n^{2}x^{n}f(x)\dd{x} = -a.
    \end{align*}
    \item 设 $ f(x), g(x) $ 是次数不小于 $ 1 $ 的互素多项式, 求证, 必唯一地存在两个多项式 $ u(x), v(x) $ 使得
    \begin{align*}
        f(x)u(x) + g(x)v(x) = 1,
    \end{align*}
    且 $ \deg v(x) < \deg f(x) $, $ \deg u(x) < \deg g(x) $.  
    \item 设 $ f(x) $ 是次数大于 $ 0 $ 的首一整系数多项式, 若 $ f(0), f(1) $ 都是奇数, 求证 $ f(x) $ 没有有理根.
    \item 设 $ f(x) $ 是次数大于 $ 1 $ 的奇数次有理系数多项式, 且它在有理数域上不可约, 求证: 若 $ x_{1}, x_{2} $ 是 $ f(x) $ 在复数域上的两个不同的根, 则 $ x_{1} + x_{2} $ 必不是有理数.
    \item 设 $ A $ 是实矩阵, 又 $ I_{n} - A $ 的特征值的模长都小于 $ 1 $, 求证: $ 0<\abs{A}<2^{n} $. 
    \item 设
    \begin{align*}
        \Q(\sqrt[n]{2}) = \set{a_{0} + a_{1}\sqrt[n]{2} + a_{2}\sqrt[n]{4} + \dots + a_{n-1}\sqrt[n]{2^{n-1}} : a_{i} \in \Q, 0\le i\le n-1}
    \end{align*}
    证明 $ \Q(\sqrt[n]{2}) $ 是一个数域, 并求 $ \Q(\sqrt[n]{2}) $ 做为 $ \Q $ 上线性空间的一组基.
    \item 设 $ f(x) = x^{n} + a_{1}x^{n-1} + \dots + a_{n-1}x + a_{n} $ 是数域 $ \K $ 上的不可约多项式, $ \varphi $ 是 $ \K $ 上的 $ n $ 维线性空间 $ V $ 上的线性变换, $ \alpha_{1} \ne 0, \alpha_{2}, \dots, \alpha_{n} $, 是 $ V $ 中的向量, 满足 
    \begin{align*}
        \varphi(\alpha_{i}) = \alpha_{i + 1}\,(i = 1, 2, \dots, n-1), \quad \varphi(\alpha_{n}) = -a_{n}\alpha_{1} - a_{n-1}\alpha_{2} - \dots - a_{1}\alpha_{n}.
    \end{align*}
    证明 $ \set{\alpha_{1}, \alpha_{2}, \dots, \alpha_{n}} $ 是 $ V $ 的一组基. 
    \item 设 $ A, B \in \MM $, 存在可逆复矩阵 $ P $, 使得 $ P^{-1}AP = B $, 证明存在可逆实矩阵 $ Q $ 使得 $ Q^{-1}AQ = B $.   
    \item 设 $ A, B $ 为 $ n $ 阶方阵, 满足 $ \rank(ABA) = \rank(A) $, 求证: $ AB $ 与 $ BA $ 相似.
    \item 设 $ A, B $ 为 $ n $ 阶方阵, 则 $ AB $ 与 $ BA $ 相似的充要条件是 $ \rank((AB)^{i}) = \rank((BA)^{i})\,(1 \le i \le n - 1) $.
    \item  设 $ A, B $ 为 $ n $ 阶方阵, 满足 $ \rank(ABA) = r(B) $, 求证: $ AB $ 与 $ BA $ 相似.
    \item 设 $ f $ 在 $ \R $ 上连续, 又 $ \varphi(x) = f(x)\int_{0}^{x} f(t) \dd{t} $ 单调递减, 证明 $ f(x) \equiv 0, x\in\R $.
    \item 计算积分
    \begin{align*}
        I = \int_{0}^{\pi/2} \frac{\sin x}{\sin x + \cos x} \dd{x}.
    \end{align*}
    \item 讨论广义积分
    \begin{align*}
        \int_{1}^{+\infty} \frac{\sin x}{x^{p} + \sin x} \dd{x}
    \end{align*}
    在何时绝对收敛或条件收敛.
    \item 设 $ f(x) $ 在 $ [a, +\infty) $ 上一阶连续可导, 且 $ x \to +\infty $ 时, $ f(x) $ 单调递减趋于 $ 0 $, 证明无穷积分 $ \int_{a}^{+\infty} f(x) \dd{x} $ 收敛当且仅当 $ \int_{a}^{+\infty} xf'(x) \dd{x} $ 收敛. 
    \item 设 $ \set{a_{n}} $ 是正数列, $ \liminf\limits_{n\to\infty} a_{n} = 1 $, $ \limsup\limits_{n\to\infty} a_{n} = A < +\infty $, 且 $ \limit{n}{\infty} \sqrt[n]{a_{1}a_{2}\dots a_{n}} = 1 $, 求证:
    \begin{align*}
        \limit{n}{\infty}\frac{a_{1} + a_{2} + \dots + a_{n}}{n} = 1.
    \end{align*}
    \item 设 $ \limit{n}{\infty} a_{n} = A $, 求 $ \limit{n}{\infty} \sum_{k=1}^{n} \frac{a_{n+k}}{n+k} $.  
    \item 设 $ f(x) \in C[1, +\infty) $, 对任意 $ x\in [1, +\infty) $, 有 $ f(x) > 0 $, 且 $ \limit{x}{+\infty} \ln(f(x))/\ln(x) = -\lambda $, 证明: $ \lambda > 1 $ 时 $ \int_{1}^{+\infty} f(x) \dd{x} $ 收敛.\footnote{数列形式下的该判别法称为\emph{对数判别法}}
    \item 设函数 $ f(x) $ 在 $ [0, 1] $ 上单调, 并且积分 $ \int_{0}^{1} f(x) \dd{x} $ 收敛, 证明:
    \begin{align*}
        \int_{0}^{1} f(x) \dd{x} = \limit{n}{\infty} \frac{1}{n} \sum_{k=1}^{n} f\qty(\frac{k}{n}).
    \end{align*}
    并且举反例说明``去掉单调条件, 结论则不成立.''
    \item 设 $ V $ 是 $ n $ 维线性空间, 对于整数 $ k \ge n $, 证明存在一组向量 $ \alpha_{1}, \alpha_{2}, \dots, \alpha_{k} \in V $, 使得其中任意 $ n $ 个线性无关.
    \item 设 $ \set{a_{n}} $ 是递减正数列, 证明: $ \sum_{n=1}^{\infty}a_{n} $ 与 $ \sum_{n=1}^{\infty}2^{n}a_{2^{n}} $ 同时敛散.
    \item 设对任意 $ n\in \N $, $ a_{n} > 0 $, 且级数 $ \sum_{n=1}^{\infty}1/a_{n} $ 收敛, 证明下述级数收敛 (利用绝对收敛函数重排不改变敛散性与级数值):
    \begin{align*}
        \sum_{n=1}^{\infty}\frac{n}{a_{1} + a_{2} + \dots + a_{n}}.
    \end{align*}
    \item 判断级数 $ \sum_{n=1}^{\infty} (-1)^{[\sqrt{n}]}/n^{p} $ 的敛散性.
    \item 设 $ f(x) $ 在 $ [-1, 1] $ 上二次连续可微, 且有 $ \limit{x}{0} f(x)/x = 0 $, 证明级数 $ \sum_{n=1}^{\infty}f(1/n) $ 绝对收敛.
    \item 已知 $ \sum_{n=1}^{\infty}(a_{n} - a_{n-1}) $ 绝对收敛, $ \sum_{n=1}^{\infty}b_{n} $ 收敛, 证明 $ \sum_{n=1}^{\infty}a_{n}b_{n} $ 收敛.
    \item 设 $ A, B \in \MM[\C] $, 若 $ AB = BA $, 则 $ A, B $ 至少有一个公共的特征向量. 
    \item 设 $ \varphi $ 是 $ n $ 维复线性空间 $ V $ 上的线性变换, 求证 $ \varphi $ 可对角化的充要条件是对 $ \varphi $ 的任一特征值 $ \lambda_{0} $, 总有 $ \Ker(\varphi - \lambda_{0}I) \cap \Image(\varphi - \lambda_{0}I) = 0 $. 
    \sitem 设在数域 $ \K $ 上, 一元多项式 $ f(x) = f_{1}f_{2} $, 且 $ (f_{1}, f_{2}) = 1 $, $ V $ 是数域 $ \K $ 上的 $ n $ 维线性空间, $ \varphi $ 是 $ V $ 上的线性变换, 证明 $ \Ker f(\varphi) = \Ker f_{1}(\varphi) \oplus \Ker f_{2}(\varphi) $. 
    \item 设 $ f(x) $ 在 $ [a, +\infty) $ 上可微, 且对任意 $ x \in [a, +\infty) $, 都有
    \begin{align*}
        f(x+1) - f(x) = f'(x)
    \end{align*}
    若 $ \limit{x}{+\infty}f'(x) = c $, 证明 $ f'(x) = c $ 在 $ [a, +\infty) $ 上恒成立.
    \item 设 $ a_{n} > 0 $, $ \sum_{n=1}^{\infty}a_{n} < +\infty $, 对于 $ \alpha, \beta > 0 $, 且 $ \alpha + \beta > 1 $, 证明 $ \sum_{n=1}^{\infty} a_{n}^{\alpha}/n^{\beta} < +\infty $. 
    \item 设 $ A \in \MM $, 对任意 $ 0\ne x \in \R^{n} $, 都有 $ x'Ax > 0 $, 利用 \ref{item:反称加对角} 题证明 $ \abs{A} > 0 $.
    \item 设有 $ n $ 阶分块对角阵 
    \begin{align*}
        A = \begin{pmatrix}
            A_{1} & & \\
            & \ddots & \\
            & & A_{k}
        \end{pmatrix}\quad 
        B = \begin{pmatrix}
            B_{1} & & \\
            & \ddots & \\
            & & B_{k}
        \end{pmatrix}
    \end{align*}
    其中 $ A_{i} $ 与 $ B_{i} $ 为同阶方阵, 假定矩阵 $ A_{i} $ 适合非零多项式 $ g_{i}(x) $, 且 $ g_{i}(x)\,(i = 1, \dots, k) $ 两两互素. 求证: 若对于每个 $ i $, 存在多项式 $ f_{i}(x) $, 使 $ B_{i} = f_{i}(A_{i}) $, 则必存在次数不超过 $ n-1 $ 的多项式 $ f(x) $, 使得 $ B = f(A) $.
    \item 设 $ n $ 阶方阵 $ A $ 的秩为 $ n-1 $, $ B $ 是同阶非零阵, 且有 $ AB = BA = 0 $, 证明: 存在不超过 $ n-1 $ 阶的多项式 $ f(x) $, 使得 $ B = f(A) $.
    \item 如 \ref{item:AX=XB} 题, 可以进一步证明逆命题也成立, 即: 如果 $ AX = XB $ 只有零解, 则 $ A, B $ 无公共特征值.
    \sitem 设 $ V $ 为数域 $ \K $ 上的 $ n $ 维线性空间, $ \varphi $ 是 $ V $ 上的线性变换, 其特征多项式与极小多项式分别设为 $ f(\lambda) $ 与 $ m(\lambda) $, 设
    \begin{align*}
        f(\lambda) = P_{1}(\lambda)^{r_{1}}P_{2}(\lambda)^{r_{2}}\dots P_{t}(\lambda)^{r_{t}}, \quad m(\lambda) = P_{1}(\lambda)^{s_{1}}P_{2}(\lambda)^{s_{2}}\dots P_{t}(\lambda)^{s_{t}}
    \end{align*}
    分别为 $ f(\lambda) $ 与 $ m(\lambda) $ 的不可约分解, 其中 $ P_{i}(\lambda) $ 为 $ \K $ 上互异的首一不可约多项式, $ r_{i}, s_{i} > 0\, (i = 1, 2, \dots, t) $. 设 $ V_{i} = \Ker P_{i}(\varphi)^{r_{i}} $, $ U_{i} = \Ker P_{i}(\varphi)^{s_{i}}\,(i = 1, 2, \dots, t) $. 求证:
    \begin{enumerate}
        \item $ V = V_{1} \oplus V_{2} \oplus \dots \oplus V_{t} $, $ U = U_{1} \oplus U_{2} \oplus \dots \oplus U_{t} $, 且 $ U_{i} = V_{i}\,(i = 1, 2, \dots, t) $;
        \item $ \varphi|_{V_{i}} $ 的特征多项式为 $ P_{i}(\lambda)^{r_{i}} $, 极小多项式为 $ P_{i}(\lambda)^{s_{i}} $. 特别地, $ \dim V_{i} = r_{i}\deg P_{i}(\lambda) $. 
    \end{enumerate}
    \item 证明任一 $ n $ 阶复矩阵 $ A $ 都相似于一个复对称阵.
    \item 设 $ A $ 为 $ n $ 阶实对称矩阵, 求证: $ A $ 为半正定阵或半负定阵的充要条件是对任意满足 $ \alpha' A\alpha = 0 $ 的 $ n $ 维实向量 $ \alpha $, 都有 $ A\alpha = 0 $.
    \item 设 $ f(x) $ 在 $ [a, b] $ 上连续, $ (a, b) $ 上可导, 且 $ f(a) = f(b) $, 若 $ \abs{f'(x)} \le 1 $, 证明对任意的 $ x_{1}, x_{2} \in [a, b] $, 都有
    \begin{align*}
        \abs{f(x_{1}) - f(x_{2})} \le \frac{(b - a)}{2}.
    \end{align*}
    \item 设 $ f(x) $ 在 $ [0, 1] $ 上连续, 且 $ f(1) = 0 $, 证明 $ \set{f(x)x^{n}} $ 在 $ [0, 1] $ 上一致收敛.
    \item 设 $ p(x), q(x), r(x) $ 是数域 $ \K $ 上的正次数多项式, 且 $ p(x) $ 与 $ q(x) $ 互素, $ \deg r(x) < \deg p(x) + \deg q(x) $, 证明存在数域 $ \K $ 上的多项式 $ u(x), v(x) $, 满足 $ \deg u(x) < \deg p(x) $, $ \deg v(x) < \deg q(x) $, 使得 $ r(x) = p(x)v(x) + q(x)u(x) $.
    \item (\emph{Dini 定理}) 设函数列 $ \set{f_{n}(x)} $ 在有限闭区间 $ [a, b] $ 上连续. 如果对每一个 $ x\in [a, b] $, 数列 $ \set{f_{n}(x)} $ 关于 $ n $ 递减趋于 $ 0 $. 那么 $ f_{n}(x) $ 在 $ [a, b] $ 上一致收敛于 $ 0 $.
    \item 对任意 $ n\in \N $, $ f_{n}(x) $ 在 $ [a, b] $ 上关于 $ x $ 单调递增, 且 $ \set{f_{n}(x)} $ 收敛于连续函数 $ f(x) $. 证明: $ \set{f_{n}(x)} $ 在 $ [a, b] $ 上一致收敛于 $ f(x) $.
    \item 设 $ f(x) $ 在 $ [a, b] $ 上可导, 且 $ f'(x) $ 在 $ [a, b] $ 上可积, 记 
    \begin{align*}
        A_{n} = \frac{b - a}{n}\sum_{i = 1}^{n} f\qty(a + \frac{i}{n}(b - a)) - \int_{a}^{b} f(x) \dd{x},
    \end{align*}
    证明 $ \limit{n}{\infty} nA_{n} = (b - a)(f(b) - f(a))/2 $.
    \item 设 $ V $ 是数域 $ \K $ 上的 $ n $ 维线性空间, $ \sigma, \tau $ 是 $ V $ 上的线性变换, 且 $ \sigma^{2} = \tau^{2} = 0 $, 且 $ \sigma\tau + \tau\sigma = I_{V} $, 其中 $ I_{V} $ 是 $ V $ 上的恒等变换, 证明
    \begin{enumerate}
        \item $ V = \Ker \sigma \oplus \Ker \tau $;
        \item $ V $ 必是偶数维线性空间.
    \end{enumerate}
    \item 设函数 $ f(x) \in C[a, b] $, $ f(x) $ 不恒为 $ 0 $ 并且满足 $ 0 \le f(x) \le M $. 证明:
    \begin{align*}
        \qty(\int_{a}^{b} f(x) \cos x \dd{x})^{2} + \qty(\int_{a}^{b} f(x) \sin x \dd{x})^{2} + \frac{M^{2}(b-a)^{4}}{12} \ge \qty(\int_{a}^{b} f(x) \dd{x})^{2}.
    \end{align*}
    \item 计算极限 $ \limit{\lambda}{\infty} \int_{0}^{1} \ln x\cos^{2}(\lambda x) \dd{x} $.
    \sitem 设 $ A, B \in \MM[\K][m\times n] $, 求证: 方程组 $ Ax = 0 $ 与 $ Bx = 0 $ 同解的充分必要条件是存在可逆阵 $ P $, 使得 $ B = PA $.
    \item 计算行列式
    \begin{align*}
        D = \begin{vNiceMatrix}[cell-space-limits = 1pt]
            1 & 0 & 0 & \cdots & 0 & 1 \\
            1 & {1}\choose{1} & 0 & \cdots & 0 & x\\
            1 & {2}\choose{1} & {2}\choose{2} & \cdots & 0 & x^{2} \\
            \vdots & \vdots & \vdots & \ddots & \vdots & \vdots\\
            1 & {n-1}\choose{1} & {n-1}\choose{2} & \cdots & {n-1}\choose{n-1} & x^{n-1}\\
            1 & {n}\choose{1} & {n}\choose{2} & \cdots & {n}\choose{n-1} & x^{n}
        \end{vNiceMatrix}
    \end{align*}
    \item 设 $ f(x) $ 是定义在 $ [0, 1] $ 上的单调非增函数, 对于任意 $ a \in (0, 1) $, 证明: $ \int_{0}^{a}f(x) \dd{x} \ge a\int_{0}^{1}f(x) \dd{x} $.
    \item 设 $ f(x) $ 在 $ [0, 1] $ 上 Riemann 可积, 且有 $ 0 < m \le f(x) \le M $. 证明:
    \begin{align*}
        1 \le \int_{0}^{1}f(x) \dd{x} \int_{0}^{1} \frac{1}{f(x)} \dd{x} \le \frac{(M+m)^{2}}{4mM}.
    \end{align*}
    \item 设开集 $ D\subset \R^{2} $, $ f: D \to \R $ , 如果 $ \pdv{f}{x}, \pdv{f}{y}, \pdv{f}{x}{y} $ 在 $ (\vb*{x}_{0}, \vb*{y}_{0}) $ 的某个邻域上存在, 且 $ \pdv{f}{x}{y} $ 在 $ (\vb*{x}_{0}, \vb*{y}_{0}) $ 处连续, 那么 $ \pdv{f}{y}{x} $ 在 $ (\vb*{x}_{0}, \vb*{y}_{0}) $ 处存在, 且
    \begin{align*}
        \pdv{f}{x}{y}{} (\vb*{x}_{0}, \vb*{y}_{0}) = \pdv{f}{y}{x}{} (\vb*{x}_{0}, \vb*{y}_{0}).
    \end{align*}
    \item 设 $ D \subset \R^{2} $ 是一个凸区域, $ f: D \to \R $ 有连续的一阶偏导数, 则 $ f $ 在 $ D $ 内为凸函数的充必条件为对任意 $ \vb*{x}, \vb*{y}\in D $, 有 $ f(\vb*{y}) \ge f(\vb*{x}) + (\vb*{y} - \vb*{x})\cdot\grad f(\vb*{x}) $. 
    \item 设 $ \symscr{A} $ 为数域 $ \K $ 上的 $ n\,(n \ge 3) $ 维线性空间 $ V $ 上的线性变换, $ \symscr{A} $ 的特征多项式为 
    \begin{align*}
        f(\lambda) = \lambda^{n} + a_{n-1}\lambda^{n-1} + a_{n-2}\lambda^{n-2} + \dots + a_{1}\lambda + a_{0},
    \end{align*}
    试证明:
    \begin{align*}
        a_{n - 2} = \frac{1}{2}\qty(\tr^{2}(\symscr{A}) - \tr(\symscr{A}^{2})).
    \end{align*}
    \item 设 $ a\ne 0 $, 计算积分: 
    \begin{align*}
        \int_{0}^{+\infty}\frac{\dd{x}}{(1+x^{2})(1+x^{a})}.
    \end{align*}
    \item 设 $ A $ 是 $ n $ 阶实正定阵, $ x \in \R^{n} $ 是非零列向量, 求证:
    \begin{enumerate}
        \item $ A + xx' $ 可逆.
        \item $ 0 < x'(A+xx')^{-1}x < 1 $. 
    \end{enumerate}
    其中 $ \lambda_{1}, \lambda_{2}, \dots, \lambda_{n} $ 为 $ A^{-1}B $ 的全体特征值.
    \item 设 $ A $ 是 $ n $ 阶半正定的实对称阵, $ S $ 为 $ n $ 阶实反对称阵, 满足 $ AS + SA = 0 $, 证明 $ \abs{A + S} > 0 $ 的充分必要条件为 $ \rank(A) + \rank(S) = n $. 
    \item 设 $ A, B $ 为 $ n $ 阶正定实对称阵, 若 $ A - B $ 半正定, 证明 $ B^{-1} - A^{-1} $ 为半正定阵.
    \item 设 $ A $ 是 $ n $ 阶正定实对称阵, $ B $ 是同阶半正定实对称阵, 求证 $ \abs{A + B} \ge \abs{A} + \abs{B} $. 
    \item 设 $ V $ 是 $ n $ 维酉空间, $ \varphi $ 是 $ V $ 上的线性变换, 求证: $ \varphi $ 是正规算子的充分必要条件是 $ \norm{\varphi(\alpha)} = \norm{\varphi^{*}(\alpha)} $ 对任意 $ \alpha \in V $ 都成立.
    \sitem 设 $ V $ 是 $ n $ 维酉空间, $ \varphi $ 是 $ V $ 上的线性变换, 求证: $ \varphi $ 是正规算子的充分必要条件是若 $ v $ 是 $ \varphi $ 属于特征值 $ \lambda $ 的特征向量, 则 $ v $ 也是 $ \varphi^{*} $ 属于特征值 $ \bar\lambda $ 的特征向量.
    \sitem 设 $ A = (a_{ij}) $ 是 $ n $ 阶复矩阵, $ \lambda_{1}, \lambda_{2}, \dots, \lambda_{n} $ 是其特征值, 求证: $ A $ 是正规矩阵的充分必要条件是
    \begin{align*}
        \sum_{i=1}^{n}\abs{\lambda_{i}}^{2} = \tr(A^{\symrm{H}}A) = \sum_{i, j=1}^{n}\abs{a_{ij}}^{2}.
    \end{align*}
\end{enumerate} 
    \begin{tcolorbox}[title={同时上三角化/对角化/标准型化}]
        \begin{enumerate}[resume=exer]
            \item 设 $ n $ 阶矩阵 $ \set{A_{i} : i = 1, 2, \dots, m} $ 两两可交换, 即 $ A_{i}A_{j} = A_{j}A_{i} $ 对一切 $ i, j $ 都成立, 假定每一个 $ A_{i} $ 均可对角化, 证明: 它们可同时对角化.
            \item 若 $ A, B \in \MM[\K] $, 且 $ AB = BA $, 假定 $ A, B $ 的特征值都在 $ \K $ 中, 证明: 存在 $ \K $ 上的可逆阵 $ P $, 使得 $ P^{-1}AP $ 与 $ P^{-1}BP $ 都是上三角矩阵.
            \item 设 $ A $ 为 $ n $ 阶正定实对称阵, $ B $ 为同阶对称阵, 则存在可逆阵 $ C $ 使得
            \begin{align*}
                C'AC = I_{n}, \quad C'BC = \diag\set{\lambda_{1}, \lambda_{2}, \dots, \lambda_{n}}
            \end{align*}
            其中 $ \lambda_{1}, \lambda_{2}, \dots, \lambda_{n} $ 是 $ A^{-1}B $ 的特征值.
            \item 设 $ A, B $ 都是 $ n $ 阶半正定实对称矩阵, 证明: 存在可逆阵 $ C $, 使得
            \begin{align*}
                C'AC = \diag\{\underbrace{1, \dots, 1}_{r\text{ 个 }}, 0, \dots, 0\}, \quad C'BC = \diag\set{\lambda_{1}, \dots, \lambda_{r}, \lambda_{r+1}, \dots, \lambda_{n}}
            \end{align*}
            \item 设 $ A $ 为 $ n $ 阶正定实对称阵, $ S $ 为同阶实反对称阵, 则存在可逆矩阵 $ C $, 使得
            \begin{align*}
                C'AC = I_{n}, \quad C'SC = \diag\set{\mqty(0 & b_{1} \\ -b_{1} & 0), \dots, \mqty(0 & b_{r} \\ -b_{r} & 0), 0, \dots, 0}
            \end{align*}
            \item 设 $ A_{i}\,(i = 1, 2, \dots, m) $ 是 $ m $ 个实对称(复正规)矩阵且两两可交换, 证明: 存在正交(酉)矩阵 $ P $, 使得 $ P'A_{i}P $ ($ P^{\symrm{H}}A_{i}P $)都是对角阵.
            \item 设 $ A, B $ 是两个 $ n $ 阶实正规矩阵, 且 $ AB = BA $, 证明: 存在正交矩阵 $ P $, 使得 $ P'AP $ 和 $ P'BP $ 同时为如下形状的分块对角矩阵: $ \diag\set{A_{1}, \dots, A_{r}, c_{2r+1}, \dots, c_{n}} $, 其中 $ c_{i} $ 是实数, $ A_{i} $ 为形如 $ \smqty(a_{i} & b_{i} \\ -b_{i} & a_{i}) $ 的二阶实矩阵. 
            \item 设 $ A, B $ 是 $ n $ 阶实对称矩阵, 满足 $ AB + BA = 0 $, 证明: 若 $ A $ 半正定, 则存在正交矩阵 $ P $, 使得:
            \begin{align*}
                P'AP = \diag\set{\lambda_{1}, \dots, \lambda_{r}, 0, \dots, 0}, \quad P'BP = \diag\set{0, \dots, 0, \mu_{r + 1}, \dots, \mu_{n}}.
            \end{align*}
        \end{enumerate}
    \end{tcolorbox}
    \begin{tcolorbox}[title={矩阵/线性算子分解}]
        \begin{enumerate}[resume=exer]
            \item 设 $ A \in \MM[\C] $, 则 $ A $ 可以分解为 $ A = B + C $, 其中 $ B $ 为 Hermite 阵, $ C $ 为斜 Hermite 阵.
            \item 设 $ A\in \MM[\C] $, 则 $ A $ 可以分解成两个对角阵的乘积, 即 $ A = BC $, 且可以任意指定 $ B $ 或 $ C $ 为可逆阵.
            \item 设 $ A $ 是 $ n $ 阶(半)正定实对称阵, 则
            \begin{enumerate}
                \item 存在主对角线上元素全等于 $ 1 $ 的上三角矩阵 $ B $, 使得 $ A = B'DB $, 其中 $ D $ 是(半)正定对角矩阵;
                \item (\emph{Cholesky 分解})存在主对角线上元素全为正(非负)的上三角阵 $ C $, 使得 $ A = C'C $. 
            \end{enumerate}
            \item (\emph{QR 分解}) 设 $ A $ 是 $ n $ 阶实(复)矩阵, 则 $ A $ 可以分解为 $ A = QR $, 其中 $ Q $ 是正交(酉)矩阵, $ R $ 是一个上三角阵, 且主对角线上的元素非负, 若 $ A $ 可逆, 则这样的分解唯一.
            \item (\emph{极分解}) 设 $ V $ 是 $ n $ 维酉(欧式)空间, $ \varphi $ 是 $ V $ 上的任一线性算子, 则存在 $ V $ 上的酉(正交)算子 $ \omega $ 以及 $ V $ 上的半正定自伴随算子 $ \psi $, 使得 $ \varphi = \omega\psi $, 其中 $ \psi $ 是唯一的, 并且若 $ \varphi $ 是非异线性算子, 则 $ \omega $ 也唯一 \footnote{ $ \varphi $ 也可以做这样的分解: $ \varphi = \psi_{1}\omega_{1} $ 其中 $ \omega_{1} $ 为酉(正交)算子, $ \psi_{1} $ 为半正定自伴随算子, 这样的分解也叫极分解, 下面矩阵版本的同理.}. 
            \item \label{item:矩阵极分解}(\emph{极分解}) 设 $ A \in \MM $, 则存在 $ n $ 阶正交阵 $ Q $ 以及 $ n $ 阶半正定对称阵 $ S $, 使得 $ A = QS $. 又设 $ B \in \MM[\C] $, 则存在 $ n $ 阶酉矩阵 $ U $, 以及 $ n $ 阶半正定 Hermite 阵 $ H $, 使得 $ B = UH $, 上述分解式当 $ A, B $ 为非异阵的时候被唯一确定.
            \item (\emph{谱分解}) 设 $ V $ 是 $ n $ 维欧式(酉)空间, $ \varphi $ 为 $ V $ 上的自伴随(正规)算子. $ \lambda_{1}, \lambda_{2}, \dots, \lambda_{k} $ 为 $ \varphi $ 的全体不同特征值, $ W_{i} $ 为属于 $ \lambda_{i} $ 的特征子空间, 则 $ V $ 是 $ W_{i}\,(i = 1, 2, \dots, k) $ 的正交直和. 这时若设 $ E_{i} $ 是 $ V $ 到 $ W_{i} $ 的正交投影, 则 $ \varphi $ 有如下分解式:
            \begin{align*}
                \varphi = \lambda_{1}E_{1} + \lambda_{2}E_{2} + \dots + \lambda_{k}E_{k}.
            \end{align*}
            \item (\emph{奇异值分解}) 设 $ V, U $ 分别为 $ n, m $ 维欧式(酉)空间, $ \varphi : V \to U $ 是线性映射, 则存在 $ V $ 和 $ U $ 的标准正交基, 使 $ \varphi $ 在这两组基下的表示矩阵为 $ \smqty(S & 0 \\ 0 & 0) $, 其中 $ S = \diag\set{\sigma_{1}, \sigma_{2}, \dots, \sigma_{r}} $, $ \sigma_{1} \ge \sigma_{2} \ge \dots \ge \sigma_{r} > 0 $ 是 $ \varphi $ 的非零奇异值.
            \item (\emph{奇异值分解}) 设 $ A $ 是 $ m\times n $ 的实(复)矩阵, 则存在 $ m $ 阶正交(酉)矩阵 $ P $, $ n $ 阶正交(酉)矩阵 Q, 使得 $ A = P\smqty(S & 0 \\ 0 & 0)Q $, 其中 $ S = \diag\set{\sigma_{1}, \sigma_{2}, \dots, \sigma_{r}} $, $ \sigma_{1} \ge \sigma_{2} \ge \dots \ge \sigma_{r} > 0 $ 是 $ A $ 的非零奇异值.
        \end{enumerate}
    \end{tcolorbox}
    \begin{enumerate}[resume=exer]
        \item 利用 \ref{item:矩阵极分解} 题证明: 存在正交阵 $ Q_{1}, Q_{2} $, 使得 $ Q_{1} A Q_{2} = \diag(\lambda_{1}, \lambda_{2}, \dots, \lambda_{n}) $, 并且 $ \lambda_{1}^{2}, \lambda_{2}^{2}, \dots, \lambda_{n}^{2} $ 是 $ A'A $ 的特征值. 
        \item 利用 $ \cos px $ 在 $ [-\pi, \pi] $ 上的 Fourier 展开证明积分 
        \begin{align*}
            \int_{0}^{+\infty} \frac{x^{p-1}}{1+x} \dd{x} = \frac{\pi}{\sin p\pi}\quad (0<p<1).
        \end{align*}
        \item 判断含参变量反常积分 $ \int_{0}^{+\infty}\sqrt{u}\me^{-ux^{2}} \dd{x} $ 对于 $ u \in [0, +\infty) $ 的一致收敛性. 
        \item 计算积分
        \begin{align*}
            I(r) = \int_{0}^{\pi} \ln(1 - 2r\cos x + r^{2}) \dd{x}
        \end{align*}
        \item 设 $ f(x)\in C^{1}[a, b] $, 记
        \begin{align*}
            A = \frac{1}{b - a}\int_{a}^{b} f(x) \dd{x}
        \end{align*}
        证明:
        \begin{align*}
            \int_{a}^{b}(f(x) - A)^{2} \dd{x} \le (b-a)^{2} \int_{a}^{b} (f'(x))^{2} \dd{x}.
        \end{align*}
        \item 计算积分 $ \int_{0}^{\pi/2} \ln(a^{2}\sin^{2}x + b^{2}\cos^{2} x)\dd{x}\,(a^{2} + b^{2} \ne 0) $.
        \item 已知 $ f(x) \in C[-1, 1] $, 证明:
        \begin{align*}
            \limit{y}{0+} \int_{-1}^{1}\frac{yf(x)}{x^{2} + y^{2}} \dd{x} = \pi f(0).
        \end{align*}
        \item 设 $ A $ 是 $ n $ 阶实对称阵:
        \begin{align*}
            A = \begin{pmatrix}
                a_{1} & b_{1} & & & \\
                b_{1} & a_{2} & b_{2} & & \\
                & b_{2} & a_{3} & \ddots & \\
                & & \ddots & \ddots & b_{n-1} \\
                & & & b_{n-1} & a_{n}
            \end{pmatrix},\quad b_{j} \ne 0
        \end{align*}
        证明: 
        \begin{enumerate}
            \item $ \rank(A) \ge n-1 $;
            \item $ A $ 的特征值各不相同. 
        \end{enumerate}
        \item 设 $ \lambda_{1}, \lambda_{2}, \dots, \lambda_{k} $ 是 $ A\in \MM[\C] $ 的全体特征值, 若有 $ \rank(\lambda_{i}I - A) = \rank(\lambda_{i}I - A)^{2}\,(i = 1, 2, \dots, k) $, 证明 $ A $ 可以相似对角化.
        \item 证明
        \begin{align*}
            \int_{1}^{+\infty}\exp\qty(-\frac{1}{\alpha^{2}}\qty(x-\frac{1}{\alpha})^{2}) \dd{x}
        \end{align*}
        在 $ (0, 1) $ 上一致收敛.
        \item 设 $ f(x) $ 为单调递减的正值函数, 证明 $ \int_{a}^{+\infty} f(x) \dd{x} $ 与 $ \int_{a}^{+\infty} f(x)\sin^{2}{x} \dd{x} $ 同时敛散.
        \item 设 $ \symscr{A}, \symscr{B} $ 均是 $ n $ 维线性空间 $ V $ 上的线性变换, 且 $ \Ker \symscr{A} \subset \Ker \symscr{B} $, 证明, 存在线性变换 $ \symscr{T} $, 使得 $ \symscr{B} = \symscr{TA} $.
        \item 计算极限
        \begin{align*}
            \limit{r}{+\infty} r\int_{0}^{+\infty} \me^{-x^{2}}\sin(rx) \dd{x}.
        \end{align*}
        \item 设数域 $ \K $ 上的全体矩阵 $ \MM[\K] $ 上有线性变换 $ \sigma(X) = AX - XA $, 其中 $ A \in \MM[\K] $:
        \begin{enumerate}
            \item 若 $ A $ 为幂零阵, 证明 $ \sigma $ 为幂零变换;
            \item 若 $ A $ 有特征值 $ \lambda_{1}, \lambda_{2}, \dots, \lambda_{n} $, 证明 $ \lambda_{i} - \lambda_{j}\,(1\le i, j\le n) $ 为 $ \sigma $ 的特征值.
        \end{enumerate}
        \item 设 $ f(x) $ 是实系数多项式, 若有 $ f(x) \mid f(x^2 + x + 1) $, 证明 $ 2 \mid \deg f(x) $.
        \item 设函数 $ f(x) $ 在 $ [0, 1] $ 上连续, 证明:
        \begin{align*}
            \int_{0}^{1} \dd{x}\int_{x}^{1} \dd{y}\int_{x}^{y}f(x)f(y)f(z)\dd{z} = \frac{1}{6}\qty(\int_{0}^{1}f(t) \dd{t})^{3}.
        \end{align*}
        \item 设 $ A, B $ 都是 $ n $ 阶正交阵, 证明 $ \abs{\det(A+B)} \le 2^{n} $.
        \item 设 $ A, B \in \MM[\C] $, 若 $ AB = BA = 0 $, $ \rank(A^{2}) = \rank(A) $, 证明
        \begin{align*}
            \rank(A+B) = \rank(A) + \rank(B).
        \end{align*}
        \item 设 $ V $ 是 $ n $ 维内积空间,  $ V_{1}, V_{2}, \dots, V_{r} $ 是 $ V $ 的 $ r $ 个真子空间, 证明: 存在 $ V $ 的一组标准正交基 $ \alpha_{1}, \alpha_{2}, \dots, \alpha_{n} $, 使得对任意的 $ 1 \le i \le n, 1\le j \le r $ 都有 $ \alpha_{i} \notin V_{j} $. 
        \item 设 $ f(x) $ 在 $ \R $ 上连续且有界, 证明对任意 $ T > 0 $, 都存在一个数列 $ \set{x_{n}} $, 满足
        \begin{align*}
            \limit{n}{\infty} x_{n} = +\infty, \quad \limit{n}{\infty}(f(x_{n}+T) - f(x_{n})) = 0.
        \end{align*}
        \item (华师 2020) 计算级数 $ \sum_{n=0}^{\infty}\frac{(-1)^{n}}{3^{n}(2n+1)} $.
        \item (华师 2020) 计算
        \begin{align*}
            \oiint_{\Sigma} (z^{2}+x) \dd{y}\dd{z} + \sqrt{z} \dd{x}\dd{y}
        \end{align*}
        其中 $ \Sigma $ 为抛物面 $ z = (x^{2}+y^{2})/2 $ 在平面 $ x=0 $ 与 $ x=2 $ 之间的部分, 方向取下侧.
        \item 计算第二型曲线积分
        \begin{align*}
            \oint_{L} \frac{4x-y}{4x^{2}+y^{2}}\dd{x} + \frac{x-y}{4x^{2}+y^{2}}\dd{y}
        \end{align*}
        其中 $ L $ 为 $ x^{2} + y^{2} = 2 $ 的圆周, 方向为逆时针.
        \item (浙大 2018) 设 $ A $ 为 $ n $ 阶非零实方阵, 且 $ A^{2} = A $, 设 $ \rank(A) = r $, 求证 $ A $ 正交相似于分块矩阵: $ \smqty(I_{r} & 0 \\ B & 0) $, 其中 $ I_{r} $ 为 $ r $ 阶单位阵, $ B $ 为某一实矩阵.
        \item 求抛物面 $ x^{2} + y^{2} + az = 4a^{2} $ 将球体 $ x^{2} + y^{2} + z^{2} \le 4az $ 分成两部分的体积之比.
        \item 计算第一型曲面积分
        \begin{align*}
            I = \iint_{S} \frac{x^{3} + y^{3} + z^{3}}{1-z} \dd{S}
        \end{align*}
        其中 $ S $ 为 $ x^{2} + y^{2} = (1-z)^{2}\,(0 \le z \le 1) $. 
        \item 计算 $ I = \iiint_{\Omega} x^{2}\sqrt{x^{2} + y^{2}}\,\dd x\dd y\dd z $, 其中 $ \Omega $ 是曲面 $ z = \sqrt{x^{2} + y^{2}} $ 与 $ z = x^{2} + y^{2} $ 围成的有界区域.
        \item 计算第二型曲面积分
        \begin{align*}
            \iint_{\Sigma} x\,\dd y\dd z + (2 + y^{3})\,\dd z\dd x + z^{3}\,\dd x\dd y
        \end{align*}
        其中 $ \Sigma $ 为 $ x = \sqrt{1 - 2y^{2} - 3z^{2}} $, 方向取 $ x $ 轴正向.
        \item 计算积分 $ I = \iint_{S} (x + z) \dd{\sigma} $, 其中 $ S $ 是曲面 $ x^{2} + z^{2} = 2az\,(a > 0) $ 被曲面 $ z = \sqrt{x^{2} + y^{2}} $ 所截取的有限部分.
        \item 设 $ A, B \in \MM $ 为对称阵, 若 $ A $ 正定, 证明 $ AB $ 的特征值全为实数.
    \end{enumerate}
\end{document}